\section{Auswertung}
\label{sec:Auswertung}




  \subsection{Eichung des Elektromagneten}
  In Tabelle \ref{tab:mit} sind die eingestellten Stromstärken $I$ und die mit einer Hallsonde gemessenen zugehörigen Magnetfeldstärken $B$ aufgelistet.
  
  \begin{table}[H] 
	\centering
	\caption{Für die Eichung des Magneten aufgenommene Messwerte des Magnetfeldes neben der zugehörigen Stromstärke} 
	\begin{tabular}{c|c}

Strom I / A & Magnetische Flussdichte B \\ 
\hline 
1	& 60 \\
2	& 123 \\
4	& 236\\
6	& 330\\
8	& 460\\
10	& 580\\
12	& 700\\
14	& 800\\
16	& 890\\
18	& 960\\
20	& 1020\\	
		
	\end{tabular} 
	  \label{tab:mit}
\end{table} 

\begin{figure}[h]
	\centering
	\includegraphics[width=12cm,height=8cm]{plot/V27_plot_1.pdf}
	\caption{Magnetischer Flussdichte gegen Stromstärke aufgetragen}
	\label{plot:1}
\end{figure}

Mit den Messwerten aus Tabelle \ref{tab:mit} wird eine Ausgleichsrechnung mittels der Funktion 
\begin{align}
f(x)= ax+b
\end{align}

durchgeführt. Für die Parameter erhält man durch den Fit:

\begin{center}
a = 52,4581 \pm 1,512 $\frac{mT}{A}$
\\
b = 30,5592 \pm 17,9 $mT$

\end{center}    



  \subsection{\texorpdfstring{Aufl"oseverm"ogen $A$ und Dispersionsgebiet $\Delta \lambda$ der verwendeten Lummer-Gehrcke-Platte f"ur die Wellenl"angen des roten und blauen "Ubergangs}{Aufl"oseverm"ogen A und Dispersionsgebiet Delta lambda der verwendeten Lummer-Gehrcke-Platte f"ur die Wellenl"angen des roten und blauen "Ubergangs}}
  
  Zur Berechnung sind die wesentlichen Größen der Versuchsanleitung gegeben
  
  \begin{center}
  L = 120 mm
  d = 4 m
  n(644nm) = 1,4567
  n(480nm) = 1,4635
  \end{center}
  
    \begin{table}[H] 
	\centering
	\caption{Für die Eichung des Magneten aufgenommene Messwerte des Magnetfeldes neben der zugehörigen Stromstärke} 
	\begin{tabular}{c|c|c}

  & rot & blau\\ 
\hline 
$\lambda$	 		& 643,8 nm & 480 nm \\
$\Delta \lambda $	& 48,91 pm & 26,95 pm \\
A				& 209129 & 285458 \\

		
	\end{tabular} 
	  \label{tab:mit}
\end{table} 
  
    \subsection{\texorpdfstring{Bestimmung des Lande-Faktors $g_{12}$ der $\sigma$-"Uberg"ange f"ur die rote Linie}{Bestimmung des Lande-Faktors g_{12} der sigma-"Uberg"ange f"ur die rote Linie}}
  
      \begin{table}[H] 
	\centering
	\caption{Für die Eichung des Magneten aufgenommene Messwerte des Magnetfeldes neben der zugehörigen Stromstärke} 
	\begin{tabular}{c|c|c|c}

  & $\Delta$ s in px & $\delta s$ in px & $\delta \lambda$ in $10^{-12} m$\\
  \hline 
1 &479&214&10,92 \pm 0,17 \\
2 &339&148&10,68 \pm 0,24 \\
3 &276&131&11,61 \pm0,29 \\
4 &238&104&10,69 \pm0,34 \\
5 &215&95&10,81 \pm 0,37 \\
6 &198&75&9,26 \pm 0,40 \\
7 &174&64&8,99 \pm 0,45 \\
8 &157&60&9,35 \pm 0,50 \\

		
	\end{tabular} 
	  \label{tab:mit}
\end{table} 

\begin{figure}[h]
	\centering
	\includegraphics[width=16cm,height=12cm]{Fotos/V27_1.jpg}
	\caption{Magnetischer Flussdichte gegen Stromstärke aufgetragen}
	\label{plot:1}
\end{figure}

\begin{center}
$\delta \lambda$ = (10,29 \pm 0,21) pm
\end{center}

  \subsection{\texorpdfstring{Bestimmung des Lande-Faktors $g_{12}$ der $\pi$-"Uberg"ange f"ur die blaue Linie}{Bestimmung des Lande-Faktors g_12 der pi-"Uberg"ange f"ur die blaue Linie}}
  
        \begin{table}[H] 
	\centering
	\caption{Für die Eichung des Magneten aufgenommene Messwerte des Magnetfeldes neben der zugehörigen Stromstärke} 
	\begin{tabular}{c|c|c|c}

  & $\Delta$ s in px & $\delta s$ in px & $\delta \lambda$ in $10^{-12} m$\\
  \hline 
1&188&92&6,59 \pm 0,24 \\
2&176&84&6,43 \pm 0,25 \\
3&160&72&6,06 \pm 0,28 \\
4&147&63&5,77 \pm 0,30 \\
5&131&58&5,97 \pm 0,34 \\
6&122&52&5,74 \pm 0,36 \\

		
	\end{tabular} 
	  \label{tab:mit}
\end{table} 

\begin{figure}[h]
	\centering
	\includegraphics[width=16cm,height=12cm]{Fotos/V27_2.jpg}
	\caption{Magnetischer Flussdichte gegen Stromstärke aufgetragen}
	\label{plot:1}
\end{figure}

\begin{center}
$\delta \lambda$ = (6,09 \pm 0,22) pm
\end{center}

  \subsection{\texorpdfstring{Bestimmung des Lande-Faktors $g_{12}$ der $\sigma$-"Uberg"ange f"ur die blaue Linie}{Bestimmung des Lande-Faktors g_{12} der sigma-"Uberg"ange f"ur die blaue Linie}}
  
          \begin{table}[H] 
	\centering
	\caption{Für die Eichung des Magneten aufgenommene Messwerte des Magnetfeldes neben der zugehörigen Stromstärke} 
	\begin{tabular}{c|c|c|c}

  & $\Delta$ s in px & $\delta s$ in px & $\delta \lambda$ in $10^{-12} m$\\
  \hline 
1&178&175&13,25 \pm 0,32 \\
2&172&171&13,40 \pm 0,33\\
3&160&165&13,90 \pm 0,36\\
4&144&153&14,32 \pm 0,41\\
5&124&142&15,43 \pm 0,50\\
6&118&135&15,42 \pm 0,52\\
7&111&127&15,42 \pm 0,55 \\

		
	\end{tabular} 
	  \label{tab:mit}
\end{table} 

\begin{figure}[h]
	\centering
	\includegraphics[width=16cm,height=12cm]{Fotos/V27_3.jpg}
	\caption{Magnetischer Flussdichte gegen Stromstärke aufgetragen}
	\label{plot:1}
\end{figure}

\begin{center}
$\delta \lambda$ = (14,45 \pm 0,25) pm
\end{center}


  \subsection{\texorpdfstring{Berechnung der experimentellen Lande-Faktoren $g_{ij}$ aus den Abst"anden $\Delta s$ und $\delta s$ der erhaltenen Aufnahmen}{Berechnung der experimentellen Lande-Faktoren g_{ij} aus den Abst"anden Delta s und delta s der erhaltenen Aufnahmen}}

  $\Delta s$ ist der Abstand der benachbarten Interferentzmaxima in der Aufnahme ohne Magnetfeld, $\delta s$ ist die Breite der Aufspaltung eines Interferenzmaximas, in der Aufnahme mit angelegtem Magnetfeld mit St"arke $B$.
  Damit ergibt sich f"ur den Wellenl"angenunterschied $\delta \lambda$ zwischen den beiden Energieniveaus des "Ubergangs
  \begin{equation}
    \delta \lambda = \frac{1}{2}\frac{\delta s}{\Delta s} \Delta \lambda \; .
  \end{equation}
  Mit
  \begin{align}
    \begin{split}
    \frac{\partial E}{\partial \lambda} = \frac{\delta E}{\delta \lambda} &= \frac{\delta}{\delta \lambda} \frac{hc}{\lambda} = -\frac{hc}{\lambda^2}\\
    \iff \delta E &= \frac{ch}{\lambda^2} \delta \lambda
   \end{split}
  \end{align}
  folgt f"ur den Lande-faktor $g_{12}$ mit der Energieniveaudifferenz $\delta E$/Wellenl"angendifferenz $\delta \lambda$ nach Formel (\ref{g_ij}) der Zusammenhang
  \begin{equation}
    g_{12}=\frac{\delta E}{\mu_BB}=\frac{hc\delta \lambda}{\lambda^2\mu_BB} \; .
  \end{equation}




















