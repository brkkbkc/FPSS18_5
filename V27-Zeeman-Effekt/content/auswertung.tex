\section{Auswertung}
\label{sec:Auswertung}

\begin{figure}
  \centering
  \includegraphics{plot.pdf}
  \caption{Plot.}
  \label{fig:plot}
\end{figure}


  \subsection{\texorpdfstring{Berechnung der experimentellen Lande-Faktoren $g_{ij}$ aus den Abst"anden $\Delta s$ und $\delta s$ der erhaltenen Aufnahmen}{Berechnung der experimentellen Lande-Faktoren g_{ij} aus den Abst"anden Delta s und delta s der erhaltenen Aufnahmen}}

  $\Delta s$ ist der Abstand der benachbarten Interferentzmaxima in der Aufnahme ohne Magnetfeld, $\delta s$ ist die Breite der Aufspaltung eines Interferenzmaximas, in der Aufnahme mit angelegtem Magnetfeld mit St"arke $B$.
  Damit ergibt sich f"ur den Wellenl"angenunterschied $\delta \lambda$ zwischen den beiden Energieniveaus des "Ubergangs
  \begin{equation}
    \delta \lambda = \frac{1}{2}\frac{\delta s}{\Delta s} \Delta \lambda \; .
  \end{equation}
  Mit
  \begin{align}
    \begin{split}
    \frac{\partial E}{\partial \lambda} = \frac{\delta E}{\delta \lambda} &= \frac{\delta}{\delta \lambda} \frac{hc}{\lambda} = -\frac{hc}{\lambda^2}\\
    \iff \delta E &= \frac{ch}{\lambda^2} \delta \lambda
   \end{split}
  \end{align}
  folgt f"ur den Lande-faktor $g_{12}$ mit der Energieniveaudifferenz $\delta E$/Wellenl"angendifferenz $\delta \lambda$ nach Formel (\ref{g_ij}) der Zusammenhang
  \begin{equation}
    g_{12}=\frac{\delta E}{\mu_BB}=\frac{hc\delta \lambda}{\lambda^2\mu_BB} \; .
  \end{equation}


  \subsection{\texorpdfstring{Bestimmung des Lande-Faktors $g_{12}$ der $\sigma$-"Uberg"ange f"ur die rote Linie}{Bestimmung des Lande-Faktors g_{12} der sigma-"Uberg"ange f"ur die rote Linie}}

  \subsection{\textorpdfstring{Bestimmung des Lande-Faktors $g_{12}$ der $\pi$-"Uberg"ange f"ur die blaue Linie}{Bestimmung des Lande-Faktors g_{12} der pi-"Uberg"ange f"ur die blaue Linie}}


  \subsection{\textorpdfstring{Bestimmung des Lande-Faktors $g_{12}$ der $\sigma$-"Uberg"ange f"ur die blaue Linie}{Bestimmung des Lande-Faktors g_{12} der sigma-"Uberg"ange f"ur die blaue Linie}}

    Hier zwei verschiedene g12 f"ur die zwei  uebergaenge, deshlmb. die zwei aufgeteilten linien sind eigentlich 4, die 3/2 linie und die 2 linie ueberlagern sich nur, weil mit dieser apperatir nicht genau aufgeloest werden koennen.
























b
