\section{Durchführung}
\label{sec:Durchführung}
\begin{figure}[h]
	\centering
	\includegraphics[width=14cm,height=11cm]{fotos/Bilddurchfuhrung1.pdf}
	\caption{Schematischer Aufbau der Messapparatur}
	\label{durch:1}
\end{figure}
Um mit der Durchführung beginnen zu können, soll zunächst eine Vorbereitungsaufgabe gelöst werden. Als erstes soll die Cd-Lampe mit einem Objektiv und einer Linse $L1$ so eingestellt werden, dass sie scharf auf den Spalt $S1$ abbildet. Des Weiteren wird die Linse $L2$ so justiert, dass ein paralleles Lichtbündel auf das Glasprisma fällt. Der Durchmesser des Lichtbündels soll möglichst klein bleiben und in der Größe des Prismas sein, um Strahlungsverlust zu vermeiden. Als nächstes gilt der erste Vorgang für $L3$ und $2S$ und es ist möglich mit diesem Bild eine Wellenlänge auszuwählen. Nun soll mittels $L4$ ein scharfes Bild auf die Lummer-Gehrcke-Platte abgebildet werde. Der nächste Schritt wäre den Polarisator für den Übergang $\Delta m = \pm 1,0$ einzustellen. An diesem Schritt ist es sinnvoll das Magnet einzuschalten und die Zeeman-Linien zu beobachten. Als letztes wird das Bild mit einer Digitalkamera aufgenommen und gespeichert.


