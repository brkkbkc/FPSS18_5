\section{Zielsetzung}
  Im Versuch wird die, durch den (an)normalen Zeeman-Effekt versuchsachte, Aufspaltung der Energieniveaus von Cadmium (Cd) mit einem geeigneten Versuchsaufbau visualisiert und die experimentel ermittelten Energiedifferenzen mit den theoretisch berechneten verglichen.


\section{Theorie}
\label{sec:Theorie}
  Als Zeeman-Effekt wird das Phenomen bezeichnet, dass sich die diskreten Energieniveaus der Elektronen in Atomen unter Einfluss eines "au"seren Magnetfeldes $\vec{B}$ aufspalten.
  Es wird zwischen normalem und annormalem Zeeman-Effekt unterschieden:
  \subsection{magnetisches moment von atomen}


  \subsection{Zeeman-Effekt}
    Hat ein Atom ein magnetisches momen mue, wird erhaelt dieses zus"atzliche energie Emag=-mujB im aeusseren magnetfeld.
    res koennen nur winkel zwischen mue und b auftreten koennen. bei denen mujz =  ist (Richtungsquantelung), m ist debei element -j bis j.
    Damit wird e mag zu :

    F"ur S=0 ist gj immer 1 (normaler zeeman effekt), fuer s =/= ist gj = ... - gji (annormaler zeeman effekt)

  \subsection{Auswahlregeln}  
    .
