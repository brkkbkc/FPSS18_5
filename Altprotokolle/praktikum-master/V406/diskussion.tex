% This work is licensed under the Creative Commons
% Attribution-NonCommercial 3.0 Unported License. To view a copy of this
% license, visit http://creativecommons.org/licenses/by-nc/3.0/.

\section{Diskussion}

Zunächst wird auf die Messung der Beugungsfigur am Einzelspalt
eingegangen. Nach Betrachtung von Abbildung~\ref{fig:single-slit} kann
festgestellt werden, daß die Meßwerte sich gut durch eine nichtlineare
Ausgleichsrechnung annähern können. Aus dem bestimmten Funktionsterm in
Formel~\eqref{eq:single-slit-fit} kann auch die Breite $b$ des Spalts
abgelesen werden. Diese beträgt:
%
\begin{equation}
  b = \SI{0.0778e-3}{\metre} \text{.}
\end{equation}

Die Bestimmung mithilfe des Mikroskops ist verglichen damit
ungenau. Die Breite eines Skalenteils muß bestimmt werden, was nicht
ohne Ablesefehler vollzogen werden kann. Und die Vermessung des Spalts
ist wieder mit Ablesefehlern behaftet. Bei der Untersuchung der
Beugungsfigur sind so viele Meßdaten aufgenommen worden, daß davon
ausgegangen werden kann, daß sich die Ablesefehler herausmitteln.

Auch die Bestimmung der Spaltbreite am variablen Einzelspalt gelingt,
weil die nichtlineare Ausgleichsrechnung gute Ergebnisse liefert. Der
variable Spalt kann zwischen \SI{0.02e-3}{\metre} und
\SI{0.2e-3}{\metre} variert werden. Die eingestellte Spaltbreite kann
aus der Formel~\eqref{eq:variable-slit-fit} zu
%
\begin{equation}
  b = \SI{0.0503e-3}{\metre}
\end{equation}
%
abgelesen werden. Zur berechneten Figur muß zwar angemerkt werden, daß die
Meßwerte sich nicht so gut anschmiegen, wie beim Einzelspalt, dies läßt
sich wohlmöglich darauf zurückführen, daß für die Berechnung der
Beugungsfigur einfach die Verteilung für einen Einzelspalt genommen
worden ist. Dies ist natürlich nur in Näherung gültig, weil sich der
Spalt ja in $y$-Richtung verjüngt bzw. verbreitert und somit ein anderes
Interferenzmuster zu erwarten ist.

Beim Doppelspalt läßt sich die Spaltbreite und der Spaltabstand leider
nicht bestimmen. Statt dessen können die Meßwerte nur mit einer
berechneten Beugungsfigur für die angegeben Daten des verwendeten Spalts
verglichen werden. Beim Betracheten von Abbildung~\ref{fig:double-slit}
kann nicht davon gesprochen werden, daß die berechnete Figur die
Meßwerte annähert. Zwar liegen Intensitätsmaxima in der Nähe der
berechneten Maxima, aber die Form der Beugungsfigur kann nicht bestätigt
werden. 