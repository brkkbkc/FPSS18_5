% This work is licensed under the Creative Commons
% Attribution-NonCommercial 3.0 Unported License. To view a copy of this
% license, visit http://creativecommons.org/licenses/by-nc/3.0/.

\section{Durchführung}

In diesem Versuch wird die Fraunhofersche Beugung am Einzelspalt,
variablen Spalt und Doppelspalt untersucht. Dabei ist der variable Spalt
als Einzelspalt zu betrachten.  Die für die Fraunhofersche Beugung benötigten
Lichtstrahlen werden mithilfe eines He-Ne-Lasers realisiert, welcher
sich in einem Abstand von \SI{10}{\centi\metre} zum Spalt befindet. In
einem Abstand von \SI{100}{\centi\metre} zum Spalt wird die Beugungsfigur
vermessen. Dazu wird ein Verschiebereiter mit Photoelement
verwendet. Die Beugungsfigur wird mit dem Photoelement von $x=
\SI{0}{\centi\metre}$ bis $x = \SI{5}{\centi\metre}$ in
\SI{1}{\milli\metre} Schritten vermessen. Dabei ist der Laser so
ausgerichtet, dass sich das Hauptmaximum der Beugungsfigur bei
ca. \SI{25}{\milli\metre} befindet. Da das Photolelement bereits bei
ausgeschaltetem Laser einen sogenannten Dunkelstrom abgibt, wird dieser
vor Beginn jeder Messreihe gemessen und bei der Auswertung berücksichtigt.