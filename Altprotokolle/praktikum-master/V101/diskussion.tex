% This work is licensed under the Creative Commons
% Attribution-NonCommercial 3.0 Unported License.  To view a copy of
% this license, visit http://creativecommons.org/licenses/by-nc/3.0/.

\section{Diskussion}
Zum Schluss soll noch die Güte der Messergebnisse beurteilt werden.

Der statistische Fehler der in diesem Versuch bestimmten Winkelrichtgröße 
$D$ = \SI{2.4}{\newton\centi\metre} beträgt $\approx$ \SI{8.3}{\percent}.
Das Ablesen der Messwerte der statischen Methode erweist sich als 
schwierig. Dafür ist der statistische Fehler relativ gering. 

Für das Eigenträgheitsmoment der verwendeten Drillachse ergibt sich 
ein negativer Wert nach Abzug des Trägheitsmomentes des Stabes, 
an welchem die Gewichte angebracht wurden. Dies impliziert, dass 
bei der Bestimmung der Trägheitsmomente verschiedener Körper 
das Eigenträgheitsmoment der Drillachse vernachlässigt werden kann,
 was in diesem Versuch auch getan wurde.

Die experimentell bestimmten Trägheitsmomente der Kugel und des 
Zylinders weichen von den Theoriewerten um maximal $\approx$ 
\SI{16}{\percent} ab. Diese Abweichungen rühren zum einen davon, 
dass die verwendeten Körper evtl. nicht eine exakt homogene 
Massendichte besitzen und zum anderen, dass die Periodendauern 
manuell mit einer Uhr gemessen wurden. Diese Zeitmessung ist als 
die größte Fehlerquelle in diesem Versuch anzusehen.
Die Verhältnisse der theoretisch errechneten und experimentell 
bestimmten Trägheitsmomente für die Kugel und die Zylinder 
weichen allergins nur um maximal um $\approx$ \SI{4}{\percent} ab.
Innerhalb dieser Fehlerbetrachtungen zeigt sich, dass Theorie 
und experimentelle Beobachtungen kompatibel sind.

Bei den Trägheitsmomenten der Holzpuppen in den verschiedenen 
Stellungen ergeben sich in Stellung Eins eine relative Abweichung 
zwischen experimentellem und theoretischem Wert von $\approx$ 
\SI{66}{\percent}. Für die zweite Stellung beträgt die Abweichung 
$\approx$ \SI{22}{\percent}.
Es ist zu sehen, dass der Fehler in der Stellung mit dem kleineren 
Trägheitsmoment deutlich größer ist. Dies liegt daran, dass die 
Fehler der manuellen Zeitmessung bei einer geringeren 
Periodendauer größere relative Abweichungen erzeugen.

Aus diesem Grund ist die in diesem Versuch verwendete Methode 
zur Bestimmung des Trägheitsmomentes eines Körpers gut für 
 größere Trägheitsmomenten geeignet, als für welche 
mit kleinem Trägheitsmoment.
Die Fehler, die aufgrund der Zeitmessung entstehen, lassen sich 
minimieren, indem die Zeit gemessen wird, die der Körper benötigt, 
um mehrer Schwingungen durchzuführen. Anschließend wird diese Zeit 
durch die Anzahl der Schwingungen geteilt.