% This work is licensed under the Creative Commons
% Attribution-NonCommercial 3.0 Unported License.  To view a copy of
% this license, visit http://creativecommons.org/licenses/by-nc/3.0/.

\section{Auswertung}

\subsection{Winkelrichtgröße und Eigenträgheitsmoment der Drillachse}

Die mit der statischen Methode bestimmten Kräfte $F$ im Abstand $r$ zur
Drillachse unter Auslenkung um den Winkel $\phi$ sind in
\cref{tab:winkelrichtgroesse} aufgeführt.  Pro Messwertepaar ergibt sich
mit Formel~\eqref{eq:winkelricht-kraft-abstand-winkel} ein Wert für die
Winkelrichtgröße $D$ der verwendeten Drillachse.  Diese sind ebenfalls
in \cref{tab:winkelrichtgroesse} eingetragen.  Als Mittelwert der in
\cref{tab:winkelrichtgroesse} eingetragenen Werte für die
Winkelrichtgröße der verwendeten Drillachse ergibt sich
\begin{equation}
D = \SI{2.4(2)}{\newton\centi\metre}
\end{equation}
Der angegebene Fehler ist hierbei die Standardabweichung des
Mittelwertes der errechneten Winkelrichtgrößen.

\begin{table}
  \centering
  \begin{tabular}{SSSS|SSSS}
    \toprule
    {r/}\si{\centi\metre}&{F/}\si{\milli\newton}&
    {$\phi$/}\si{\degree}&{D/}\si{\newton\centi\metre}&
    {r/}\si{\centi\metre}&{F/}\si{\milli\newton}&
    {$\phi$/}\si{\degree}&{D/}\si{\newton\centi\metre}\\
    \midrule
    5&20&10&1.1&11&75&40&2.4\\
    7&60&20&2.4&12&70&50&1.9\\
    9&83&30&2.9&14&114&60&3.0\\
    9&90&30&3.1&7&30&10&2.4\\
    10&42&20&2.4&16&42&35&2.2\\
    \bottomrule
  \end{tabular}
  \caption{Aufgenommene Messwerte zur Bestimmung der Winkelrichtgröße
    der verwendeten Drillachse.  Ebenfalls eingetragen sind die
    errecheten Werte für die Winkelrichtgröße.}
  \label{tab:winkelrichtgroesse}
\end{table}

Das Eigenträgheitsmoment der Drillachse wird mithilfe des steinerschen
Satzes bestimmt. Dieser ist in Gleichung~\ref{eq:steiner} wiedergegeben.
Es werden die in Tabelle~\ref{tab:b} eingetragenen Messwerte für die
Schwingungsdauer $T$ des gewichtbelasteten Stabes und die jeweiligen
Abstände $a$ der Gewichte von der Drehachse verwendet. Hierfür wird
$T^2$ gegen $a^2$ aufgetragen. Dieser Plot ist is Abb.~\ref{fig:steiner}
zu sehen.
%
\begin{figure}
  \centering
  \includegraphics[width=0.7\textwidth]{steiner}
  \caption{Hier zu sehen ist der Abstand a des Gewichtschwerpunktes
    zu der Drehachse zum Quadrat aufgetragen gegen die 
    gemessenen Schwingungsdauern zum Quadrat.}
  \label{fig:steiner}
\end{figure}
%
%
\begin{table}
  \centering
  \begin{tabular}{SSS}
    \toprule
    {a/}\si{\centi\metre}& {T/}\si{\second}&
    {T/}\si{\second}\\
    \midrule
    7.5&2.97&2.98\\
    9.5&3.41&3.39\\
    11.5&3.90&3.91\\
    13.5&4.44&4.39\\
    15.5&4.93&4.75\\
    29.5&8.20&8.16\\
    \bottomrule
  \end{tabular}
  \caption{Schwingungsdauer des gewichtbelasteten Stabes 
    für verschiedene Abstände a von Gewichtsschwerpunkt
    und Drehachste.}
  \label{tab:b}
\end{table}
%
Die lineare Ausgleichsrechnung\footnote{Dazu wurde \texttt{ipython} in
  der Version 0.13 verwendet} durch diese Punkte ergibt die
Geradengleichung
%
\begin{equation}
f(x) = x\cdot\SI{706.26}{\second^2\per\metre^2} + \SI{5.73}{\second^2}
\end{equation}
%
Das Eigenträgheitsmoment der verwendeten Drillachse kann aus dem
Ordinatenabschnitt $b = \SI{5.73}{\second^2}$ der Ausgleichsgeraden
bestimmt werden.  Aus Formel~\eqref{eq:regressionsform} ist ersichtlich,
dass
\begin{equation}
  b = \frac{4\pi^2}{D}\big(I_\text{D} + I_\text{S} + 2I_\text{G}\big)
\end{equation}
gilt.  Durch Umstellen ergibt sich für das gesuchte Eigenträgheitsmoment
$I_\text{D}$, dass
%
\begin{equation}
  I_\text{D} = \frac{b D}{4\pi^2} - I_\text{S} - 2I_\text{G},
\label{eq:ix}
\end{equation}
%
wobei $D$ die oben bestimmte Winkelrichtgröße, $I_\text{G}$ das
Trägheitsmoment eines der zylinderförmigen Gewichte bezüglich der
Drehung durch den Schwerpunkt, und $I_\text{S}$ das Trägheitsmoment des
Aufhängestabes ist.  Für $I_\text{G}$ ergibt sich bei einer Masse $m$
von \SI{223}{\gram}, einem Radius $R$ von \SI{1.75}{\centi\metre} und
einer Höhe $h$ von \SI{3}{\centi\metre} mit
Formel~\eqref{eq:traegheit-zylinder-stehend} ein Wert von
%
\begin{equation}
I_\text{G} = \SI{3.4e-5}{\kilo\gram\metre^2}
\label{eq:iz}
\end{equation}
%
Für das Trägheitsmoment $I_\text{S}$ des verwendeten Stabes wird analog
vorgegangen. Die Masse $m$ des Stabes wird zu \SI{96.5}{\gram} und seine
Länge $l$ zu \SI{60}{\centi\metre} gemessen.
Formel~\eqref{eq:traegheit-duenner-stab} ergibt, dass
%
\begin{equation}
I_\text{S} = \SI{0.0048}{\kilo\gram\metre^2}
\label{eq:is}
\end{equation}
%
Somit ergibt sich das Eigenträgheitsmoment der verwendeten Drillachse
mit Formel~\eqref{eq:ix} und den Trägheitsmomenten~\eqref{eq:iz}
und~\eqref{eq:is} zu
%
\begin{equation}
I_\text{D} = \SI{-0.0014(3)}{\kilo\gram\metre^2}
\end{equation}
%
Der hierbei angegebene Fehler wird durch eine Gaußsche
Fehlerfortpflanzung der Formel~\eqref{eq:ix} für das
Eigenträgheitsmoment bestimmt. Die fehlerbehafteten Größen sind die
Winkelrichtgröße $D$ mit dem statistischen Fehler und der
Ordinatenabschnitt der Ausgleichsgerade aus dem Plot\footnote{Dazu wurde
  ebenfalls \texttt{ipython} in der Version 0.13
  verwendet}~\ref{fig:steiner}.
%
\FloatBarrier
\subsection{Trägheitsmoment zweier Körper}
%
In Tabelle~\ref{tab:c} sind die gemessenen Schwingungsdauern $T$ der
verwendeten Kugel und des Zylinders zu sehen.  Mit
Formel~\eqref{eq:periode} werden aus den Messwerten die Trägheitsmomente
der Kugel und des Zylinders bestimmt.  Es ergibt sich das in
Tabelle~\ref{tab:zweikoerper} wiedergegebene Ergebnis.  Die dort
angebenen Fehler sind die statistischen Fehler der Messreihe.  Es wird
ebenfalls das Verhältnis der Trägheitsmomente sowohl für die theoretisch
errechneten, als auch für die experimentell bestimmten Werte angegeben.
%
\begin{table}
  \centering
  \begin{tabular}{SSSS}
    \toprule
    \multicolumn{2}{S}{{Kugel}}&\multicolumn{2}{S}{{Zylinder}}\\
    {T/}\si{\second}&{T/}\si{\second}&
    {T/}\si{\second}&{T/}\si{\second}\\
    \cmidrule(r){1-2} \cmidrule(l){3-4}
    1.71&1.75&1.50&1.5\\
    1.72&1.75&1.52&1.51\\
    1.65&1.73&1.56&1.62\\
    1.73&1.71&1.53&1.51\\
    1.68&1.70&1.60&1.58\\
    \bottomrule
  \end{tabular}
  \caption{Schwingungsdauer der untersuchten Kugel und Zylinder}
  \label{tab:c}
\end{table}


\begin{table}
  \centering
  \begin{tabular}{SSS}
    \toprule
    {Objekt}&{Theoriewert}&{Experimentalwert}\\
    \midrule
    {Kugel}&\SI{0.0015}{\kilo\gram\metre^2}&\SI{0.0018(1)}{\kilo\gram\metre^2}\\
    {Zylinder}&\SI{0.0013}{\kilo\gram\metre^2}&\SI{0.0014(1)}{\kilo\gram\metre^2}\\
    {Verhältnis $\frac{I_\text{K}}{I_\text{Z}}$}&1.18&1.23\\
    \bottomrule
  \end{tabular}
  \caption{Trägheitsmomente zweier Körper}
  \label{tab:zweikoerper}
\end{table}
% 
\FloatBarrier
\subsection{Trägheitsmoment einer Holzpuppe für zwei verschiedene
  Stellungen}
%
Für die beiden verschiedenen Stellungen der verwendeten Holzpuppe werden
die in Tabelle~\ref{tab:holzkopp} aufgeführten Schwingungsdauern für die
Schwingung der Holzpuppe um die Drehachse gemessen.
%
\begin{table}
  \centering
  \begin{tabular}{SS|SS}
    \toprule
    \multicolumn{2}{S|}{{1.Stellung}}&\multicolumn{2}{S}{{2.Stellung}}\\
    \midrule
    {T/}\si{\second}&{T/}\si{\second}&
    {T/}\si{\second}&{T/}\si{\second}\\
    \midrule
    0.44&0.47&0.64&0.66\\
    0.41&0.45&0.67&0.69\\
    0.43&0.47&0.67&0.68\\
    0.45&0.46&0.64&0.66\\
    0.41&0.45&0.67&0.69\\
    \bottomrule
  \end{tabular}
  \caption{Schwingungsdauer der Holzpuppe für zwei verschiedene 
    Stellungen.}
  \label{tab:holzkopp}
\end{table}
% 
Mit Formel~\eqref{eq:periode} wird über die Schwingungsdauer das
Trägheitsmoment der Holzpuppe für beide Stellungen errechnet.  Durch
Näherung des Kopfes, der Arme, der Beine und des Torsos als Zylinder
wird über den steinerschen Satz~\eqref{eq:steiner} durch Addition der
Einzelträgheitsmomente das Gesamtträgheitsmoment der Modellnäherung der
Holzpuppe errechnet.  In Tabelle~\ref{tab:fritten} sind die so
experimentell und theoretisch bestimmten Werte vorzufinden. Auch hier
wird das Verhältnis der Trägheitsmomente in den beiden Stellungen
jeweils für die Experimental- und Theoriewerte angegeben.
%
\begin{table}
  \centering
\begin{tabular}{SSS}
    \toprule
    {Stellung}&{Theoriewert}&{Experimentalwert}\\
    \midrule
{1}&\SI{0.00004}{\kilo\gram\metre^2}&\SI{0.00012(12)}{\kilo\gram\metre^2}\\
{2}&\SI{0.00033}{\kilo\gram\metre^2}&\SI{0.00027(10)}{\kilo\gram\metre^2}\\
{Verhältnis $\frac{I_\text{1}}{I_\text{2}}$}&0.112&0.437\\
    \bottomrule
  \end{tabular}
  \caption{Trägheitsmomente der Holzpuppe in den verschiedenen Stellungen}
  \label{tab:fritten}
\end{table}
%
