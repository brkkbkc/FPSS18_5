% This work is licensed under the Creative Commons
% Attribution-NonCommercial 3.0 Unported License. To view a copy of this
% license, visit http://creativecommons.org/licenses/by-nc/3.0/.

\section{Durchführung}
Eine schematische Abbildung der in diesem Versuch verwendeten Neutronenquelle ist in Abb. \ref{fig:neutronenquelle} zu finden. Um die beim Zerfall der radioaktiven Isotopenkerne emittierte Strahlung nachzuweisen wird ein Geiger-Müller zählrohr verwendet, welches von außen isoliert ist. Vor dem Start einer Messreihe muss eine sogenannte ,,Nullmessung" gemacht durchgeführt werden, da es bereits ohne radioaktivem Isotop im Zählrohr zur Registrierung radioaktiver Strahlung kommt. Dies liegt an der Höhenstrahlung und an natürlicher Radioaktivität.
Als erstes wird in diesem Versuch Rhodium untersucht. Dabei wird ein Zeitintervall $\Delta$t von \SI{20}{\second} gewählt. Es werden 36 Messwerte aufgenommen, also insgesamt \SI{12}{\minute} lang gemessen. 
Anschließend wird erneut eine Nullmessung durchgeführt. Nun wird aktiviertes Indium in analoger Weise zum Rhodium vermessen. Das hierbei gewählte Zeitintervall $\Delta$t beträgt \SI{240}{\second}. Es werden 15 Messwerte aufgenommen, dies entspricht einer gesamten Messdauer von \SI{1}{\hour}.
