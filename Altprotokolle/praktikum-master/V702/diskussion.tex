% This work is licensed under the Creative Commons
% Attribution-NonCommercial 3.0 Unported License. To view a copy of this
% license, visit http://creativecommons.org/licenses/by-nc/3.0/.

\section{Diskussion}

Zum Schluss soll noch etwas über die Güte der Messungen und der Ergebnisse dieses Versuchs gesagt werden.
Bei der Untersuchung des aktivierten Indiums ergibt sich eine Halbwertszeit von T = \SI{54.35}{\minute}. Der Literaturwert dieser Halbwertszeit beträgt nach \textcite{chemglobe} \SI{54.2}{\minute}. Hierbei ist also eine Abweichung zwischen Versuchsergebnis und gefundenem Literaturwert von \SI{0.22}{\percent}. Dies ist eine durchaus akzeptable Abweichung.

Der in diesem Versuch bestimmte Wert für die Halbwertszeit des langlebigem Rhodiumisotops beträgt $T_1$ = \SI{896}{\second}, besitzt allerdings einen sehr großen Fehler von $\Delta T_1$ = \SI{1026}{\second}. Dies liegt zum einen daran, dass die Werte, die zur Bestimmung dieser Halbwertszeit genutzt wurden, sehr weit streuen. Außerdem erkennt man anhand von Abb. \ref{fig:plot_rhodium}, dass der statistische Fehler jeder Einzelmessung der letzten 13 Messwerte ebenfalls bereits sehr groß ist.
Auf Grundlage dieses sehr ungenauen Ergebnisses wurde nun die Halbwertszeit des kurzlebigen Rhodiumisotops ermittelt. Als gefundenen Wert ergibt sich $T_2$ = \SI{49.64}{\second} und $\Delta T_2$ = \SI{2}{\second}. Der hierbei angegebene Fehler ist wieder deutlich geringer. Dies liegt daran, dass für die Berechnung dieses Wertes die ersten 16 neu errechneten Impulse verwendet wurden. Diese besitzen einen geringeren statistischen Fehler und sind nicht so start gestreut. Als Literaturwert für diese Halbwertszeit ist bei \textcite{periodensystem} ein Wert von T = \SI{42.3}{\second} angegeben. Also beträgt die Abweichung zum Wert dieses Versuches \SI{17.35}{\percent}. Da dieser Fehler deutlich unter \SI{100}{\percent} liegt, lässt sich sagen, dass der in diesem Versuch gefundene Wert für die Halbwertszeit $T_1$ des langlebigen Rhodiumisotops ebenfalls um deutlich weniger als \SI{100}{\percent} vom tatsächlichen Wert abweicht.