% This work is licensed under the Creative Commons
% Attribution-NonCommercial 3.0 Unported License. To view a copy of this
% license, visit http://creativecommons.org/licenses/by-nc/3.0/.
\section{Theorie}
\subsection{Prinzip eines Lasers}
Ein Laser emittiert monochromatisches Licht hoher Intensität und
Kohärenz.  Er besteht aus drei grundlegenden Komponenten: Lasermedium,
Pumpquelle und Resonator.  Das Lasermedium bestimmt durch die möglichen
Energieübergänge das Strahlungsspektrum des Lasers.  Die Pumpquelle
erzeugt die benötigte Besetzungsinversion durch stimulierte
Emission, um eine Lasertätigkeit zu erhalten.  Der Resonator sorgt für
einen selbsterregenden Oszillator durch Rückkopplung des Lichts, um das
Strahlungsfeld des Lichts zu verstärken.  Kurz gefaßt: Das grundlegende
Prinzip eines Lasers ist die Verstärkung des einfallenden Lichts durch
Wechselwirkung mit Lasermaterial.  Das einfachste Modell ist der
sogenannte Zwei-Niveau-Laser, welcher jedoch nur theoretisch 
funktioniert.

\subsection{Zwei Niveau-Laser}
Für einen Zwei-Niveau-Laser wird ein Zwei-Niveau-System benötigt,
z. B. ein Grundzustand und ein angeregter Zustand in einem Atom.  Ein
einfallendes Photon wird dann absorbiert, wenn es die passende 
Energie hat.
Dadurch findet ein Zustandswechsel in ein höheres Niveau, den angeregten
Zustand, statt.  Ein Übergang in den Grundzustand unter Aussendung eines
Photons kann nun spontan oder durch stimulierte Emission geschehen.  Bei
stimulierter Emission hat das emittierte Photon die gleiche Energie,
Phase und Ausbreitungsrichtung wie das einfallende Photon. Im
Lasermedium wird der Besetzungszustand durch die Besetzungsdichten $n_1$
(Grundzustand) und $n_2$ (angeregter Zustand) beschrieben.  Bei
Wechselwirkung mit dem Strahlungsfeld $\rho$ erhöht sich 
Besetzungsdichte 
$n_2$ durch Absorption und vermindert sich durch spontane und induzierte
Emission.  Wenn keine Verluste auftreten, ergeben sich entsprechend die
Gleichungen:
\begin{align}
  \dot{n}_1 &= -\rho B_{12} n_1 + A_{21} n_2 + \rho B_{21} n_2\\
  \dot{n}_2 &=  \rho B_{12} n_1 - A_{21} n_2 - \rho B_{21} n_2
\end{align}
mit den Einstein-Koeffizienten $A_{21}$, $B_{12}$, $B_{21}$, die ein Maß
für die Wahrscheinlichkeit von Absorption und spontaner bzw. induzierter
Emission darstellen.  Ein solches Zwei-Niveau-System kann aber nicht als
Laser funktionieren, da die Besetzungsinversion stationär sein muss.  
Aus
den Gleichungen folgt aber, dass im stationären Betrieb die Inversion
nicht aufrechterhalten werden kann, da die Wahrscheinlichkeit dafür, dass
im unteren Niveau ein Photon absorbiert wird, genauso groß ist wie die
Wahrscheinlichkeit, dass im oberen Niveau ein Photon durch stimulierte
Emission abgegeben wird, sobald die Hälfte aller Teilchen im oberen
Niveau ist.

\subsection{Rolle der Besetzungsinversion bei der stimulierten Emission}
Um eine dauerhafte Verstärkung des Lichts zu erhalten, muß die
induzierte Emission häufiger auftreten als die spontane Emission, da nur
so das ausgesendete Photon die gleiche Phase und Ausbreitungsrichtung
wie das einfallende Photon hat.  Im thermischen Gleichgewicht ist jedoch
nach der \name{Maxwell}-\name{Boltzmann}-Verteilung der Grundzustand am
häufigsten besetzt.  Daher wird durch äußere Energiezufuhr, das
sogenannte Pumpen, dem Medium Energie zugeführt, damit die angeregten
Zustände höher besetzt sind als die Grundzustände.  Dieses Pumpen kann dabei
durch Elektronenstöße oder durch optische Anregung geschehen.  Die Verstärkung
wächst exponentiell mit der Länge des Laufweges der Photonen im 
aktiven Medium an.

\subsection{Der optische Resonator}
Der optische Resonator sorgt daher dafür, daß das Licht einen möglichst
langen Weg im Lasermedium zurücklegt und so mehrmals die Atome anregt.
Dies wird durch zwei Spiegel, die sich gegenüberstehen, realisiert.
Einer der Spiegel ist teildurchlässig, um einen Teil des Laserstrahls
für die Nutzung auszukoppeln.  Für die Spiegel gibt es unterschiedliche
Bauformen, z.\,B. planparallel oder sphärisch.  Dabei möchte man
möglichst geringe Verluste haben. Die Verluste sind besonders bei
konfokalen Spiegeln minimal, deren Spiegelbrennpunkte zusammenfallen.
Der Resonator heißt optisch stabil, wenn die Verluste geringer sind als
die Verstärkung durch induzierte Emission, so daß die Lasertätigkeit
einsetzen kann.  Dabei gilt folgende Stabilitätsbedingung:
\begin{equation}
  0 \le g_1 g_2 \le 1,
\end{equation}
mit den Resonatorparametern $g_i = 1 - L/r_i$, wobei $r_i$ die
Krümmungsradien der Spiegel und $L$ die Resonatorlänge bedeuten.

Da die Resonatorlänge sehr viel größer als die Wellenlänge des Lasers ist,
gibt es viele Frequenzen, die die Resonanzbedingung für stehenden Wellen
im Resonator erfüllen.  Die Moden werden mit $q$ durchnumeriert und
longitudinale Moden genannt.  Aufgrund von unvollkommenen Spiegeln
und/oder Verkippungen können auch transversal verschiedene Moden
auftreten. Mit $l$ und $p$ werden die Knoten in $x$- bzw. $y$-Richtung
numeriert, die sog. transversalen Modenzahlen.  Die Moden des Resonators
werden wie beim Hohlleiter mit TEM$_{lpq}$ gekennzeichnet, wobei die
longitudinale Mode oft weggelassen wird.  Da höhere Moden größere
Verluste haben als niedrigere Moden, werden nur wenige transversale
Moden verstärkt.  Die Mode mit den geringsten Verlusten ist die
Grundmode TEM$_{00}$, ihre Intensitätsverteilung ist durch eine
Gauß-Verteilung beschrieben:
\begin{equation}
  I(r) = I_0 e^{-\frac{2r^2}{w^2}}
\end{equation}
\FloatBarrier