% This work is licensed under the Creative Commons
% Attribution-NonCommercial 3.0 Unported License. To view a copy of this
% license, visit http://creativecommons.org/licenses/by-nc/3.0/.

\section{Aufbau}

\subsection{Erzeugung des Elektronenstrahls in der Kathodenstrahlröhre}
\label{sec:erzeugung-elektronenstrahl}

Für beide Versuche wird ein Elektronenstrahl benötigt, der dann im
elektrischen bzw. magnetischen Feld abgelenkt wird. Dieser Strahl wird
mithilfe einer Kathodenstrahlröhre erzeugt.

Die Röhre besteht aus einem evakuiertem Glaskolben, in dessen Innerem sich im
wesentlichen diese sechs Komponenten befinden:

\begin{itemize}
\item Die Glühkathode, aus der die Elektronen austreten,
\item der sog. \name{Wehnelt}-Zylinder, zur Regulierung der
  Strahl-Intensität,
\item die Beschleunigungselektrode zur Beschleunigung der ausgetretenen
  Elekronen,
\item die Elektronenlinse (eine Anordnung zur Fokussierung der Strahlen)
\item die Ablenkplatten,
\item der Leuchtschirm zum Strahlnachweis.
\end{itemize}

Die ersten vier Komponenten werden auch zusammenfassend als
Elektronenkanone bezeichnet.

\subsubsection{Die Elektronenkanone}

Durch eine Heizwendel wird ein Material mit niedriger Austrittsarbeit
indirekt beheizt, so daß Elektronen austreten können. 

Der umgebende \name{Wehnelt}-Zylinder, der ein negatives Potential
gegenüber der Kathode hat, wirkt als Abschirmung, die Elektronen mit
niedriger Energie davon abhält den Bereich der Kathode zu
verlassen. Außerdem hat er eine fokussierende Wirkung: Er lenkt
Elektronen auf seine Rotationsachse.

Hinter dem Zylinder befindet sich die positiv geladene
Beschleunigerelektrode, welche die Elektronen, die den Kathodenbereich
verlassen können, beschleunigt. Die Geschwindigkeit der Elektronen nach
Verlassen der Elektrode läßt sich aus dem Energiesatz berechnen.
%
\begin{equation}
  \label{eq:acc-volt-vel-z}
  \frac{1}{2} m_0 v_z^{\,2} = e_0 U_\text{B} \:.
\end{equation}

Dann gelangen diese in die die sog. Elektronenlinse, die aus einem Zylinder mit
mehreren Lochblenden im Innern besteht. Diese sind negativ geladen und
fokussieren den Elektronenstrahl. Die Brechkraft dieser Linse läßt sich
mithilfe einer Spannung regeln.

\subsubsection{Die Ablenkplatten und der Leuchtschirm}

Es gibt zwei um $\pi/2$ zueinander gedrehte gegenüberliegende
Plattenpaare. Diese Paare können einzeln mit einer Spannung belegt
werden, die die Elektronen in $x$- bzw. $y$-Richtung ablenkt.

Nach dem Verlassen der Ablenkplatten trifft der Elektronenstrahl
schließlich auf den Leuchtschirm. Dort regt er die sog. Aktivatorzentren
des Materials zur Emission von Lichtquanten an.

\subsection{\name{Helmholtz}-Spulen-Anordnung}

Für die Ablenkung im magnetischen Feld wird die Kathodenstrahlröhre
zwischen ein \name{Helmholtz}-Spulenpaar gestellt. Das ist eine
besondere Konfiguration von zwei Spulen, die ein nahezu homogenes Feld
in dem von den Spulen begrenzten Volumen erzeugen.

Das besondere am Spulenpaar ist, daß der Abstand der Spulen gleich dem
Radius der verwendeten Spulen ist. Hierdurch ist das Feld auf der
axialen Achse praktisch homogen. Die magnetische Flußdichte ist
natürlich abhängig vom Spulenstrom und beträgt
%
\begin{equation}
  \label{eq:mag-force-helmholtz}
  \Vert\vec{B}\Vert = \mu_0 \frac{8}{\sqrt{125}} \frac{NI}{R} \:.
\end{equation}
%
Hier bezeichnen $\mu_0$ die magnetische Permeabilität des Vakuums, $N$
die Windungszahl einer Spule, $I$ die Spulenstromstärke und $R$ den
Spulenradius.

\subsection{Geometrie zur Bestimmung der spezifischen Ladung}

Aus Formel~\eqref{eq:traj-radius} kann nun die spezifische Ladung des
Elektrons aus dem Krümmungsradius berechnet werden. Gemessen wird aber
die Leuchtfleckverschiebung. Um einen Zusammenhang herzustellen wird der
Satz des Pythagoras benutzt. Am Ende gilt für den Zusammenhang zwischen
den einer Messung zugänglichen Größen und der spezifischen Ladung
%
\begin{equation}
  \label{eq:spec-charge-measurement}
  \frac{D}{L^2+D^2} = \frac{1}{\sqrt{8U_\text{B}}} \sqrt{\frac{e_0}{m_0}} B.
\end{equation}
