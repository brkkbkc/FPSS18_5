% This work is licensed under the Creative Commons
% Attribution-NonCommercial 3.0 Unported License. To view a copy of this
% license, visit http://creativecommons.org/licenses/by-nc/3.0/.

\section{Diskussion}
%
\subsection{Modenuntersuchung}
%
Die Untersuchung dreier Moden des Klystrons mit einem 
Oszilloskop ist eine gut funktionierende Methode. 
Für höhere Reflektorspannungen ergeben sich höhere Leistungen 
der Mikrowellen.

Lediglich die Beobachtung des dips aufgrund der Einstellung des 
Frequenzmessers war an den Rändern einer Mode schwierig, aber 
machbar.
%
\subsection{Elektronische Abstimmung}
%
Hierbei wird eine Bandbreite von \SI{40}{\mega\hertz} für die Mode 
mit einer Mittenfrequenz von \SI{9001}{\mega\hertz} ermittelt und 
eine Abstimm-Empfindlichkeit des Klystrons von 
\SI{1.6}{\mega\hertz\per\volt}. 

Die Reflektorspannungsabhängigkeit der Mikrowellenfrequenz ist 
sehr hoch, sodass diese sehr genau einstellbar sein muss, wenn 
Mikrowellen mit dem Klystron 2K25 zur Anwendung kommen sollen.
%
\subsection{Frequenzmessungen}
%
Bei der direkten Frequenzmessung mit dem Frequenzmesser wird 
für eine fest eingestelle Mikrowelle eine Frequenz von 
\SI{8994}{\mega\hertz} gemessen. Über eine Wellenlängenmessung 
einer stehenden Welle und den Abmessungen des Hohlleiters wird 
hingegen eine Frequenz von \SI{9132}{\mega\hertz} für die gleiche 
Mikrowelle errechnet. 

Die relative Abweichung beider Werte beträgt zwar nur 
\SI{1.5}{\percent}, aber eine absolute Abweichung von 
\SI{138}{\mega\hertz} reicht aus, um den gewollten 
Anwendungsbereich in der Praxis verlassen zu können. 

In diesem Fall ist für die direkte Messung nicht anzugeben, wie 
genau diese ist, da bei dem verwendeten Frequenzmesser nicht 
direkt zu überprüfen ist, ob die Anzeige stimmt. 

Die Wellenlängen- und Hohlleitermessung hingegen ist mit 
bekannten Fehlern zu versehen, sodass die indirekte 
Frequenzmessung in diesem Versuch als die bessere anzusehen ist, 
die eingestellte Frequenz der Mikrowelle also eher 
ca. \SI{9132}{\mega\hertz} beträgt.
%
\subsection{Dämpfungsmessung}
%
Zwischen Eichkurvenanzeige und SWR-Meteranzeige ist eine 
große Diskrepanz in der Dämpfung des Dämpfungsgliedes 
festzustellen. 
Die mithilfe des SWR-Meters ermittelte Dämpfung ist als die 
richtige anzusehen. 

Im zum Versuch gehördenden Handbuch wird die 
Funktionsweise des Dämpfungsgliedes über eine absorbierende Folie 
erklärt, welche ihre Position im Hohlleiter je nach 
Mikrometerschraubenstellung ändert. Es ist möglich, dass diese sich 
mit der Zeit ausnutzt und das Dämpfungsglied somit seine Arbeitsweise 
im Laufe der Zeit geändert hat. Diese Aussage müsste allerdings noch 
weiter untersucht werden, was im Rahmen dieses Praktikumsversuches 
nicht möglich ist.
Außerdem gibt es große Ableseungenauigkeiten bei der Verwendung 
der Eichkurve.
%
\subsection{Kleine und mittlere Welligkeiten}
%
Die direkte Messung der Welligkeiten stehender Wellen für verschiedene 
Sondentiefen des Gleitschraubentransformators liefert plausible Werte 
für Sondentiefen von \SI{3}{\milli\metre}, \SI{5}{\milli\metre} und 
\SI{7}{\milli\metre}. Bei einer Sondentiefe von 
\SI{9}{\milli\metre} hingegen Versagt das Verfahren, weswegen 
für große Welligkeiten andere Verfahren verwendet werden.
%
\subsection{Große Welligkeiten}
%
Bei der \SI{9}{\milli\metre}-Sondentiefe ergibt 
die \SI{3}{\decibel}-Methode eine Welligkeit von ca. 
\SI{437.7}{} und die Abschwächer-Methode eine Welligkeit von 
\SI{9}{}. Dies ist eine riesige Abweichung voneinander. 

Die Abschwächer-Methode ergibt hierbei aber wohl den falschen Wert, 
da eine Welligkeit von \SI{9}{} noch mit der direkten Methode für 
kleine und mittlere Welligkeiten messbar ist, wohingegen 
eine Welligkeit von \SI{437.7}{} zu dem dort gefundenen unendlich 
großen Wert passt.

Bei der Abschwächer Methode wird das Dämpfungsglied verwendet und 
die Dämpfung an der Eichkurve abgelesen. Der scheinbar zu kleine 
Welligkeitswert ist ein weiteres Indiz dafür, dass die Eichkurve des 
Dämpfungsgliedes fehlerhaft ist.
%