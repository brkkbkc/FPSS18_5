% This work is licensed under the Creative Commons
% Attribution-NonCommercial 3.0 Unported License. To view a copy of this
% license, visit http://creativecommons.org/licenses/by-nc/3.0/.

\section{Durchführung}

\subsection{Versuchsaufbau}

Der Versuch besteht prinzipiell aus drei Teilen: Einer Platine, auf der
die Metalle, das Peltier-Element und die Thermoelemente befestigt sind,
einer Spannungsquelle und dem Datenlogger.

Eine Skizze der rechteckigen Platine findet man in
Abbildung~\ref{fig:platine}. In der Mitte sieht man das Peltier-Element,
das sowohl für die Heizung, als auch die Kühlung der Metalle zuständig
ist. Rechts und links vom Peltier-Element verlaufen die Stäbe. In
gleichen Abständen finden sich dort die Thermoelemente
\texttt{T1}--\texttt{T8}.

Die Spannungsquelle ist für die Versorgung des Peltier-Elements
zuständig und mit dem Datenlogger werden über ein sogenanntes
\emph{Temperature Array} die Meßdaten aufgenommen und zur
Weiterverarbeitung auf einem USB-Stick gespeichert.

\begin{figure}
  \centering
  \includegraphics{platine}
  \caption{Schematische Darstellung der Platine mit den Aufbauten}
  \label{fig:platine}
\end{figure}

\subsection{Statische Messung}

Bei der statischen Methode wird der zeitliche Temperaturverlauf bei
einer Heizspannung~$U_P = \SI{5}{\volt}$ aufgezeichnet. Die Abtastrate
des Datenloggers wird hierbei auf \SI{5}{\second} eingestellt. Die
Messung wird beendet, wenn das Thermoelement \texttt{T7} ungefähr
\SI{45}{\degreeCelsius} anzeigt.

\subsection{Dynamische Messung}

Bei der dynamischen Methode, die auch \name{\r Angström}-Methode genannt
wird, heizt bzw. kühlt das Peltier-Element mit einer Periode von
\SI{80}{s}. Die Heizspannung~$U_P$ beträgt hierbei \SI{8}{\volt}. Jetzt
wird die Auflösung des Datenloggers auf \SI{2}{\second} eingestellt. Nun
werden zehn Perioden gemessen. Dann wird eine zweite Messung mit einer
Periodendauer von \SI{200}{\second} durchgeführt.
