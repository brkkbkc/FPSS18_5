% This work is licensed under the Creative Commons
% Attribution-NonCommercial 3.0 Unported License.  To view a copy of
% this license, visit http://creativecommons.org/licenses/by-nc/3.0/.

% Richtige Trennung
\hyphenation{Dreh-impuls}

\section{Theorie}

\subsection{Magnetische Suszeptibilität und Paramagnetismus}

Zur Beschreibung des magnetischen Feldes im Vakuum wird die magnetische
Flußdichte~$\vec{B}$ verwendet.  Sie erfüllt die
\name{Maxwell}-Gleichungen für das Vakuum.  Werden nun Felder in Materie
studiert, ist es zweckmäßig eine neue Größe, die magnetische
Feldstärke~$\vec{H}$, so einzuführen:
\begin{equation}
  \vec{H} = \frac{1}{\mu_0}(\vec{B} - \vec{M}).
\end{equation}
Hierbei ist $\vec{M}$ die sogenannte Magnetisierung, die von der Materie
abhängig ist.  Im wesentlichen ist sie ein Mittelwert der im Material
vorkommenden magnetischen Momente~$\vec{\mu}$.  Durch Einführung der
Größe~$\vec{H}$ haben die \name{Maxwell}-Gleichungen in Materie wieder
dieselbe Form wie im Vakuum.  Es gilt:
\begin{equation}
  \vec{M} = \mu_0 \chi \vec{H}.
\end{equation}
Die Größe~$\chi$ heißt Suszeptibilität und ist keine Konstante, sondern
von Temperatur~$T$ und magnetischer Feldstärke~$\vec{H}$ abhängig.

Im Fall $\chi < 0$ wird von Diamagnetismus gesprochen.  Alle Atome
weisen unabhängig davon, ob sie Bestandteil eines Gases, einer
Flüssigkeit oder eines Festkörpers sind, diesen Effekt auf.  Er basiert
im Prinzip auf einer Induktion magnetischer Momente durch äußere
Magnetfelder.  Ab einer magnetischen Suszeptibilität~$\chi>0$ spricht
man von Paramagnetismus.  Dieser tritt nur bei Materie auf, deren
Bausteine (wie z.\,B. Atome, Moleküle) einen Drehimpuls aufweisen.
Damit ist eine Ausrichtung der mit dem Drehimpuls verknüpften
magnetischen Momente in einem äußeren Magnetfeld verbunden.  Mithilfe
der Quantenmechanik ergeben sich diese Momente so:
\begin{align}
  \vec\mu_L &= -\frac{\mu_B}{\hbar} \vec{L} &
  \vec\mu_S &= -g_S\frac{\mu_B}{\hbar} \vec{S},
\end{align}
wobei hier $\mu_B$ das sogenannte \name{Bohr}sche Magneton (das
magnetische Moment zum Drehimpuls $\hbar$) und $g_S$ das gyromagnetische
Verhältnis des freien Elektrons bezeichnen.  Der Paramagnetismus ist
also im wesentlichen durch den Bahndrehimpuls des Atoms und dem Spin der
Elektronen bestimmt.  Der Einfluß des Kernspins kann hier vernachlässigt
werden.  Außerdem kann bei einem hinreichend schwachen Magnetfeld davon
ausgegangen werden, daß sich der Gesamtdrehimpuls~$\vec{J}$ des Atom
durch eine Addition
\begin{equation}
  \vec{J} = \vec{L} + \vec{S}
\end{equation}
bestimmen läßt.  Daraus ergibt sich mithilfe der aus der Quantenmechanik
bekannten Beträge $|\vec{L}| = \sqrt{L(L + 1)}, |\vec{S}| = \sqrt{S(S +
  1)}$ und $|\vec{J}| = \sqrt{J(J+1)}$ für den Betrag~$|\vec{\mu}_J|$
des magnetischen Moments:
\begin{equation}
  |\vec{\mu}_J| = \mu_B g_J \sqrt{J(J + 1)}
\end{equation}
Hier ist
\begin{equation}
g_J := \frac{3}{2} \left(1 + \frac{S(S + 1) - L(L+1)}{J(J + 1)}\right)
\end{equation}
der sogenannte \name{Landé}-Faktor des Atoms.  Über die Energieniveaus
der Elektronen des Materials kann die Magnetisierung einer Probe
bestimmt werden.  Dazu werden die magnetischen Momente aufsummiert, die
von den Drehimpulsen bestimmt sind.  Durch die
\name{Boltzmann}-Verteilung ist ein Zusammenhang zwischen der
Besetzungshäufigkeit und der Temperatur gegeben.  Hieraus ergibt sich
für hohe Temperaturen
\begin{equation}
  \label{eq:curie-gesetz}
  \chi = \frac{\mu_0 \mu_B^2 g_J^2 N J(J + 1)}{3 k T},
\end{equation}
wobei $T$ die Temperatur, $k$ die \name{Boltzmann}-Konstante, $\mu_0$
die magnetische Permeabilität des Vakuums und $N$ die Teilchendichte
bezeichnet.  Die Ausrichtung der magnetischen Momente wird also durch
thermische Bewegungen gestört, so daß der Paramagnetismus ein
temperaturabhängiger Effekt ist.  Insbesondere für hohe Temperaturen ist
$\chi$ antiproportional zu $T$.  Dies nennt man \name{Curie}sches Gesetz
des Paramagnetismus.

\subsection{Suszeptibilität der Verbindungen seltener Erden}

Die Verbindungen von Elementen, die Seltene Erden genannt werden, weisen
einen starken Paramagnetismus auf.  Daraus ergibt sich, daß die
Elektronen in diesen Atomen große Drehimpulse aufweisen müssen, der von
den inneren Elektronen ausgeht.  Bei den Seltenen Erden sind dies die
4f-Elektronen.  Die Elektronenkonfiguration und der Gesamtdrehimpuls des
Atoms kann mithilfe der \name{Hund}schen\footnote{Friedrich Hund
  (1896--1997) deutscher Physiker. (nach
  \textcite{wikipedia:friedrich-hund})} Regeln bestimmt werden. Diese
lauten:

\begin{enumerate}
\item Die Anordnung der Spins ist so, daß der Gesamtspin maximal ist
  unter Beachtung des Pauli-Prinzips.
\item Die Bahndrehimpulse ordnen sich so an, daß der
  Gesamtbahndrehimpuls maximal wird und Regel~1 und das
  \name{Pauli}-Prinzip beachtet wird.
\item Der Gesamtdrehimpuls ist gegeben durch $\vec{J} = \vec{L} -
  \vec{S}$, falls die Schale weniger als halb besetzt ist, und durch
  $\vec{J} = \vec{L} + \vec{S}$, falls die Schale mehr als halb besetzt
  ist.
\end{enumerate}
Diese Regeln werden für die theoretische Berechnung der
Suszeptibilitäten der Verbindungen Seltener Erden benötigt.  Gemäß
Gleichung~\eqref{eq:curie-gesetz} muß dazu unter anderem der
\name{Landé}-Faktor und die Gesamtdrehimpulsquantenzahl bestimmt werden.
