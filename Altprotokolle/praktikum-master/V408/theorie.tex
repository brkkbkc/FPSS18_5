% This work is licensed under the Creative Commons
% Attribution-NonCommercial 3.0 Unported License. To view a copy of this
% license, visit http://creativecommons.org/licenses/by-nc/3.0/.

\section{Theorie}

In der geometrischen Optik wird die Bewegung von Licht als Lichtstrahlen
untersucht, die sich geradlinig ausbreiten und an Übergängen zwischen
zwei Medien gebrochen werden.

Ein Beispiel für so einen Übergang zwischen zwei Medien wird durch eine
Linse realisiert.  Linsen bestehen meist aus einem Material, das optisch
dichter als das umgebende Medium ist.  Ein Lichtstrahl, der eine Linse
passiert, wird also zweimal gebrochen.  Das erste Mal beim Eintritt in
die Linse, das zweite Mal beim Austritt.  Es gibt verschiedene
Linsentypen.  Die Sammellinse bündelt parallele Strahlen im Brennpunkt
der Linse.  Sie werden zum Linsenrand dünner und sind nahe der optischen
Achse dicker, sind also konvex geformt.  Ein weiterer Linsentyp ist die
Zerstreuungslinse.  Parallele Lichstrahlen werden hier von der optischen
Achse weggebrochen, daher der Name.  Dieser Linsentyp wird zum Rand hin
dicker und ist in der Nähe der optischen Achse dick, ist also konkav
geformt.  Die Sammellinse bildet einen Gegenstand hinter der Linse auf
ein reelles Bild ab, das mit einem Schirm sichtbar gemacht werden
kann. Brennweite~$f$ und Bildweite~$b$, d.\,i. die Entfernung des
reellen Bildes von der Linse, werden durch eine positive Zahl angegeben.

Bei einer Sammellinse entsteht weder vor der Linse noch dahinter ein
Bild, das mit einem Schirm sichtbar gemacht werden kann.  Blickt man
allerdings durch die Linse scheint es, als ob der Gegenstand vor der
Linse stünde.  Es entsteht ein virtuelles Bild.  Die Brennweite und
Bildweite wird daher negativ gemessen.

Zur Konstruktion des Abbildungsverhaltens bei dünnen Linsen wird die
Brechung des Lichtstrahls nur auf eine Brechhung an der Mittelebene der
Linse vereinfacht.  Bei dicken Linsen werden dagegen zwei sogenannte
Hauptebenen eingeführt, an denen der Lichstrahl dann gebrochen wird. Zur
eigentlichen Konstruktion werden dann drei verschiedene Strahlentypen
verwendet.  Der Mittelpunktstrahl geht durch die Mitte der Linse und
wird nicht gebrochen.  Ein Parallelstrahl verläuft vor seine Brechung
parallel zur optischen Achse und wird an der Mittelebene so gebrochen,
daß er durch den Brennpunkt der Linse verläuft. Hierbei muß beachtet
werden, ob die Linse eine Sammel- oder Zerstreuungslinse ist, d.\,h. ob
die Brennweite~$f$ positiv oder negativ ist.  Ein Strahl, der durch den
Brennpunkt der Linse verläuft, ein Brennpunktstrahl, wird so an der
Mittelebene gebrochen, daß sein Strahl danach parallel zur optischen
Achse verläuft.  Eine beispielhafte Konstruktion der Abbildung einer
dünnen Sammel- und Zerstreuungslinse ist in
Abbildung~\ref{fig:konstruktion} zu sehen.

\begin{figure}
  \centering
  \includegraphics{konstruktion}
  \caption{Die Konstruktion der Abbildung einer Sammellinse (links) und
    einer Zerstreuungslinse (rechts). Die Abbildung stammt aus der
    Anleitung zu \textcite{v408} und ist geringfügig modifiziert
    worden.}
  \label{fig:konstruktion}
\end{figure}

Aus der Konstruktion folgt mit den Strahlensätzen der folgende
Zusammenhang
%
\begin{equation}
  \label{eq:abbildungsmassstab}
  V = \frac{B}{G} = \frac{b}{g},
\end{equation}
%
wobei $B$ und $G$ hier jeweils Bild- und Gegenstandsgröße und $b$ und
$g$ Bild- und Gegenstandsweite bezeichnen.  Das Verhältnis $V$ nennt man
auch Abbildungsmaßstab.  Für dünne Linsen ergibt sich noch die
sogenannte Linsengleichung
%
\begin{equation}
  \label{eq:linsengleichung}
  \frac{1}{f} = \frac{1}{b} + \frac{1}{g}.
\end{equation}

Die Vereinfachung der Abbildung einer dünnen Linse durch eine Brechung
an der Mittelebene gilt streng genommen nur für achsennahe Strahlen.
Treffen die Strahlen in größerer Entfernung von der optischen Achse auf
die Linse kommt es zu Abbildungsfehlern.  Teile des Bildes sind dann
nicht mehr scharf auf dem Schirm abgebildet.

Es treten zwei Haupteffekte auf: Zum einen die sphärische Abberation,
bei der der Brennpunkt achsenferner Strahlen näher an der Linse liegt,
als derjenige von achsenfernen Strahlen. Zum anderen die chromatische
Abberation, bei der der Brennpunkt von blauem Licht näher an der Linse
liegt, als derjenige von rotem Licht (Stichwort Dispersion).

Zur Beschreibung des Brechverhaltens von Linsen wird eine neue
größe, die sogenannte Brechkraft 
%
\begin{equation}
  \label{eq:brechkraft}
  D = \frac{1}{f}
\end{equation}
%
eingeführt, die die Einheit Dioptrie (Abk. dpt) erhält. Wird ein System
aus mehreren dünnen Linsen zusammengebaut, so addieren sich die
Brechkräfte.
