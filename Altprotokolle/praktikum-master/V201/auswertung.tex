% This work is licensed under the Creative Commons
% Attribution-NonCommercial 3.0 Unported License.  To view a copy of
% this license, visit http://creativecommons.org/licenses/by-nc/3.0/.

\section{Auswertung}

Die Temperaturen sind mithilfe von Thermoelementen gemessen worden.
Daher sind die erhaltenen Spannungen mit Formel~\eqref{eq:thermo} in
Temperaturen umzurechnen.  Die Formel gibt die Temperatur in Grad
Celsius aus, welche schließlich in Kelvin umgerechnet wurde.  Die Werte
befinden sich in \cref{tab:temp1} und \cref{tab:temp2}.

\begin{table}
  \centering
  \begin{tabular}{SSSSSS}
\toprule
{$U_x/\si{\mV}$}&
{$U_y/\si{\mV}$} & 
{$U_m'/\si{\mV}$} &
{$T_x/\si{\K}$} &
{$T_y/\si{\K}$} &
{$T_m'/\si{\K}$} \\
\midrule
0.890 & 4.000 & 2.192 & 295.39 & 370.74 & 327.38 \\
0.872 & 3.965 & 2.265 & 294.94 & 369.91 & 329.16 \\
1.036 & 4.058 & 2.400 & 299.01 & 372.11 & 332.43 \\
\bottomrule
\end{tabular}

\caption{Hier finden sich die gemessenen Thermospannungen aus der Meßreihe zur Bestimmung der Wärmekapazität des Kalorimeters.  Daneben befinden sich die gemäß Formel~\eqref{eq:thermo} errechneten Temperaturen in Kelvin.}
\label{tab:temp2}
\end{table}

\begin{table}
  \centering
  \begin{tabular}{lSSSSSS}
  \toprule
  & {$U_w/\si{\mV}$}
  & {$U_k/\si{\mV}$}
  & {$U_m/\si{\mV}$} 
  & {$T_w/\si{\K}$}
  & {$T_k/\si{\K}$}
  & {$T_m/\si{\K}$} \\
  \midrule
  Blei &
  0.672 & 3.937 & 0.760 & 289.97 & 369.25 & 292.16 \\
  & 
  0.938 & 3.948 & 1.067 & 296.58 & 369.51 & 299.78 \\
  &
  1.169 & 3.944 & 1.188 & 302.30 & 369.41 &  302.77 \\
  \midrule
  Kupfer &
  0.750 & 3.960 & 0.938 & 291.91 & 369.79 & 296.58 \\
  \midrule
  Zinn &
  1.084 & 4.003 & 1.152 & 300.20 & 370.81 & 301.88 \\
  \bottomrule
\end{tabular}

\caption{Hier werden die Thermospannungen zu den Meßreihen zur Bestimmung der spezifischen Wärmekapazitäten der Proben aufgeführt.  Daneben finden sich wieder die umgerechneten Temperaturen in Kelvin.}
\label{tab:temp1}
\end{table}


\subsection{Wärmekapazität des Kalorimeters}

Aus den Meßwerten in \cref{tab:temp2} ergibt sich gemäß
Formel~\eqref{eq:kalorikapa} die Wärmekapazität des Kalorimeters.  Die
Massen der verwendeten Wassermengen sind nicht direkt bekannt, sondern
werden über die Dichte, die mit $\rho = \SI{1000}{\kg\per\cubic\metre}$
angesetzt wird, aus dem gemessen Volumen ermittelt.  Die Volumina der
beiden Wassermengen betragen jeweils $V_\text{x, y} =
\SI{320}{\cubic\centi\metre}$.

\begin{table}
  \centering
  \begin{tabular}{lS}
    \toprule
    & $c_g m_g/\si{\J\per\kelvin}$\\
    \midrule
    & 475 \\
    & 255 \\
    & 250 \\
    \midrule
    Mittelwert & 327(74) \\
    \bottomrule
  \end{tabular}

  \caption{Diese Tabelle zeigt die ermittelte Wärmekapazität des
    Kalorimeters, die in  drei Messungen bestimmt worden ist.  Darunter
    befindet sich der Mittelwert aus den Messungen.}
  \label{tab:waermekapkalori}
\end{table}

\subsection{Spezifische Wärmekapazitäten und Molwärmen der Proben}
\label{sec:molwaerme}

Mithilfe der gemessenen Temperaturen aus \cref{tab:temp1} und der
Wärmekapazität des Kalorimeters erhält man nach Einsetzen in
Formel~\eqref{eq:waermekapa} die gewünschten spezifischen
Wärmekapazitäten der Proben.  Mit
\begin{equation}
  C_p = c_\text{k} \, M_\text{k}
\end{equation}
lassen sich daraus mithilfe der molaren Masse $M_\text{k}$ der Proben
die Molwärmen bei konstantem Druck bestimmen.  Die Umrechnung von
Molwärme bei konstantem Druck auf Molwärme bei konstantem Volumen
geschieht nach Formel~\eqref{eq:cv-und-cp}.  In \cref{tab:molwaermen}
finden sich die Ergebnisse dieser Rechnungen.  

Beim Blei sind drei Messungen durchgeführt worden, wobei auffällt, daß
der letzte ausreißt.  Da es zwei verschiedene Bleiproben gegeben hat,
liegt die Vermutung nahe, daß die beiden verwechselt worden sind.  Daher
werden für den Mittelwert der Molwärme nur die ersten beiden Meßwerte
berücksichtigt.
\begin{equation}
  C_V = \SI{35(9)}{\J\per\K}.
\end{equation}

\begin{table}
  \centering
  \begin{tabular}{lSSS}
    \toprule
    & {$c_\text{k}/(\si{\J\per\kg\per\K})$}
    & {$C_p/(\si{\J\per\mol})$}
    & {$C_V/(\si{\J\per\mol})$} \\
    \midrule
    Blei & 136.8 &  28.4 & 26.7  \\
    & 220.8 &  45.7 & 44.0  \\
    &  33.9  &   7.0 &  5.3  \\
    \midrule
    Kupfer & 760.0 & 48.3  &   48.2  \\
    \midrule
    Zinn & 287.7 &   34.2 &   32.4  \\
    \bottomrule
  \end{tabular}
  \caption{Spezifische Wärmekapazitäten der vermessenen Proben aus Blei,
    Kupfer und Zinn.  Dazu sind die Molwärmen berechnet worden.}
  \label{tab:molwaermen}
\end{table}


