% This work is licensed under the Creative Commons
% Attribution-NonCommercial 3.0 Unported License.  To view a copy of
% this license, visit http://creativecommons.org/licenses/by-nc/3.0/.

\section{Diskussion}

\subsection{Vergleich mit der klassischen Theorie}

Die in \cref{sec:molwaerme} experimentell bestimmten Molwärmen der
Proben, die nach dem Gesetz von \name{Dulong}-\name{Petit} unabhängig
von Probe, Temperatur, usw. den Wert $3R$ aufweisen sollten (hier
bezeichnet $R = \SI{8.3144621}{\J\per\mol\per\kelvin}$ die universelle
Gaskonstante), weichen folgendermaßen ab: Blei \SI{41.6}{\percent},
Kupfer \SI{93.3}{\percent}, Zinn \SI{30.0}{\percent}.  Trotz der
theoretischen Vorhersage, daß die Molwärmen bei Zimmertemperatur schon
durch das \name{Dulong}-\name{Petit}sche Gesetz beschrieben werden
können, ergeben sich relativ hohe Abweichungen.  Aufgrund des
Meßverfahrens konnte die Temperatur der Probe nicht direkt bestimmt
werden, sondern es ist davon ausgegangen worden, daß die Probe sich mit
dem Wasserbad gleichmäßig erwärmt und daher das Wasser die gleiche
Temperatur aufweist wie die zu messende Probe.  Hierbei könnten
Meßungenauigkeiten auftreten, wenn die Probe und das Wasserbad sich
nicht gleich schnell erwärmen.

\subsection{Vergleich mit der quantenmechanischen Vorhersage}

Nach der quantenmechanischen Theorie dürften derartige Abweichungen nur
bei geringem Atomgewicht wie z.\,B. Bor oder Beryllium auftreten.  Die
Atomgewichte der hier untersuchten Proben liegen aber deutlich über den
Werten von Bor oder Beryllium (B ca. 10 u, Be ca. 9 u, Cu ca. 63 u, Pb
ca. 207 u, Sn ca. 118 u).  Trotz der relativ hohen Abweichungen läßt
sich zumindest bestätigen: je geringer das Atomgewicht, desto größer die
Abweichung von der klassischen Vorhersage.  Beim Kupfer, welches das
geringste Atomgewicht aufweist, ist die Abweichung mit
\SI{93.3}{\percent} am größten.  Obgleich in diesem Versuch auch die
klassische Theorie gute Vorhersagen hätte machen müssen, konnte diese
durch den Versuch nicht nachgewiesen werden.  Statt dessen läßt sich die
zugrundeliegende Folge der quantenmechanischen Theorie, daß die Molwärme
vom Atomgewicht abhängt, qualitativ am Kupfer bestätigen.
