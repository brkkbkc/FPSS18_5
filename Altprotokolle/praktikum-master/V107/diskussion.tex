% This work is licensed under the Creative Commons
% Attribution-NonCommercial 3.0 Unported License. To view a copy of this
% license, visit http://creativecommons.org/licenses/by-nc/3.0/.

\section{Diskussion}
Zum Abschluss soll das Ergebis dieses Protokolls bewertet werden.
Die Bestimmung der Durchmesser der Kugeln mithilfe der Schieblehre
erweist sich als äußerst ungenau, da sich während des Messvorganges
nicht feststellen lässt, ob der Abstand zwischen zwei diametrisch auf
der Kugeloberfläche liegenden Punkte vermessen wird.

Das Verwenden von Winkeln ist frei von dieser Störanfälligkeit. Das
Aufstellen einer Messreihe gibt ebenfalls mehr Sicherheit bei der
Bestimmung des Durchmessers. Alle ermittelten Werte für die Viskosität
des verwendeten Wassers sind deutlich größer als die gefundenen Literaturwerte, welche aus \textcite{demtroeder-1} entnommen werden. In Tabelle \ref{tab:literatur} sind sowohl die berechneten Viskositäten $\eta_\text{ber}$, als auch die Literaturwerte $\eta_\text{lit}$ aufgeführt.
%
\begin{table}[h]
  \centering
  \begin{tabular}{SSS}
    \toprule
{T /}\si{\celsius}&$\eta_\text{ber}$ {/ }\si{\milli\pascal\second}& $\eta_\text{lit}$ {/ }\si{\milli\pascal\second}\\
\midrule
20	&1.700&1.002\\
40	&0.901&0.653\\
60	&0.667&0.466\\
    \bottomrule
  \end{tabular}
  \caption{Gemessene Fallzeiten der großen Kugel und errechnete Werte}
  \label{tab:literatur}
\end{table}
%
Die Abweichung kann entweder dadurch erklärt werden, dass die Viskosität des untersuchten
Wassers größer als in der Literaturangabe, oder aber dass es sich um
einen systematischen Fehler handelt.
Bei der Betrachtung des Plots in Abb. \ref{fig:viskoseplot} fällt die starke Abweichung der Messwerte von einer Geraden auf. Dies ist ein Indiz dafür, dass die Messwerte aufgrund eines systematischen Fehlers stark fehlerhaft sind. Besonders im Bereich niedriger Temperaturen ist die Abweichung groß. Da die Messung der Fallzeiten manuell erfolgt, steckt in dieser Messung das größte Fehlerpotential, weswegen die Fallzeitmessung als Quelle der starken Abweichungen der berechneten Werte von den Literaturwerten anzusehen ist.

Ebenfalls zu beachten ist, dass sich ab
einer Temperatur von \SI{60}{\celsius} Blasen innerhalb des Wassers
gebildet haben, welche die Strömungseigenschaften beeinflusst haben.

Da der Wert der Reynoldszahl bei Zimmertemperatur zwischen 29 und 96
liegt und bei höheren Temperaturen bis \SI{70}{\celsius} für die große Kugel maximal den Wert 71,9 erreicht, ist destilliertes Wasser also als laminare Flüssigkeit zu betrachten.
