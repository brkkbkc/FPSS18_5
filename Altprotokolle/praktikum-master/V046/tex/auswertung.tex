% This work is licensed under the Creative Commons
% Attribution-NonCommercial 3.0 Unported License. To view a copy of this
% license, visit http://creativecommons.org/licenses/by-nc/3.0/.

\section{Auswertung}

\subsection{Magnetfeldmessung}
Nach der Justage der Apparatur wird die Komponente des Magnetfeldes,
welche parallel zum Lichtstrahl verläuft, vermessen.  Dies geschieht mit
einer Hallsonde.  Die örtliche Abhängigkeit des Magnetfeldes ist in
\cref{fig:magnetfeld} zu sehen. Die dazugehörigen Messwerte befinden 
sich in Tabelle~\ref{tab:magnetmess}. 
Die maximale magnetische Flussdichte
beträgt $B_\text{max} = \SI{448}{mT}$.

\begin{table}[h]
  \centering
  \sisetup {
    per-mode = fraction
  }
  \begin{tabular}{SSSSSSSS}
    \toprule
    {Pos./}\si{\milli\metre}&{B/}\si{\milli\tesla}&
    {Pos./}\si{\milli\metre}&{B/}\si{\milli\tesla}&
    {Pos./}\si{\milli\metre}&{B/}\si{\milli\tesla}&
    {Pos./}\si{\milli\metre}&{B/}\si{\milli\tesla}\\
    \cmidrule(rl){1-2}
    \cmidrule(rl){3-4}
    \cmidrule(rl){5-6}
    \cmidrule(rl){7-8}
    0&0&95&3&121&446&135&60\\
    10&1&100&12&122&444&140&12\\
    20&1&105&58&123&439&145&3\\
    30&0&110&267&124&432&150&1\\
    40&1&115&429&125&422&160&0\\
    50&1&116&437&126&409&170&0\\
    60&0&117&442&127&390&180&0\\
    70&0&118&446&128&364&190&1\\
    80&0&119&447&129&328&200&1\\
    90&1&120&448&130&281& & \\
    \bottomrule
  \end{tabular}
  \caption{Die mit einer Hallsonde aufgenommenen 
   Werte der magnetischen Flussdichte, welche durch den 
   in diesem Versuch verwendeten Elektromagneten bei einem 
   Feldstrom von \SI{10.48}{\ampere} erzeugt wird in Abhängigkeit 
   der Position der Messsonde.}
  \label{tab:magnetmess}
\end{table}

\begin{figure}
  \centering
  \includegraphics[width=0.8\textwidth]{abbildungen/magnetfeld.pdf}
  \caption{Die grapische Darstellung der Ergebnisse der
    Magnetfeldmessung.  Das Maximum ist deutlich zu erkennen.  Außerdem
    beschränkt sich das Magnetfeld auf die Zone, in der sich die Probe
    befindet.}
  \label{fig:magnetfeld}
\end{figure}
\FloatBarrier
\subsection{Die effektiven Masse der Leitungselektronen in GaAs}

In \cref{fig:faraday-rot} sind die Faraday-Rotationswinkel der 
drei vermessenen Proben gegen die Wellenlänge aufgetragen. 
Diese Winkel werden aus den erhaltenen 
Messdaten gemäß \cref{eq:drehwinkel} berechnet. Die Messwerte und 
die errechneten Faraday-Rotationswinkel sind in 
\cref{tab:faraday-rotation} eingetragen.

Zur Berechnung der effektiven Masse der freien Leitungselektronen
bestimmen wir die auf die Probenlänge normierte Differenz $\Delta\theta$ der
Faraday-Rotationen von jeweils einer dotierten und der hochreinen Probe.
Gemäß Formel \eqref{eq:winkelfrei} kann ein lineares Regressionmodell
für die Meßdaten $y = (y_i) = (\Delta\theta_i)$ und $x = (x_i) =
(\lambda_i^2)$ zugrundegelegt werden:
%
\begin{equation}
  \label{eq:linregress}
  y_i = \beta_0 x_i + \beta_1.
\end{equation}
Die Ausgleichsrechnung wird von der Bibliothek \texttt{scipy.stats}
übernommen, die in der Version 0.10 verwendet wird und eine Funktion
\texttt{scipy.stats.linregress} enthält, die die Parameter $\beta_0$ und
$\beta_1$ und die Standardabweichung $s$ berechnet.  Gemäß den Formeln
%
\begin{align}
  \label{eq:stat-formeln}
  s_{\beta_0} &= \frac{s}{\sqrt{n}} \sqrt{1 + \frac{n\bar{x}^2}
    {\operatorname{var}(x)}}\\
  s_{\beta_1} &= \frac{s}{\sqrt{n \operatorname{var}(x)}}
\end{align}
%
werden dann die Standardabweichungen der Parameter geschätzt.  Das
Ergebnis der Rechnung ist in \cref{tab:linregress} numerisch und in
\cref{fig:linregress} graphisch gegeben.  Aus Formel
\eqref{eq:winkelfrei} wird abgelesen, dass für den Zusammenhang zwischen
effektiver Masse und dem Parameter $\beta_0$ gilt:
%
\begin{equation}
m^{*} = \sqrt{\frac{e_0^3}{8\pi^2\varepsilon_0c^3\beta_0} \frac{NB}{n}}
\label{eq:effekt}
\end{equation}
%
Der Brechungsindex $n = \num{3.455}$ von GaAs ist \cite{filmetrics}
entnommen. \Cref{tab:linregress} zeigt die Ergebnisse, die aus dieser
Formel erhalten werden.  Im nächsten Abschnitt werden diese Ergebnisse
kurz diskutiert.

\begin{table}\centering
  \begin{tabular}{SSSSSSSSSS}
    \toprule
    &
    \multicolumn{3}{c}{hochreine Probe} &
    \multicolumn{3}{c}{schwach dotierte Probe} &
    \multicolumn{3}{c}{stark dotierte Probe}
    \\
    \cmidrule(rl){2-4}
    \cmidrule(rl){5-7}
    \cmidrule(rl){8-10}
    {$\lambda/\si{\micro\meter}$} &
    {$\theta_1/\si{\degree}$} &
    {$\theta_2/\si{\degree}$} &
    {$\theta/\si{\degree}$} &
    {$\theta_1/\si{\degree}$} &
    {$\theta_2/\si{\degree}$} &
    {$\theta/\si{\degree}$} &
    {$\theta_1/\si{\degree}$} &
    {$\theta_2/\si{\degree}$} &
    {$\theta/\si{\degree}$}
    \\
    \midrule
    1.06 & 315. & 337.7 & 11.3 & 75. &  64.5 &  5.5 & 320.7 & 333. & 6.2\\
    1.29 & 335. & 320. & 7.5 & 64.7 & 72.6 & 3.9 & 333. & 323. & 5.\\
    1.45 & 320.3 & 333. & 6.3 & 74. & 62.4 & 5.8 & 321.3 &  330.7 & 4.7\\
    1.72 & 332.6 & 324. & 4.3 & 61.6 & 71.5 & 5.0 & 333. & 324. & 4.5\\
    1.96 & 330.8 & 337.3 & 3.3 & 336. & 329. & 3.5 & 328. & 339. & 5.5\\
    2.16 & 336. & 329.6 & 3.2 & 59. & 66. & 3.5 & 340. & 327. & 6.5\\
    2.34 & 355.8 & 358.5 & 1.4 & 43. & 35. & 4. & 350.5 & 369. & 9.3\\
    2.51 & 13. & 22. & 4.5 & 19.3 & 11.4 & 4.0 & 21. & 13. & 4. \\
    2.65 & 53.4 & 58.7 & 2.6 & 345. & 335.7 & 4.7 & 331.6 & 347. & 7.7\\ 
    \bottomrule
   \end{tabular}
   \caption{Hier sind die gemessenen Winkel $\theta_1$ und $\theta_2$
     eingetragen.  Gemäß \cref{eq:drehwinkel} sind die Winkel $\theta$
     ausgerechnet worden.  Die schwach dotierte Probe hat eine
     Ladungsträgerdichte $N=\SI{1.2e18}{cm^{-3}}$ und ist \SI{1.296}{mm}
       lang, die stark dotierte Probe hat $n=\SI{2.8e18}{cm^{-3}}$ und
       ist \SI{1.36}{mm} lang.}
  \label{tab:faraday-rotation}
\end{table}

\begin{figure}
  \centering
  \includegraphics[width=0.8\textwidth]{abbildungen/faraday-rotation}
  \caption{Hier ist die Faraday-Rotation der drei verschiedenen Proben
    gegen die Wellenlänge aufgetragen.  Das oberste Diagramm zeigt die
    Faraday-Rotation der hochreinen Probe, das mittlere das der 
    schwach dotierten Probe und im untersten ist die Faraday-Rotation 
    der stark dotierten Probe zu sehen.}
  \label{fig:faraday-rot}
\end{figure}

\begin{figure}
  \centering
  \includegraphics[width=0.8\textwidth]{abbildungen/effektive_masse}
  \caption{Hier sind die Differenzen der Faraday-Rotationeswinkel von 
    dotierter und hochreiner Probe im Bogenmaß gegen das
    Wellenlängenquadrat aufgetragen.  Nach Formel \eqref{eq:winkelfrei}
    ist ein linearer Zusammenhang zu vermuten.  Die eingezeichneten
    Geraden zeigen die Ergebnisse einer linearen Regressionsrechnung.}
  \label{fig:linregress}
\end{figure}

\begin{table}
  \centering
  \sisetup{
    table-figures-integer  = 1,
    table-figures-decimal  = 1,
    table-figures-uncertainty = 1,
    table-number-alignment = center,
    retain-unity-mantissa  = false
  }
  \begin{tabular}{lSSSS[table-figures-decimal=2]}
    \toprule
    {Dotierung} &
    {$\beta_0/(\SI{1e12}{\per\meter})$}&
    {$\beta_1/(\SI{1e-2}{\per\meter})$} &
    {$m^*/(\SI{1e-31}{kg})$} &
    {$m^*/m_\text{e}$}
    \\
    \midrule
    schwach& 1.8(13) & 3.2 & 1.4(5) & 0.15(5)\\
    stark  & 8.3(30) & 3.0 & 1.0(2) & 0.11(2)\\
    \midrule
    Mittelwerte & & & 1.2(3) & 0.13(4)\\
    \bottomrule
  \end{tabular}
  \caption{Hier sind die Ergebnisse der linearen Ausgleichsrechnung und
    die Berechnung der effektiven Masse dargestellt. Die Fehler der 
    effektiven Massen sind nach einer 
    Fehlerfortpflanzung von Formel~\eqref{eq:effekt} entstanden.
    Die Fehler für den Parameter
    $\beta_1$ sind nicht angegeben, da er mehrerer Größenordnungen über
    dem Parameter selbst liegt, in die Berechnung der effektiven 
    Masse aber nicht eingeht.}
  \label{tab:linregress}
\end{table}
