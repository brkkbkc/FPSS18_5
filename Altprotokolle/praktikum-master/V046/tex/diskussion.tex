% This work is licensed under the Creative Commons
% Attribution-NonCommercial 3.0 Unported License. To view a copy of this
% license, visit http://creativecommons.org/licenses/by-nc/3.0/.

\section{Diskussion}
Die Magnetfeldmessung zeigt die erwarteten Ergebnisse. Das Feld ist in
der Nähe der Probe maximal und zeigt über einen sehr kleinen Bereich
Homogenität.  Weit außerhalb fällt das Feld stark ab.

Die Bestimmung der Faraday-Rotationswinkel ist nicht optimal gelungen.
Die statistischen Fehler der effektiven Massen sind zwar sehr gering, da
die Fehler der Steigungen der Ausgleichsgeraden aus
Abb.\ref{fig:linregress} gering sind. Allerdings übersteigen die Fehler
der Achschenabschnittswerte diese um einige Größenordnungen.  Das
Verhältnis $m^*/m_\text{e}$ ist in \cite{ecee-colorado} für
Galliumarsenid mit \num{0.067} angegeben.  Unser Wert weicht mit
\num{0.13} um \SI{92}{\percent} ab.  Die großen Fehler der
Achsenabschnitte lassen sich zum Teil auf die beschädigten
Interferenzfilter zurückführen, von denen einige Kratzer oder Löcher
aufwiesen. Außerdem befindet sich am Ort der Probe kein perfekt
homogenes Magnetfeld, wie es leicht in Abbildung~\ref{fig:magnetfeld} zu
sehen ist und Anhand der Messwerte in Tabelle~\ref{tab:magnetmess}
abzulesen ist. Die Messwerte zeigen, dass das Magnetfeld sich über die
Probendicke von ca. \SI{1}{\milli\metre} bereits um mindestens
\SI{1}{\milli\tesla} ändert.

Es kann aber abschließend gesagt werden, dass der Faraday-Effekt 
eine Möglichkeit bietet, die effektive Masse von Elektronen in 
Halbleitern zu bestimmen. Für vertrauenswürdigere Messwerte sind aber 
bessere Geräte als die in diesem Versuch verwendeten Apparaturen 
notwendig. 
