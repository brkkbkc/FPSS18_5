% This work is licensed under the Creative Commons
% Attribution-NonCommercial 3.0 Unported License. To view a copy of this
% license, visit http://creativecommons.org/licenses/by-nc/3.0/.

\section{Auswertung}

\subsection{Bestimmung der Sättigungsstromstärke aus den Kennlinien}

In den \crefrange{fig:kennlinie-2_1A}{fig:kennlinie-2_6A} auf den
\cpagerefrange{fig:kennlinie-2_1A}{fig:kennlinie-2_6A} sind die
aufgenommenen Kennlinien zu sehen. Die horizontale Linie kennzeichnet
den Sättigungsstrom. In \cref{tab:saettigung} sind die abgelesenen Werte
mit den zugehörigen Heizspannungen und -strömen gelistet.

\begin{figure}
  \centering
  \includegraphics[width=.7\textwidth]{kennlinie-2_1A}
  \caption{Die Abbildung zeigt den Strom-Spannungsverlauf bei einem
    Heizstrom von \SI{2.1}{\ampere}. Die gestrichelte Linie kennzeichnet
  die Sättigungsstromstärke. Raumladungsgebiet und Sättigungsbereich
  sind nicht sehr deutlich voneinander zu unterscheiden.}
  \label{fig:kennlinie-2_1A}
\end{figure}

\begin{figure}
  \centering
  \includegraphics[width=.7\textwidth]{kennlinie-2_2A}
  \caption{Die Abbildung zeigt den Strom-Spannungsverlauf bei einem
    Heizstrom von \SI{2.2}{\ampere}. Die gestrichelte Linie kennzeichnet
    die Sättigungsstromstärke. Raumladungsgebiet und Sättigungsbereich
    sind nicht sehr deutlich voneinander zu unterscheiden.}
  \label{fig:kennlinie-2_2A}
\end{figure}

\begin{figure}
  \centering
  \includegraphics[width=.7\textwidth]{kennlinie-2_3A}
  \caption{Die Abbildung zeigt den Strom-Spannungsverlauf bei einem
    Heizstrom von \SI{2.3}{\ampere}. Die gestrichelte Linie kennzeichnet
    die Sättigungsstromstärke. Raumladungsgebiet und Sättigungsbereich
    sind nicht sehr deutlich voneinander zu unterscheiden.}
  \label{fig:kennlinie-2_3A}
\end{figure}

\begin{figure}
  \centering
  \includegraphics[width=.7\textwidth]{kennlinie-2_4A}
  \caption{Die Abbildung zeigt den Strom-Spannungsverlauf bei einem
    Heizstrom von \SI{2.4}{\ampere}. Die gestrichelte Linie kennzeichnet
    die Sättigungsstromstärke. Raumladungsgebiet und Sättigungsbereich
    sind nicht sehr deutlich voneinander zu unterscheiden.}
  \label{fig:kennlinie-2_4A}
\end{figure}

\begin{figure}
  \centering
  \includegraphics[width=.7\textwidth]{kennlinie-2_6A}
  \caption{Die Abbildung zeigt den Strom-Spannungsverlauf bei einem
    Heizstrom von \SI{2.6}{\ampere}. Die gestrichelte Linie kennzeichnet
    die Sättigungsstromstärke. Die Sättigungsstromdichte wird vermutlich
    nicht erreicht. Das Raumladungsgebiet ist nicht vom Sättigungsgebiet
    zu unterscheiden}
  \label{fig:kennlinie-2_6A}
\end{figure}

\begin{table}
  \centering
  \begin{tabular}{SSS}
    \toprule
    {$U$/V} & {$I$/A} & {$I_S$/A} \\
    \midrule
    4.0000e+00 & 2.1000e+00 & 0.171 \\
    4.5000e+00 & 2.2000e+00 & 0.400 \\
    5.0000e+00 & 2.3000e+00 & 0.797 \\
    5.0000e+00 & 2.4000e+00 & 1.537 \\
    6.0000e+00 & 2.6000e+00 & 3.020 \\
    \bottomrule
  \end{tabular}
  \caption{Aus den \crefrange{fig:kennlinie-2_1A}{fig:kennlinie-2_6A}
    abgelesene Sättigungsstromstärken. Es ist zu erkennen, daß der
    Sättigungsstrom sehr stark mit der Temperatur ansteigt.}
  \label{tab:saettigung}
\end{table}

\subsection{Der Gültigkeitsbereich des
  \name{Langmuir}-\name{Schottky}-Raumladungsgesetzes}

Auf \cref{fig:raumladung} ist der Strom-Spannungsverlauf in ein
Koordinatensystem eingetragen, dessen beide Achsen logarithmiert
sind. Das Raumladungsgebiet ist nur schwer zu erkennen. Ab dem
12. Messwert in etwa scheinen die Punkte von einer gedachten Geraden
abzuweichen, daher wird für die ersten 12 Punkte eine Ausgleichsrechnung
durchgeführt. Um eine lineare Ausgleichsrechnung durchzuführen, wird eine
Substitution durchgeführt.
%
\begin{equation*}
  y = \log I = \frac{3}{2} \log U + \log A = m x + b,
\end{equation*}
%
mit einer Konstanten $A$. Durch diese Ausgleichsrechnung kann überprüft
werden, wie genau der tatsächliche Exponent mit dem erwarteten
übereinstimmt. Für die Ausgleichsrechnung wird die Funktion
\texttt{scipy.stats.linregress} aus der \texttt{scipy}-Bibliothek in der
Version 0.12 verwendet. Die Ergebnisse lauten:
%
\begin{align*}
  m &= \num{1.206(16)} &
  b &= \num{-5.455(4)}\\
  \frac{|m - \frac{3}{2}|}{\frac{3}{2}} &= \SI{19.6}{\percent}
\end{align*}

\begin{figure}
  \centering
  \includegraphics[width=0.7\textwidth]{raumladung}
  \caption{Strom gegen Spannung im Anlaufgebiet halblogarithmisch
    aufgetragen.}
  \label{fig:raumladung}
\end{figure}

\subsection{Bestimmung der Kathodentemperatur aus dem Anlaufgebiet}

Bei der maximal möglichen Heizleistung kann durch Messung des
Anlaufgebiets die Kathodentemperatur bestimmt werden. Dazu trägt man die
Messpunkte von Strom und Spannung in ein Koordinatensystem ein, dessen
Ordinate logarithmiert ist (\cref{fig:anlaufgebiet}). Dann führt man
eine lineare Ausgleichsrechnung durch. Aus der Steigung $m$ erhält man
nach Gl.~(\eqref{eq:anlaufstromdichte}) die Temperatur
%
\begin{equation*}
  T = \frac{e}{m k}
\end{equation*}
Da die Steigung fehlerbehaftet ist, wird eine \name{Gauß}sche
Fehlerfortpflanzung durchgeführt:
%
\begin{equation*}
  \Delta T = \frac{e}{m^2k} \Delta m
\end{equation*}
%
Das Ergebnis für die Kathodentemperatur lautet dann:
%
\begin{equation*}
  T = \SI{2630(89)}{\kelvin}.
\end{equation*}

\begin{figure}
  \centering
  \includegraphics[width=0.7\textwidth]{anlaufgebiet}
  \caption{Strom gegen Spannung im Anlaufgebiet halblogarithmisch
    aufgetragen.}
  \label{fig:anlaufgebiet}
\end{figure}

\subsection{Kathodentemperatur aus einer Leistungsbilanz der Apparatur}

Über eine Leistungsbilanz der verwendeten Geräte der Apparatur kann die
Temperatur der Kathode ebenfalls bestimmt werden. Die zugeführte
Leistung $P_\text{zu} = UI$ wird über Wärmestrahlung und Wärmeleitung
von der Kathode abgegeben. Die Wärmeleitungsleistung wird auf
\SI{1}{\watt} geschätzt. Für die Wärmestrahlung gilt nach dem
\name{Stefan}-\name{Boltzmann}-Gesetz
%
\begin{equation*}
  P_\text{str} = f \eta \sigma T^4, 
\end{equation*}
%
wobei $f = \SI{0.35}{\squared\centi\metre}$ die emittierende
Kathodenoberfläche, $\eta = \num{0.28}$ der Emissionsgrad der Oberfläche
und $\sigma = \SI{5.7e-12}%
{\watt\per\centi\metre\squared\per\kelvin\tothe 4}$ ist die
\name{Stefan}-\name{Boltzmann}-Konstante. Es ergibt sich also der
folgende Term für die Kathodentemperatur:
%
\begin{equation}
  T = \sqrt[4]{\frac{UI - \SI{1}{\watt}}{f\eta\sigma}}
  \label{eq:leistungsbilanz}
\end{equation}
%
In \cref{tab:leistungsbilanz} stehen die Heizspannungen und -ströme der
fünf aufgenommenen Kennlinien mit ihren ermittelten
Kathodentemperaturen.

\begin{table}
  \centering
  \begin{tabular}{SSS}
    \toprule
    {$U$/V} & {$I$/A} & {$T$/K} \\
    \midrule
    4.0000e+00 & 2.1000e+00 & 1910 \\
    4.5000e+00 & 2.2000e+00 & 2000 \\
    5.0000e+00 & 2.3000e+00 & 2084 \\
    5.0000e+00 & 2.4000e+00 & 2109 \\
    6.0000e+00 & 2.6000e+00 & 2264 \\
    \bottomrule
  \end{tabular}
  \caption{Die Heizspannungen und -ströme der Kennlinien mit den
    zugehörigen nach \eqref{eq:leistungsbilanz} bestimmten
    Kathodentemperaturen.}
  \label{tab:leistungsbilanz}
\end{table}

\subsection{Bestimmung der Austrittsarbeit des Kathodenmaterials}

Die Gl.~\eqref{eq:richard} wird nach $\phi$ umgestellt und dann werden für
die zuvor bestimmten Temperaturen und Sättigungsströme eingesetzt. Die
Ergebnisse können in \cref{tab:austrittsarbeit} betrachtet werden. Für
den Wert der Austrittsarbeit ergibt sich:
%
\begin{equation*}
  \phi = \SI{4.30(05)}{\electronvolt}
\end{equation*}

\begin{table}
  \centering
  \begin{tabular}{SSS}
    \toprule
    { $U$/V }  & { $I$/A }  & {$\phi$/eV}\\
    \midrule
    1.7100e-01 & 1.9103e+03 & 4.1871e+00 \\
    4.0000e-01 & 2.0005e+03 & 4.2543e+00 \\
    7.9700e-01 & 2.0849e+03 & 4.3248e+00 \\
    1.5370e+00 & 2.1093e+03 & 4.2602e+00 \\
    3.0200e+00 & 2.2640e+03 & 4.4685e+00 \\
    \midrule
    \multicolumn{2}{c}{Mittelwert} & 4.30 \\
    \multicolumn{2}{c}{Standardabweichung} & 0.05 \\
    \bottomrule
  \end{tabular}
  \caption{Hier finden sich die aus den Sättigungsströmen und
    Temperaturen bestimmten Austrittsarbeiten des Kathodenmaterial. Für
    die Querschnittsfläche, durch den der Strom tritt, wird $f/2$ angenommen.}
  \label{tab:austrittsarbeit}
\end{table}
