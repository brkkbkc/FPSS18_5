\section{Diskussion}
\label{sec:Diskussion}
  Die gemessene Aktivit"at/Z"ahlrate der Am-Quelle ist $Z_{Quelle}=\SI{274}{\kilo \bequerel}$ und hat eine Abweichung von 14\% von der berechneten Aktivit"at $\si{318}{\kilo \bequerel}$.
  Da kein perfektes Vakuum im Geh"ause erzeugt werden konnte, war eine geringere Aktivit"at aufgrund von Wechselwirkung der $\alpha$-Teilchen mit verbliebenden Luftmolek"uhlen zu erwarten.
  
  Bei der Vermessung des Rutherford-Streugesetzes, zeigen die erhaltenen ($\theta,\frac{\text{d}\sigma}{\text{d}\Omega}(\theta)$)-Wertepaare, f"ur die Streuwinkelabh"angigkeit des differentiellen Wirkungsquerschnittes keine gute "Ubereinstimmung mit der Theoriekurve, wie sie durch das Rutherford-Streugesetz gegeben ist.
  Es ist lediglich zu erkennen, das die letzten 4 Messwerte ($\SI{7}{\degree}$ bis $\SI{20}{\degree}$) einer nicht-linearen abfallenden Funktion folgen.
  Diese Messwerte liegen dabei aber weit oberhalb des Theoriegraphen.

  Die ersten Messerte von $\SI{0}{\degree}$ bis $\SI{3}{\degree}$ liegen allerdings unterhalb des Graphen, und scheinen kaum einer Verteilung zu folgen, da sie eher willk"urlich Verteilt sind.

  Neben den Abweichungen von einem einheitlichen Kurvenverlauf, beschreiben die Messwerte insgesamt dennoch eine stark entlang y-Achse gestauchte Theoriekurve.
