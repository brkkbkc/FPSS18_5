\section{Zielsetzung}
  In diesem Versuch soll zuerst die Foliendicke einer $\SI{2}{\micro\meter}$ dicken Goldfolie experimentell bestimmt werden.
  Dies geschieht durch Verlustmessung der Energie von $\alpha$-Teilchen beim senkrechten Durchlauf der Folie, was durch die der Bethe-Bloch-Formel beshrieben ist.

  Weiterhin wird die Abh"angigkeit des differentiellen Wirkungsquerschnittes $\text{d}\sigma/\text{d}\Omega(\theta)$ vom Streuwinkel $\theta$ bei der Streuung von $\alpha$-Strahlung an der Goldfolie vermessen und mit dem Theorieverlauf nach der Rutherford-Streuformel verglichen.

  Zuletzt wird die die $Kernladungszahl/Z$-Abh"angigkeit des differentiellen Wirkungsquerschnittes nach der Rutherford-Streuformel "uberpr"uft.
  % und die Entstehung von Mehrfachreflexionen der $\alpha$-Teilchen in dicken Folien nachgewiesen.

\section{Theorie}
  \label{sec:Theorie}
  Bei der Wechselwirkung von $\alpha$-Teilchen mit Materie m"ussen vorwiegend zwei Effekte ber"ucksichtigt werden:
  \begin{itemize}
    \item 1) Energieverlust der $\alpha$-Teilchen druch inelastische St"o"se mit den Elektronen, beschrieben durch die Bethe-Bloch-Formel \cite{bethe}. Hierbei gibt es kaum eine Richtungs"anderung der Heliumkerne, da die deren Masse viel gr"o"ser ist, als die der Elektronen, d.\,h. $m_{\alpha}>>m_e$.
    \item 2) Elastische Streuung der $\alpha$-Teilchen an den geladen Atomkernen.
    Dieser Effekt wird durch die Rutherford-Streuformel beschrieben.
    %Inelastische Streuung, das bedeutet Energieverlust verbunden mit einer Richtungs"anderung, der $\alpha$-Teilchen an den Atomkernen.
    %Diesem Effekt liegt ebenfalls die Coulombwechselwirkung zugrunde und er wird durch die Rutherford-Streuformel beschrieben \cite{ruther}.
    %Dieser ebenfalls hervorgerufen durch Coulombwechselwirkung beschrieben durch die Rutherford-Streuformel.
  \end{itemize}


  \subsection{Die Bethe-Bloch-Formel}
    Die Bethe-Bloch-Formel beschreibt den Energieverlust schwerer geladener Teilchen pro Strecke $\text{d}E/\text{d}x$ in einem Material, welcher durch die inelastischen Streuungen des Teilchens mit den Elektronen der Materialatome entsteht.
    Es wird angenommmen, dass das Projektilteilchen durch die St"o"se im Material nicht von seiner bahn abgelenkt wird.
     %als Funktion der Teilchengeschwindigkeit $v$, der Teilchenladungszahl $z$ und als Funktion der Kernladungszahl $Z$ der Materialatome, der Teilchenzahl pro Volumen $N$ und der mittleren Ionisationsenergie $I$ der Materialelektronen.
    %Bei mehrkomponentigen Materialien kann mit effektiven $N$ und $Z$ gerechnet werden.
    Die eigentliche Bethe-Bloch-Formel ist relativistisch, im nicht relativistischem Grenzfall ($v<<c$) ergibt sich:
    \begin{equation}
      \frac{-\text{d}E}{\text{d}x} = \frac{4\pi e^4z^2NZ}{m_ev^2(4\pi \epsilon_0)^2}\ln \left(\frac{2m_ev^2}{I}\right) \: .
    \end{equation}
    In obiger Formel ist $z$ die Ladungszahl des Projektilteilchens und $v$ dessen Geschwindigkeit und $m_e$ ist die Ruhenergie des Elektrons.
    Weitherhin ist $N$ die Teilchenzahl pro Volumen im Material, $Z$ die Kernladungszahl der Materialatome und $I$ das mittlere Ionisationspotential der Materialelektronen.

    In der N"aherung, dass $v=\sqrt{2E_{\alpha}/m}=v_{\alpha}$ konstant ist, mit $E_{\alpha}$ der urspr"unglichen Energie der Teilchen (hier $\alpha$-Teilchen), kann die Formel integriert werden und eine N"aherung des absoluten Energieverlustes $E$ nach der Strecke $x$ erhalten werden:
    \begin{equation}
      E = \frac{4\pi e^4z^2NZ}{m_ev^2(4\pi \epsilon_0)^2}\ln \left(\frac{2m_ev^2}{I}\right) \cdot x
      \label{bethe} \: .
    \end{equation}
    Der Energieverlust $E$ ist somit linear abh"angig von der Strecke $x$.



  \subsection{Rutherford-Streuformel}
    Die Rutherford-Streuformel beschreibt den differentiellen Wirkungsqerschnitt $\text{d}\sigma/\text{d}\Omega(\theta)$ (Wirkungsquerschnitt $\sigma$ pro Raumwinkel $\Omega$) der elastischen Streuung von geladenen Projektilteilchen, als Funktion des Streuwinkels $\theta$ (im Gradma"s).
    Das Targetteilchen wird dabei als ruhend angenommen und Spin-Kopplungen werden vernachl"assigt.
    Die Rutherford-Streuformel ist gegeben durch
    \begin{equation}
      \frac{\text{d}\sigma}{\text{d}\Omega}(\theta) = \frac{1}{(4\pi \epsilon_0)^2} \left( \frac{zZe^2}{4E_{\alpha}} \right)^2 \frac{1}{\sin^4(\frac{\theta}{2})}
      \label{ruther}
    \end{equation}
    mit der Kernladungszahl der Projektilteilchen $z$ und der Materialatome $Z$, sowie der urspr"unglichen ($\alpha$-)Teilchenenergie $E_{\alpha}$ vor dem Wechselwirken mit Materie.
    Der differentielle Wirkungsquerschnitt ist dabei definiert als \cite{omega}:
    \begin{equation}
      \frac{\text{d}\sigma}{\text{d}\Omega}(\theta) = \frac{Z_{\text{out}}(\theta)}{Z_{\text{Folie}}} \frac{1}{\Delta \Omega} \frac{1}{N_{\text{A,Folie}}}
    \end{equation}
    Hierbei ist $Z_{out}(\theta)$ die beim Streuwinkel $\theta$ gemessene Teilchenz"ahlrate/Aktivit"at nach der Streuung an der Folie.
    Die Gr"o"se $N_{A,Folie}$ ist die Teilchenfl"achendichte des Materials/der Folie, das bedeutet
    \begin{equation}
      N_{{A.Folie}} = N_{\text{Folie}} \cdot d
    \end{equation}
    mit der Teilchenvolumendichte $N$ der Folie und der Dicke $d$ der Folie.
    $Z_{Folie}$ ist die Z"ahlrate/Aktivit"at der $\alpha$-Teilchen auf der Folie, welche sich aus der Versuchsgeometrie nach Abb. (\ref{fig:aufbau}) berechnet zu
    \begin{equation}
      Z_{Folie}=Z_{Quelle}\frac{F}{4\pi(\SI{0.039}{\meter}+\SI{0,017}{\meter})^2}
    \end{equation}
    mit der Blendenfl"ache $F=\SI{20e-6}{\meter}$ \cite{Anleitung}.
    %und der Folienfl"ache $A_{Folie}=\textbf{????????}$.
    Dabei wird ,wegen der sehr engen Blende vor der Folie, davon ausgegangen, dass alle Teilchen die durch die Blende gelangen auch auf die Folie treffen.
    %Es wird davon ausgegangen, dass die erste Blende direkt an der Quelle liegt und irrelevant ist.
    $\Delta \Omega$ ist der Raumwinkel der SB-Detektorfl"ache, welcher sich ebenfalls aus den Versuchsabst"anden berechnen l"asst. Da knapp vor dem Detektor wieder eine enge Blende ist, werden alle Teilchen die durch die Blende gelangen auch auf die Detektorfl"ache einfallen. Damit folgt:
    \begin{equation}
      \Delta \Omega = \frac{F}{4\pi(\SI{0.039}{\meter}+\SI{0.017}{\meter}+\SI{0.041}{\meter})^2} \: .
    \end{equation}

    Da im Versuch der Streuwinkel $\theta$ nicht durch Ver"andern der Detektorpostion eingestellt wird, sondern durch Drehen der Folie im Strahl, ver"andert sich mit $\theta$ auch die effektive Folienfl"ache senkrecht zur Strahlrichtung, damit ist $A_{Folie}(\theta)$ abh"angig vom Streuwinkel:
    \begin{equation}
      A_{Folie}(\theta) = A_{Folie}\cos\left(\frac{\theta}{2}\right) \:.
    \end{equation}

    Insgesamt ist der differentielle Wirkungsquerschnitt $\text{d}\sigma/\text{d}\Omega(\theta)$ damit:
    \begin{equation}
      \frac{\text{d}\sigma}{\text{d}\Omega}(\theta) =  \frac{Z_{out}(\theta)4\pi^2(\SI{0,039}{\meter}+\SI{0,017}{\meter})^2}{Z_{Quelle}F\cos(\theta/2)}
      \frac{4\pi^2 (\SI{0,041}{\meter}+\SI{0,039}{\meter}+\SI{0.017}{\meter})^2}{F}\frac{1}{N_{A,Folie}}
      \label{querschnitt}
    \end{equation}

    Bei gemessenem $Z_{out}(\theta)$ k"onnen damit nun nun die $(y=\frac{\text{d}\sigma}{\text{d}\Omega})$-Werte zu den gemessenen $(x=\theta)$-Werten berechnet werden.
    %sodass an die im Diagramm afgetragenen ($\frac{\text{d}\sigma}{\text{d}\Omega},\theta$)-Punkte die Rutherford-Streuformel (\ref{ruther}) gefittet werden kann.
