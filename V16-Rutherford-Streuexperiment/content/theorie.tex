\section{Zielsetzung}
  In diesem Versuch soll zuerst die Foliendicke einer $\SI{2}{\micro\meter}$ dicken Goldfolie bestimmt werden, durch Verlustmessung der Energie von $\alpha$-Teilchen beim senkrechten Durchlauf der Folie, mit Hilfe der Bethe-Bloch-Formel.

  Weiterhin wird die Abh"angigkeit des differentiellen Wirkungsquerschnittes $\text{d}\sigma/\text{d}\Omega(\theta)$ vom Streuwinkel $\theta$ bei der Streuung von $\alpha$-Strahlung an der Goldfolie vermessen, und mit dem Theorieverlauf nach der Rutherford-Streuformel verglichen.

  Zuletzt wird die die $Z$-Abh"angigkeit des differentiellen Wirkungsquerschnittes nach der Rutherford-Streuformel "uberpr"uft und die Entstehung von Mehrfachreflexionen der $\alpha$-Teilchen in dicken Folien nachgewiesen.

\section{Theorie}
  \label{sec:Theorie}
  Bei der Wechselwirkung von $\alpha$-Teilchen mit Materie kommt es "uberwiegend zu zwei Effekten:
  \begin{itemize}
    \item 1) Energieverlust der $\alpha$-Teilchen durch Coulombwechselwirkung mit den Elektronen der Materialatome, beschrieben durch die Bethe-Bloch-Formel. Hierbei gibt es kaum eine Richtungs"anderung der Heliumkerne, da $m_{\alpha}>>m_e$.
    \item 2) Inelastische (Energieverlust) Streuung (Richtungs"anderung) der $\alpha$-Teilchen an den Atomkernen, ebenfalls durch Coulombwechselwirkung, beschrieben durch die Rutherford-Streuformel.
  \end{itemize}


  \subsection{Bethe-Bloch-Formel}
    Die Bethe-Bloch-Formel beschreibt den Energieverlust geladener Teilchen pro Strecke $\text{d}E/\text{d}x$ in einem Material, als Funktion der Teilchengeschwindigkeit $v$, der Teilchenladungszahl $z$ und als Funktion der Kernladungszahl $Z$ der Materialatome, der Teilchenzahl pro Volumen $N$ und der mittleren Ionisationsenergie $I$ der Materialelektronen.
    Bei mehrkomponentigen Materialien kann mit effektiven $N$ und $Z$ gerechnet werden.
    Die eigentliche Bethe-Bloch-Formel ist relativistisch, im nicht relativistischem Grenzfall ($\beta \approx 0$) ergibt sich:
    \begin{equation}
      \frac{-\text{d}E}{\text{d}x} = \frac{4\pi e^4z^2NZ}{m_ev^2(4\pi \epsilon_0)^2}\ln \left(\frac{2m_ev^2}{I}\right) \: .
    \end{equation}

    In der N"aherung, dass $v=\sqrt{2E_{\alpha}/m}=v_{\alpha}$ konstant ist, mit $E_{\alpha}$ der urspr"unglichen Energie der Teilchen (hier $\alpha$-Teilchen), kann die Formel integriert werden und eine N"aherung des absoluten Energieverlustes $E$ nach der Strecke $x$ erhalten werden:
    \begin{equation}
      E = \frac{4\pi e^4z^2NZ}{m_ev^2(4\pi \epsilon_0)^2}\ln \left(\frac{2m_ev^2}{I}\right) \cdot x
      \label{bethe} \: .
    \end{equation}



  \subsection{Rutherford-Streuformel}
    Die Rutherford-Streuformel beschreibt den differentiellen Wirkungsqerschnitt $\text{d}\sigma/\text{d}\Omega(\theta)$ (Wirkungsquerschnitt $\sigma$ pro Raumwinkel $\Omega$) als Funktion des Streuwinkels $\theta$ (im Gradma"s) und ist
    \begin{equation}
      \frac{\text{d}\sigma}{\text{d}\Omega}(\theta) = \frac{1}{(4\pi \epsilon_0)^2} \left( \frac{zZe^2}{4E_{\alpha}} \right)^2 \frac{1}{\sin^4(\frac{\theta}{2})}
      \label{ruther}
    \end{equation}
    mit der Kernladungszahl der Teilchen $z$ und der Materialatome $Z$ und der urspr"unglichen ($\alpha$-)Teilchenenergie $E_{\alpha}$ vor dem Wechselwirken mit Materie.
    Der differentielle Wirkungsquerschnitt ist dabei definiert als \cite{omega}:
    \begin{equation}
      \frac{\text{d}\sigma}{\text{d}\Omega}(\theta) = \frac{Z_{out}(\theta)}{Z_{Folie}} \frac{1}{\Delta \Omega} \frac{1}{N_{A,Folie}}
    \end{equation}
    Hierbei ist $Z_{out}(\theta)$ die beim Streuwinkel $\theta$ gemessene Teilchenz"ahlrate/Aktivit"at nach der Streuung an der Folie.
    $N_{A,Folie}$ ist die Teilchenfl"achendichte des Materials/der Folie, das bedeutet
    \begin{equation}
      N_{A.Folie} = N_{Folie} \cdot d
    \end{equation}
    mit der Teilchenvolumendichte $N$ und der Dicke $d$ der Folie.
    $Z_{Folie}$ ist die Z"ahlrate/Aktivit"at der $\alpha$-Teilchen auf der Folie, welche sich aus der Versuchsgeometrie nach Abb. (\ref{fig:aufbau}) berechnet zu
    \begin{equation}
      Z_{Folie}=Z_{Quelle}\frac{F}{4\pi(\SI{0.039}{\meter}+\SI{0,017}{\meter})^2}
    \end{equation}
    mit der Blendenfl"ache $F=\SI{20e-6}{\meter}$ \cite{Anleitung}.
    %und der Folienfl"ache $A_{Folie}=\textbf{????????}$.
    Dabei wird ,wegen der sehr engen Blende vor der Folie, davon ausgegangen, dass alle Teilchen die durch die Blende gelangen auch auf die Folie treffen.
    %Es wird davon ausgegangen, dass die erste Blende direkt an der Quelle liegt und irrelevant ist.
    $\Delta \Omega$ ist der Raumwinkel der SB-Detektorfl"ache, welcher sich ebenfalls aus den Versuchsabst"anden berechnen l"asst. Da knapp vor dem Detektor wieder eine enge Blende ist, werden alle Teilchen die durch die Blende gelangen auch auf die Detektorfl"ache einfallen. Damit folgt:
    \begin{equation}
      \Delta \Omega = \frac{F}{4\pi(\SI{0.039}{\meter}+\SI{0.017}{\meter}+\SI{0.041}{\meter})^2} \: .
    \end{equation}

    Da im Versuch der Streuwinkel $\theta$ nicht durch Ver"andern der Detektorpostion eingestellt wird, sondern durch Drehen der Folie im Strahl, ver"andert sich mit $\theta$ auch die effektive Folienfl"ache senkrecht zur Strahlrichtung, damit ist $A_{Folie}(\theta)$ abh"angig vom Streuwinkel:
    \begin{equation}
      A_{Folie}(\theta) = A_{Folie}\cos\left(\frac{\theta}{2}\right) \:.
    \end{equation}

    Insgesamt ist der differentielle Wirkungsquerschnitt $\text{d}\sigma/\text{d}\Omega(\theta)$ damit:
    \begin{equation}
      \frac{\text{d}\sigma}{\text{d}\Omega}(\theta) =  \frac{Z_{out}(\theta)4\pi^2(\SI{0,039}{\meter}+\SI{0,017}{\meter})^2}{Z_{Quelle}F\cos(\theta/2)}
      \frac{4\pi^2 (\SI{0,041}{\meter}+\SI{0,039}{\meter}+\SI{0.017}{\meter})^2}{F}\frac{1}{N_{A,Folie}}
      \label{querschnitt}
    \end{equation}

    Bei gemessenem $Z_{out}(\theta)$ k"onnen damit nun nun die $(y=\frac{\text{d}\sigma}{\text{d}\Omega})$-Werte zu den gemessenen $(x=\theta)$-Werten berechnet werden.
    %sodass an die im Diagramm afgetragenen ($\frac{\text{d}\sigma}{\text{d}\Omega},\theta$)-Punkte die Rutherford-Streuformel (\ref{ruther}) gefittet werden kann.


















    
