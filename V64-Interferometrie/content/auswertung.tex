\section{Auswertung}
\label{sec:Auswertung}

 \subsection{Kontrast eines Interferometers}
Der Kontrast rechnet sich aus der Gleichung (\ref{kontrast}). Dafür werden die Werte aus der Folgenden Tabelle betrachtet.

  \begin{table}
 \centering
 \begin{tabular}{c|c|c|c}

 $\phi$ in Grad 	&	$U_{Max}/ \si{\volt}$&	$U_{Min}/ \si{\volt}$	&	$K$	\\
\hline
90	&	-4,1	&	-5	&	-0,10	\\
80	&	-2,6	&	-4,98	&	-0,31	\\
70	&	-1,53	&	-4,72	&	-0,51	\\
60	&	-0,87	&	-4,03	&	-0,64	\\
50	&	-0,656	&	-3,41	&	-0,68	\\
45	&	-0,593	&	-3,25	&	-0,69	\\
40	&	-0,656	&	-2,97	&	-0,64	\\
30	&	-0,781	&	-2,87	&	-0,57	\\
20	&	-1,219	&	-3,06	&	-0,43	\\
10	&	-2,188	&	-3,56	&	-0,24	\\
0	&	-3,344	&	-4,09	&	-0,10	\\

 \end{tabular}
 \caption{Gemessene Spannung und Kontrast in Abhängigkeit vom Polarisationswinkel}
 \end{table}

 Der Kontrast wird in Abhängigkeit des Polarisationswinkels in Abbildung (\ref{fig:plot1}) aufgetragen.
 Dabei wird eine Ausgleichsfunktion die durch

  \begin{equation}
    K = a \cdot \mid \sin(b(\phi-c)) \mid +d
    \end{equation}

    gegeben ist, an die Messwerte angepasst. Die Ausgleichsfunktion besitzt die Parameter


\begin{center}
$a= 0,62 \pm 0,02 $,
\end{center}

\begin{center}
$b= 1,98 \pm 0,02 $,
\end{center}

\begin{center}
$c=  0,02 \pm 0,01 $
\end{center}
\begin{center}
$d= 0,07 \pm 0,01 $.
\end{center}

\begin{figure}
  \centering
  \includegraphics[height=8cm]{plots/Plot_V64_1.pdf}
  \caption{Kontrast in Abhängigkeit des Polarisationswinkels mit nicht linearer Ausgleichsfunktion}
  \label{fig:plot1}
\end{figure}
Der Abbildung ist zu entnehmen, dass

 \subsection{Brechungsindex von Glas}
 Der Polarisationswinkel wird auf maximalen Kontrast gestellt, der Brechungsindex wird mithilfe von Gleichung (\ref{glas}) bestimmt.
Es gelten für die Berechnung des Brechungsindex der Glasplatten folgende Konstanten:

 \begin{align*}
 %\phi_0 &= 10° \\
  \lambda &= 632,99\, \text{nm} \\
  T &= 1 \, \text{mm} \; \text{\cite{Anleitung}}
 \end{align*}



  \begin{table}
 \centering
 \begin{tabular}{c|c|c|c|c}

Winkel $\theta$ in Grad	&	Messung 1	&	Messung 2	& Messung 3 & Messung 4 \\

 \hline
2	&	7	&	7	& 	8	&	7 \\
4	&	13	&	13	& 	14	&	14 \\
6	&	20	&	20	& 	22	& 	20 \\
8	&	28	&	28	&	27 	& 	27  \\
10	&	36	&	34	&	34 	& 	34 \\
   \end{tabular}
 \caption{Messwerte zur Bestimmung des Brechungsindex von Glas}
 \end{table}

In folgender Tabelle ist der berechnete Brechungsindex zu sehen. Dabei wird jewails die maximale Countzahl bzw. maximale Zahl der Interferenzmaxima $M$ benutzt, das hei"st, es wird der gr"o"ste Winkel betrachtet.
%Nun wird f"ur alle Messungen der Brechungsindex $n$ von Luft bei Atmospherendruck nach Formel (\ref{gas}) berechnet, das hei"st f"ur die letzte Countanzahl/Interferenzmaximazahl $M$, also die gesamte Interferenzmaximazahl bis zum Wiedererreichen des Atmospherendrucks.

 \begin{table}
 \centering
 \begin{tabular}{c|c|c|c}

 Messung	&	Counts/$M$	&	n	& Fehler \\

 \hline
1	&	36	&	1,56	& 0,03 \\
2	&	34	&	1,55	& 0,01\\
3	&	34	&	1,55	& 0,01 \\
4	&	34	&	1,55	& 0,01 \\
\hline

 \multicolumn{2}{c}{$\overline{n}$} & \multicolumn{2}{c} {1,56 $\pm$ 0,02} \\
\hline
 \end{tabular}
 \caption{Der Brechungsindex für den Maximalwinkel $\theta=\SI{10}{\degree}$ aller vier Messreihen mit der Gleichung (\ref{glas}) berechnet.}
 \end{table}
Der Durchschnittswert des Brechungsindexes zeigt eine gute "Ubereinstimmung mit dem Literaturwert von Kronglas: $n_{Glas,Lit}=1,52$ (\cite{glas}).

\subsection{Brechungsindex von Luft}

In Tabelle (4) sind die gemessenen Counts von Luft in Abh"angigkeit des Druckes notiert.



 \begin{table}
 \centering
 \begin{tabular}{c|c|c|c}
\label{test}
Druck in $\si{\milli \bar}$	&	Messung 1	& 	Messung 2	&  Messung 3 \\
	\hline
	& \multicolumn{3}{c}{Counts} \\
100	&	4	&	4	&	5 \\
200	&	9	&	9	&	9 \\
300	&	13	&	13	& 	13 \\
400	&	17	&	17	&	17 \\
500	&	21	&	21	&	21 \\
600 	&	25	&	25	&	25 \\
700	&	29	&	29	& 	30 \\
800	&	34	&	34	&	34 \\
900	&	38	&	38	&	38 \\
1000 &	42	&	42	&	42 \\

 \end{tabular}
 \label{test}
  \caption{Messwerte zur Bestimmung des Brechungsindex von Luft}
 \end{table}
In der Tabelle (5) sind die berechneten Werte für den Brechungsindex von Luft einzusehen.
Es wird f"ur alle Messungen der Brechungsindex $n$ von Luft bei Atmospherendruck nach Formel (\ref{gas}) berechnet, das hei"st f"ur die letzte Countanzahl/Interferenzmaximazahl $M$, also die gesamte Interferenzmaximazahl bis zum Wiedererreichen des Atmospherendrucks.
Die L"ange $L$ des optischen Elementes ist dabei:
\begin{equation}
  L = \SI{100}{\milli \meter}
\end{equation}

 \begin{table}
 \label{test2}
 \centering
 \begin{tabular}{c|c|c}
\label{test2}
 Messung	&	Counts	&	n	 \\

 \hline
1	&	42	&	1,000266 \\
2	&	42	&	1,000266 \\
3	&	42	&	1,000266  \\
\hline

 \multicolumn{2}{c}{$\overline{n}$} &1,000266 $\pm$ 0,0 \\
\hline
 \end{tabular}
 \caption{Der Brechungsindex für alle drei Messreihen nach Gleichung (\ref{gas})}
 \end{table}

Der Literaturwert von Luft (bei $\SI{1013}{\milli \bar}$) liegt bei $n_{Luft, Lit}=1,000292$ (\cite{luft}) und somit liegt die relative Abweichung bei $0,00001 \%$.
