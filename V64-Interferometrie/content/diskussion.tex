\section{Diskussion}
\label{sec:Diskussion}

  Der ermittelte maximale Kontrast $K(\phi)$ des Sagnac-Interferometers ist ca. $0,69$, und liegt ungef"ahr bei $\SI{45}{\degree}$, wie Abbildung (\ref{fig:plot1}) zu entnehmen ist.
  Dieser Maximalwert ist zwar ziemlich gering, aber der gefittete Graphenverlauf zeigt gute "Ubereinstimmung mit den Messwerten und alle Parameter weisen einen geringen Fehler auf, was die G"ute des Fits best"atigt.

  % der erhaltene Vorfaktor $b$ im Sinus liegt mit $1,98$ sehr nah an dem Theoriewert von 2 (siehe Formel (\ref{kontrast2})), was f"ur die Messung spricht.

  Da drei der vier Countzahlen/Maximazahlen bei der Messung des Brechungsundexes von Glas identisch sind ($M$=34) und Messung 1 mit $M=36$ eine geringe Abweichung zeigt, sind auch die Brechungsindizes nahezu identisch, und es ergibt sich der Mittelwert
  \begin{equation}
    \bar{n}=1,56 \pm 0,02
  \end{equation}
  mit einem sehr geringen Fehler.
  Der Brechungsindex liegt im Bereich der Krongl"aser und stimmt mit einer Abweichung von 0,6\% fast mit dem Literaturwert von Barium-Kronglas
  \begin{equation}
    n_{Lit}=1,57 \; \text{\cite{glas}}
  \end{equation}
  "uberein.



  Bei den drei Messungen von $M$ f"ur den Brechungsindex von Luft bei Atmospherendruck sind sogar alle Werte identisch ($M$=42) und somit auch die Werte f"ur den Brechungsindex, wodurch auch der Mittelwert mit einem Fehler von null diesem Wert entspricht:
  \begin{equation}
    \bar{n} = 1,000266 \pm 0,0
  \end{equation}
  Dieser Wert ist fast identisch zu dem Literaturwert $n=1,000292$ (\cite{luft}) von Luft bei einem Druck von $\SI{1013}{\milli \bar}$.
  Das Lorentz-Lorenz-Ges"atz l"asst vermuten, dass dieser "Abweichung Temperaturunterschiede zugrunde liegen.
  Das Lorentz-Lorenz-Gesetz beschreibt die Abh"angigkeit des Brechungsindexes von Luft von der Anzahl der Molek"uhle pro Volumen $N$, also auch von der Dichte der Luft:
  \begin{equation}
    \frac{n^2-1}{n^2+1} = \frac{4\pi}{3}N\alpha
  \end{equation}
  $\alpha$ ist hierbei die Polarisierbarkeit.
  %Solch geringe Abweichungen k"onnen schon bei der Druck"anderung von Luft nur durch "Anderung der Au"sentemperatur entstehen.

  Im Versuch konnte insgesamt festgestellt werden, dass mit dem Sagnac-Interferometer extrem genaue Messungen der Brechungsindizes m"oglich sind, was vorallem durch den bestimmten Brechungsindex von Luft best"atigt wird.
  Denn dieser liegt so nah am Literaturwert, dass (neben statistischem Fehler) auch von Temperaturunterschieden als Grund f"ur die Abweichung ausgegangen kann.
  Bekr"aftigt wird die G"ute des Sagnac-Interferometers auch durch die Tatsache, dass selbst mit vergleichsweise geringem Kontrast sehr gute Messergebnisse erhalten wurden.
