% This work is licensed under the Creative Commons
% Attribution-NonCommercial 3.0 Unported License.  To view a copy of
% this license, visit http://creativecommons.org/licenses/by-nc/3.0/.

\section{Auswertung}
%
\subsection{Gütebestimmung des Selektivverstärkers}
%
Der Funktionengenerator speist die Brücke mit einer Effektivspannung von
\SI{70}{\milli\volt}, welche der Selektivverstärker mit dem
Verstärkungsfaktor 10 verstärkt.  In Tabelle~\ref{tab:guetewerte} sind
die um diesen Faktor verstärkten, gemessenen Ausgangsspannungen zu
verschiedenen Frequenzen des Eingangssignals angegeben. Der
Selektivverstärker hat sein Resonanzfrequenz bei
ca.~\SI{35}{\kilo\hertz}, weswegen in diesem Bereich mehrere Meßwerte
genommen worden sind, um eine bessere Auflösung im Bereich der Spitze zu
erzielen.

Ein Plot dieser Messwerte ist in Abb.~\ref{fig:gueteplot} zu sehen.  Die
rot markierten Punkte, welche als Gütepunkte bezeichnet werden, werden
verwendet, um die Frequenzen $\nu_-$ und $\nu_+$ zu bestimmen.  Die
Filterfrequenz ist ebenfalls als vertikaler schwarzer Strich
eingezeichnet.  Die Frequenz der beiden Gütepunkte, welche links von der
Filterfrequenz liegen betragen \SI{35.20}{\kilo\hertz} für den unteren
und \SI{35.25}{\kilo\hertz} für den oberen Punkt.  Auf der rechten Seite
befinden sich die Gütepunkte mit den Frequenzen \SI{35.5}{\kilo\hertz},
welche dem oberen Punkt zugeordnet ist, und \SI{35.55}{\kilo\hertz} für
den unteren rechten Gütepunkt.  Durch Bildung des Mittelwertes der
Frequenzen der Gütepunkte aus der linken Seite wird die Frequenz $\nu_-$
bestimmt.  Auf der rechten Seite der Filterfrequenz wird analog
vorgangen.  Als Ergebnisse ergeben sich, dass
%
\begin{align*}
  \nu_- &= \SI{35.225}{\kilo\hertz} &
  \nu_+ &= \SI{35.525}{\kilo\hertz}.
\end{align*}
%
Mit Formel~\eqref{eq:guete} berechnet sich die Güte des
Selektivversärkers zu
%
\begin{align*}
  Q &= \num{118}.
\end{align*}
%
Eine Fehlerangabe ist nicht möglich, da die Fehler der einzelnen 
Frequenzmessungen unbekannt sind.

\begin{table}
  \centering
  \sisetup {
    table-number-alignment = center,
    table-figures-decimal  = 4,
    table-figures-integer  = 2
  }
  \begin{tabular}{SS|SS|SS}
    \toprule
    {$U_\text{A}/\si{\volt}$}& {$f / \si{\kilo\hertz}$}&
    {$U_\text{A}/\si{\volt}$}& {$f / \si{\kilo\hertz}$}& 
    {$U_\text{A}g/\si{\volt}$}& {$f / \si{\kilo\hertz}$}\\
    \midrule
    0.65&35.42&0.7&35.38&0.017&42\\
    0.55&35.3&0.54&35.5&0.009&50\\
    0.35&35.15&0.39&35.6&0.69&35.4\\
    0.23&35&0.21&35.85&0.68&35.35\\
    0.15&34.77&0.16&36&0.43&35.2\\
    0.092&34.30&0.11&36.3&0.51&35.25\\
    0.041&33&0.073&36.8&0.63&35.45\\
    0.0175&30&0.049&37.5&0.45&35.55\\
    0.0085&25&0.040&38& & \\
    0.0054&20&0.0235&40& & \\
    \bottomrule
  \end{tabular}
  \caption{Die gemessenen Ausgangsspannungen 
    des Selektivverstärkers für verschiedenen Frequenzen 
    des Eigangssignals.  Der Verstärkungsfaktor 
    beträgt 10.  Um die Durchlassfrequenz von ca. \SI{35}{\kilo\hertz}
    sind mehrere Meßwerte aufgenommen worden, um eine höhere Auflösung
    der Resonanzspitze zu erhalten.}
  \label{tab:guetewerte}
\end{table}

\begin{figure}
  \centering
  \includegraphics[width=0.8\textwidth]{gueteplot}
  \caption{Plot zur Gütebestimmung des verwendeten Selektivverstärkers.
    Eingestellte Güte: 100.  Die Gütepunkte wurden verwendet, um die
    dazu benötigten Frequenzen $\nu_-$ und $\nu_+$ zu errechnen.  Die
    Resonanzspitze ist scharf und schmal, was bei einer Güte von 100 zu
    erwarten ist.}
  \label{fig:gueteplot}
\end{figure}

\FloatBarrier
\subsection{Suszeptibilität der untersuchten Proben}
%
Die gemessenen Brückenspannungen $U_\text{Br}$ nach dem Einführen der
Probe in die Messpule, die zum Erreichen eines Spannungsminimums
eingestellten Widerstände $R_3$ und die Widerstandsdifferenz $\Delta
R_3$, welche nötig ist, um die Spannung bei probengefüllter Spule wieder
auf ein Minimum zu bringen, sind in Tabelle~\ref{tab:suszeptwerte} für
die drei untersuchten Materialien eingetragen.  Die Daten der
verwendeten Spule lauten:
%
\begin{itemize}
\item Windungszahl $n = 250$
\item Querschnittsfläche $F = \SI{86.6}{\milli\metre^2}$
\item Länge $l = \SI{13.5}{\centi\metre}$
\item Verlustwiderstand $R = \SI{0.7}{\ohm}$.
\end{itemize}

Aus den Brückenspannungen $U_\text{Br}$ und den Spulendaten werden mit
Hilfe von Formel~\eqref{eq:chi-spannungen} die Suszeptibilitäten
$\chi_\text{U}$ bestimmt.  Es wird hierbei nicht die Näherungsformel
verwendet, welche für $\omega^2 L^2 \gg R^2$ gilt, da $\frac{\omega^2
  L^2}{R^2} \approx 6$ ist.  Die Suszeptibilitäten $\chi_\text{R}$
werden mit Formel~\eqref{eq:delta-r}, und den in
Tabelle~\ref{tab:suszeptwerte} angegebenen Widerständen $R_3$ und den
Widerstandsdifferenzen $\Delta R_3$ berechnet.  Es ist unbedingt zu
beachten, dass sich in Reihe mit dem Widerstand $R_3$ ein Vorwiderstand
von $R = \SI{998}{\ohm}$ befindet, welcher in der verwendeten Formel zu
$R_3$ hinzu addiert werden muss. Für $Q$ wird nicht der Querschnitt der
staubartigen Proben verwendet, sondern der errechnete Wert
$Q_\text{real}$, der angibt, welchen Querschnitt die Probe hätte, wenn
diese ein Einkristall wäre.  In Tabelle~\ref{tab:qreal} sind die mit
Formel~\eqref{eq:qreal} errechneten Werte $Q_\text{real}$ für die
verwendeten Stoffe angegeben. $M_\text{p}$ bezeichnet hierbei die Masse
der Probe, $L$ dessen Länge und $\rho_\text{w}$ die Dichte des Stoffes.
%
\begin{equation}
Q_\text{real} = \frac{M_\text{p}}{L\cdot\rho_\text{w}}
\label{eq:qreal}
\end{equation}
%
Suszeptibilitätsbestimmungen sind ebenfalls in
Tabelle~\ref{tab:suszeptwerte} zu finden.  Mittelwertbildung und
Berechnung des statistischen Fehlers führen zu den in
\cref{tab:ergebnisse} dargestellten Ergebnissen.
%
\begin{table}
  \centering
  \begin{tabular}{SSSSS}
    \toprule
    {Material}&$M_\text{p}${/}\si{\gram}&
    {L/}\si{\centi\metre}&$\rho_\text{w}${/}\si{\gram\per\centi\metre^3}&
	$Q_\text{real}${/}\si{\centi\metre^2}\\
    \midrule
	{Dy2O3}&16.6&17.3&7.8&0.12\\
	{Nd2O3}&9.09&17.3&7.24&0.07\\
	{Gd2O3}&14.08&14.9&7.4&0.13\\
    \bottomrule
  \end{tabular}
  \caption{Daten und Querschnittsfläche der verwendeten Proben 
    $Q_\text{real}$, die diese besitzen würden, falls diese 
    einen Einkristall bilden würden.}
  \label{tab:qreal}
\end{table}
%
\begin{table}
  \centering
  \begin{tabular}{SSSS|SS}
    \toprule
    {Material}& $U_\text{Br}${ / }\si{\milli\volt}&
    {$R_{3} / \si{\milli\ohm}$}& $\Delta R_{3}${ / }\si{\milli\ohm}& 
    $\chi_\text{U}$ & $\chi_\text{R}$\\
    \midrule
    {Dy2O3}&101&2280&1715&0.0239&0.0188 \\
    {Masse = }\SI{16.6}{\gram}&98&2250&1685&0.0232&0.0185\\
    {Dichte = }\SI{7.8}{\gram\per\centi\metre^2}&98&2265&1685&0.0232&0.0185\\
    \midrule
    {Nd2O3}&14&2300&225&0.0056&0.0042\\
    {Masse = }\SI{9.0}{\gram} &15.5&2305&230&0.0062&0.0043\\
    {Dichte = }\SI{7.24}{\gram\per\centi\metre^2}&11&2255&175&0.0044&0.0036\\
    \midrule
    {Gd2O3}&51&2315&915&0.0135&0.0112\\
    {Masse = }\SI{14.08}{\gram}&48&2215&845&0.0127&0.0104\\
    {Dichte = }\SI{7.4}{\gram\per\centi\metre^2}&49&2240&870&0.0130&0.0107\\
    \bottomrule
  \end{tabular}
  \caption{Tabelle der gemessenen Brückenspannungen und Widerstände zum
    Nullabgleich der Brückenschaltung.  Die aus den jeweiligen Verfahren
    bestimmten Werte für die Suszeptibilität sind ebenfalls angegeben.}
  \label{tab:suszeptwerte}
\end{table}

\begin{table}
  \centering
  \begin{tabular}{SSS}
    \toprule
    {Material}& $\chi_\text{U}$ & $\chi_\text{R}$\\
    \midrule
    {Dy$_2$O$_3$}&\num{0.0234(1)}&\num{0.0186(1)}\\
    {Nd$_2$O$_3$}&\num{0.0054(2)}&\num{0.0039(2)}\\
    {Gd$_2$O$_3$}&\num{0.0130(1)}&\num{0.0108(2)}\\
    \bottomrule
  \end{tabular}
  \caption{Ergebnisse dieses Versuches.  Angegeben sind die Mittelwerte
    der Suszeptibilitäten der angegebenen Proben. Der Fehler ist der
    statistische Fehler.}
  \label{tab:ergebnisse}
\end{table}

\FloatBarrier
