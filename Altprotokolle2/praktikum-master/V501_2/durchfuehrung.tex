% This work is licensed under the Creative Commons
% Attribution-NonCommercial 3.0 Unported License. To view a copy of this
% license, visit http://creativecommons.org/licenses/by-nc/3.0/.

\section{Durchführung}
\label{sec:durchfuehrung}

\subsection{Nachweis der Proportionalität der Leuchtfleckverschiebung}

Zum Nachweis der Proportionalität zwischen Ablenkspannung und
Leuchtfleckverschiebung werden für fünf verschiedene
Beschleunigungsspannungen Meßreihen Ablenkspannung gegen
Leuchtfleckverschiebung aufgenommen.

\subsection{Prinzip eines Kathodenstrahloszillators}

Um die Funktionsweise eines Kathodenstrahloszillators zu verstehen, wird
auf die Ablenkspannug in $x$-Richtung ein Sägezahnsignal gegeben und auf
die Ablenkung in $y$-Richtung ein Sinus-Signal mit fester, aber
unbekannter Frequenz zwischen \SIrange{80}{90}{\hertz}. Beide Signale
werden von einem Funktionengenerator erzeugt.

Dann wird die Frequenz der Sägezahnspannung so angepaßt, daß sich
bestimmte Frequenzverhältnisse gegenüber dem Sinus-Signal
einstellen. Stehen die Frequenzen in dem Verhältnis
%
\begin{equation}
  \label{eq:synchro-constraint}
  n \cdot \nu_\text{Sägezahn} = m \cdot \nu_\text{Sinus} \quad n, m\in\mathbb{N}
\end{equation}
%
ist auf dem Leuchtschirm ein stehendes Bild des Signals zu erkennen. Die
Frequenz der Sägezahnspannung wird nun so angepaßt, daß stehende Bilder
gesehen werden.

\subsection{Bestimmung der spezifischen Ladung des Elektrons}

Vor Beginn der Messung wird die Achse der Kathodenstrahlröhre nach
Norden ausgerichtet. Zum Auffinden dieser Richtung wird ein
Inclinatorium-Declinatorium benutzt. Dann wird einmal bei
\SI{250}{\volt} Beschleunigungsspannung und einmal bei \SI{450}{\volt}
die Leuchtpunktverschiebung in Abhängigkeit des Spulenstromes gemessen.

\subsection{Bestimmung der Stärke des Erdmagnetfeldes}

Die Achse der Röhre wird wieder nach Norden ausgerichtet. Bei einer
Beschleunigungsspannung von \SI{200}{\volt} betrachtet man den
Leuchtschirm und merkt sich die Lage des Leuchtpunktes. Dann wird die
Anordnung um $\pi/2$ gedrecht, so daß die Achse der Röhre in
Ost-West-Richtung zeigt. Nun muß mit dem Feld der
\name{Helmholtz}-Spulen der Leuchtpunkt so abgelenkt werden, daß er in
der vorherigen Lage erscheint. Jetzt gleicht das Feld der Spulen das
horizontale Magnetfeld der Erde genau aus.

Um die Größe der Erdmagnetfeldstärke zu bestimmen, muß nur noch der
Inclinationswinkel $\varphi$ (zwischen Tangentialebene und Richtung des
Erdmagnetfeldes) bekannt sein. Dieser kann wieder mit dem
Inclinatorium-Declinatorium bestimmt werden. Zurnächst wird die
Nord-Süd-Richtung mit dem Declinatorium aufgefunden, dann der Teilkreis
um $\pi/2$ gedreht. Jetzt kann der Winkel abgelesen werden.
