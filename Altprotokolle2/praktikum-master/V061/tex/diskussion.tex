% This work is licensed under the Creative Commons
% Attribution-NonCommercial 3.0 Unported License. To view a copy of this
% license, visit http://creativecommons.org/licenses/by-nc/3.0/.

\section{Diskussion}
Zum Schluss soll eine kleine Diskussion zu den in diesem Versuch
erhaltenen Ergebnissen durchgeführt werden.

\paragraph{Zur Wellenlänge des Laserlichtes}

Die in diesem Versuch ermittelte Wellenlänge des verwendeten 
Lasers wird nach~\eqref{eq:wellenwert} zu  
\SI{629.32(420)}{\nano\metre} bestimmt. Im Rahmen der 
ermittelten statistischen Fehler entspricht dieser Wert dem 
Literaturwert der Wellenlänge des Laserlichtes des HeNe-Lasers 
von \SI{633}{\nano\metre}. Dieser Wert stammt von \cite{nistlaser}.

Der Abstand zwischen Schirm und Laser, sowie die Positionen der 
Interferenzmaxima auf dem Schirm wurden mit einem Maßband 
vermessen. Mit dieser Methode ist keine sehr genaue Angabe von 
Längen möglich, scheint aber ausreichend zu sein, um Wellenlängen, 
wie sie das Licht des HeNe-Lasers besitzt, zu bestimmen.

\paragraph{Zur Polarisation des Laserlichtes}

Es wurden einmal die Winkeleinstellung des Polarisationsfilters 
betrachtet, bei der der gemessene Photostrom maximal wird, 
und einmal der Polarisationswinkel ermittelt, welcher die beste 
Fitfunktion durch die gemessenen Photoströme ergibt. 

Die letztere Methode ist diejenige, welche dem wahren 
Polarisationswinkel am nächsten kommt. 

\paragraph{Zu den TEM-Moden}

Die Vermessung der Grundmode funktioniert gut. 
Der Fit einer Gaussfunktion hat ebenfalls funktioniert. 
Allerdings zeigen sich in Abb.~\ref{fig:mode1} noch 
sichtbare Abweichungen zur reinen Gaussfunktion. 
Dies liegt zum einen daran, dass der Photostrom durch 
Umgebungslicht stark beeinflusst wird.
Zum anderen sind in dem vermessenen Licht auch noch 
Oberwellen enthalten, welche nicht ausgeblendet werden. 

Dies sieht man z.B. daran, dass die Vermessung der ersten 
Oberwelle tatsächlich Messwerte ergibt, die durch eine 
Überlagerung zweier Gaussfunktionen erklärbar ist.

Das Ausblenden der Grundmode erweist sich als schwierig und 
die Tatsache, dass in Abb.~\ref{fig:mode2} die beiden 
Maxima nicht gleich hoch sind, zeigt bereits, dass dies in diesem 
Versuch auch nicht komplett gelungen ist.

\paragraph{Zu den Stabilitätsgebieten}

Theoretisch vorhergesagte und experimentell ermittelte 
Stabilitätsgebiete stimmen in etwa überein. 
Die experimentell erreichten Stabilitätsgebiete sind immer 
kleiner als die theoretisch vorhergesagten.

Dies liegt daran, dass zum einen die verwendeten Spiegel nicht 
perfekt sind und zum anderen auch der Aufbau Abweichungen 
zu den theoretischen Idealisierungen enthält.
