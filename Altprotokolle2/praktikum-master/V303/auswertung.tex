% This work is licensed under the Creative Commons
% Attribution-NonCommercial 3.0 Unported License. To view a copy of this
% license, visit http://creativecommons.org/licenses/by-nc/3.0/.

\section{Auswertung}

Aus den Abbildungen~\ref{fig:anfangssignal_ohne_rausch} und
\ref{fig:referenzsignal_ohne_rausch} liest man folgende Werte für die
Amplitude der Referenzspannung und des Signals ab:
%
\begin{align}
  \widehat{U}_\text{ref} &= \SI{3200}{\milli\volt} &
  \widehat{U}_\text{sig} &= \SI{20}{\milli\volt}
\end{align}
%
Die Frequenz der beiden Signale ist auf den Wert \SI{1.000}{\kilo\hertz}
eingestellt. Für den Wert der Ausgangsspannung $U_\text{out}$, der nach
Formel~\eqref{eq:gleichspannung} berechnet wird, ergibt sich:
\begin{equation}
  \label{eq:u_out-value}
  U_\text{out} = k \cdot \SI{32}{\milli\volt\squared}
\end{equation}

\subsection{Ohne Rauschen}

Gemessen wird die Ausgangsgleichspannung $U_\text{out}$ in Abhängigkeit
der Phase $\phi$. Die erhaltenen Werte finden sich in
Tabelle~\ref{tab:messwerte}. Diese werden nun in eine nichtlineare
Regression der Form
%
\begin{equation}
  \label{eq:cos-regress}
  U_\text{out}(\phi) = A\cos(\phi+\phi_0)
\end{equation}
%
gegeben und mit dem theoretischen Ergebnis nach
Formel~\eqref{eq:gleichspannung} verglichen. Für die nichtlineare
Regression wird die
\texttt{scipy}-Bibliothek\footnote{\url{http://www.scipy.org/}} in der
Version 0.11.0 verwendet. Das Ergebnis ist in
Abbildung~\ref{fig:ohne-rauschen} und hier zu sehen:
%
\begin{equation}
  \label{eq:cos-regress-result}
  U_\text{out}(\phi) = \SI{-85.13(384)}{\milli\volt} \cdot
  \cos(\phi + \num{0.20})
\end{equation}
%
Der Proportionalitätsfaktor kann nun mithilfe von
Formel~\eqref{eq:u_out-value} bestimmt werden:
%
\begin{equation}
  \label{eq:propt-value}
  k = \SI{2.66}{\per\milli\volt}
\end{equation}

\begin{figure}
  \centering
  \includegraphics[width=0.8\textwidth]{mit-rauschen}
  \caption{Graphische Darstellung der Meßwerte zur Gleichspannung mit
    Rauschen und ihrer Regressionskurve}
  \label{fig:mit-rauschen}
\end{figure}

\subsection{Mit Rauschen}

Nun wird das mit Rauschen überlagerte Signal untersucht. Die Meßwerte
können ebenfalls in Tabelle~\ref{tab:messwerte} gefunden werden. Für die
nichtlineare Regression mit Formel~\eqref{eq:cos-regress} wird wieder
die \texttt{scipy}-Bibliothek verwendet. In
Abbildung~\ref{fig:mit-rauschen} kann man die Regression sehen. Die
Gleichung dazu lautet:
%
\begin{equation}
  \label{eq:cos-regress-result-noise}
  U_\text{out}(\phi) = \SI{2443(76)}{\milli\volt} \cdot 
  \cos(\phi + \num{1.29})
\end{equation}
%
Hier ergibt sich der Proportionalitätsfaktor mithilfe von
Formel~\eqref{eq:u_out-value} zu:
%
\begin{equation}
  \label{eq:propt-value}
  k = \SI{76.34}{\per\milli\volt}
\end{equation}

% Tabelle mit den Meßwerten
\begin{table}
  \centering
  \begin{tabular}{SSSS}
    \toprule
    \multicolumn{2}{c}{Mit Rauschen} &
    \multicolumn{2}{c}{Ohne Rauschen} \\
    \midrule
    {\(\phi [\si{\degree}]\)} & 
    {\(U_\text{out} [\si{\milli\volt}]\)} &
    {\(\phi [\si{\degree}]\)} & 
    {\(U_\text{out} [\si{\milli\volt}]\)} \\
    \midrule
    0   &   504  &   0 & -80.0 \\
    15  &   280  &  15 & -78.0 \\
    30  &  -424  &  30 & -72.0 \\
    75  & -2240  &  45 & -52.8 \\
    45  & -1080  &  60 & -18.4 \\
    60  & -1840  &  75 &   5.6 \\
    90  & -2360  &  90 &  17.6 \\
    105 & -2400  & 105 &  26.8 \\
    180 &  -536  & 180 &  84.0 \\
    270 &  2320  & 270 & -17.6 \\
    \bottomrule
  \end{tabular}
  \caption{Meßwerte der Gleichspannung $U_\text{out}$ in Abhängigkeit der
    Phase $\phi$ des Referenzsignals}
  \label{tab:messwerte}
\end{table}


\begin{figure}
  \centering
  \includegraphics[width=0.8\textwidth]{ohne-rauschen}
  \caption{Graphische Darstellung der Meßwerte zur Gleichspannung ohne
    Rauschen und ihrer Regressionskurve}
  \label{fig:ohne-rauschen}
\end{figure}

\subsection{Abstandsverhalten der Lichtintensität}

Für die Untersuchung des Abstandsverhaltens der Lichtintensität setzt
man eine Funktion der Form
%
\begin{equation}
  \label{eq:regressionsfunktion}
  f(x) = Ax^{-2} + B
\end{equation}
%
an. Da wir aber während der Messung der Intensität den \emph{Gain} des
Verstärkers erhöhen mußten, um überhaupt noch etwas zu messen, müssen
die ermittelten Spannungen zunächst noch durch den \emph{Gain} geteilt
werden. Die entsprechenden Werte findet man in
Tabelle~\ref{tab:licht-messung}. 

Die Regression wird wieder mit der \texttt{scipy}-Bibliothek
durchgeführt. Die Ergebnisse für die Parameter $A$ und $B$ sind hier
dargestellt:
%
\begin{align}
  A &= \SI{0.20(01)}{\milli\volt\metre} & 
  B &= \SI{-1.7(6)}{\milli\volt}
\end{align}
%
Es ergibt sich also die folgende Gleichung für $U$:
%
\begin{equation}
  U = \SI{0.20}{\milli\volt\metre} \cdot r^{-2} -
  \SI{1.7}{\milli\volt}
\end{equation}
%
Einen Plot der Meßwerte mit der ermittelten Regressionskurve findet man
in Abbildung~\ref{fig:licht-messung}.

\begin{figure}
  \centering
  \includegraphics[width=0.8\textwidth]{licht-messung}
  \caption{Graphische Darstellung der Meßwerte zum Abstandsverhalten der
    Lichtintensität mit ihrer Regressionskurve}
  \label{fig:licht-messung}
\end{figure}

\begin{table}
  \centering
  \begin{tabular}{SSS}
    \toprule
    {Abstand [\si{\centi\metre}]} &
    {Spannung [\si{\milli\volt}]} &
    {Gain} \\
    \midrule
    6 & 62.4 &   1 \\
    8 & 23.2 &   1 \\
    10 & 14.0 &   1 \\
    12 &  8.4 &   1 \\
    14 &  6.4 &   1 \\
    16 &  5.0 &   1 \\
    18 &  5.8 &   2 \\
    20 &  4.8 &   2 \\
    22 &  8.2 &   5 \\
    24 &  7.8 &   5 \\
    27 &  5.4 &   5 \\
    30 &  8.6 &  10 \\
    33 &  7.4 &  10 \\
    36 &  5.8 &  10 \\
    39 & 10.2 &  20 \\
    42 &  9.2 &  20 \\
    45 &  8.4 &  20 \\
    48 &  7.0 &  20 \\
    51 &  5.8 &  20 \\
    54 & 22.8 & 100 \\
    64 & 16.8 & 100 \\
    74 & 11.8 & 100 \\
    84 & 20.4 & 200 \\
    94 & 17.6 & 200 \\
    104 & 35.2 & 500 \\
    \bottomrule
  \end{tabular}
  \caption{Meßwerte zur Bestimmung des Abstandsverhalten 
    der Lichtintensität}
  \label{tab:licht-messung}
\end{table}
