% This work is licensed under the Creative Commons
% Attribution-NonCommercial 3.0 Unported License.  To view a copy of
% this license, visit http://creativecommons.org/licenses/by-nc/3.0/.

\section{Theorie}
\subsection{Wärmekapazität}
%
Als Wärmekapazität eines Festkörpers der Masse $m$ wird der 
Proportionalitätsfaktor zwischen der Änderung der Wärmeenergie 
$\Delta Q$ und der Temperaturänderung $\Delta T$ des Körpers 
bezeichnet. Dieser Faktor hängt linear von der Masse des Körpers ab. 
Deswegen lässt sich dieser Zusammenhang mit Formel~\eqref{eq:q} 
ausdrücken. Dabei ist $c$ die massenunabhängige, materialspezifische 
Wärmekapazität.
%
\begin{equation}
\label{eq:q}
\Delta Q = mc\Delta T
\end{equation}
%
Von Interesse ist die sogenannte Molwärme $C$. Diese bezeichnet 
die Energie, die umgesetzt wird, wenn ein Mol eines Stoffes seine 
Temperatur ändert; also die Wärmekapazität für 1 Mol eines Stoffes.
Der Wert der Molwärme hängt davon ab, ob die Wärmenergie einem Körper 
bei konstantem Druck $p$ oder bei konstantem Volumen $V$ zugeführt wird.
Es gilt
%
\begin{equation}
\label{eq:cv}
C_V = \left(\frac{dQ}{dT}\right)_V.
\end{equation}
%
Da nach dem ersten Hauptsatz die Änderung der inneren Energie $dU$ 
gleich der Änderung der Wärme- und mechanischen Energie ist, bei gleich 
bleibendem Volumen allerdings keine mechanische Energie verrichtet wird, 
lässt sich Gleichung~\eqref{eq:cv} schreiben als
%
\begin{equation}
C_V = \left(\frac{dU}{dT}\right)_V
\end{equation}
%
Der Zusammenhang zwischen Molwärme $C_V$ bei konstantem Volumen und 
$C_p$ bei konstantem Druck ist durch Formel~\eqref{eq:cv-und-cp}
gegeben.
%
\begin{equation}
\label{eq:cv-und-cp}
C_p - C_V = 9\alpha^2\kappa V_0T
\end{equation}
%
Hierbei bezeichnet $\alpha$ den linearen Ausdehnungskoeffizieten, 
$\kappa = V\left(\frac{\partial p}{\partial V}\right)_T$ den Kompressionsmodul 
und $V_0$ das Molvolumen; also das Volumen, welches von einem Mol des 
Stoffes bei Raumtemperatur eingenommen wird. 
%
\subsection{Klassische Betrachtung: Das Dulong-Petitsche Gesetz}
In der klassischen Vorstellung besteht ein Festkörper aus Massenpunkten, 
die aneinander durch Gitterkräfte gebunden sind. Besitzt der Festkörper 
eine endliche Temperatur, so besitzt dieser eine innere Energie, welche 
sich auf die Massenpunkte verteilt. Diese werden dadurch aus ihren 
Gleichgewichtslagen ausgelenkt. Für kleine Auslenkungen gilt das 
Hooksche Gesetz, sodass ein Massenpunkt harmonisch oszilliert. 
Bekannterweise ist die mittlere kinetische Energie $\left<E_\text{kin}\right>$ 
eines harmonischen Oszillators gleich seiner mittleren potentiellen Energie
$\left<E_\text{pot}\right>$, sodass sich für die gesamte mittlere 
Energie eines Atoms im Festkörper der in Formel~\eqref{eq:mittlere-energie} 
wiedergegebene Zusammenhang ergibt.
%
\begin{equation}
\label{eq:mittlere-energie}
\left<u\right> = \left<E_\text{kin}\right> + \left<E_\text{pot}\right> = 2\left<E_\text{kin}\right>
\end{equation}
%
Das Äquipartitionstheorem der kinetische Wärmetheorie ordnet einem Atom 
im thermischen Gleichgewicht mit seiner Umgebung pro Freiheitsgrad eine 
mittlere kinetische Energie von $\frac{1}{2}k_\text{B}T$ zu. 
Da es drei Raumdimensionen gibt und ein Mol eines Stoffes 
$N_\text{L} = 6.02\cdot10^{23} /\text{Mol}$ Atome besitzt, kann 
Formel~\eqref{eq:mittlere-energie} mit der gesamten mittleren inneren Energie 
$\left<U\right>$ umgeschrieben werden zu
%
\begin{equation}
\left<U\right> = 3 N_\text{L} k_\text{B} T = 3RT .
\end{equation}
%
$R$ bezeichnet die universelle Gaskonstante.
Eingesetzt in Formel~\eqref{eq:cv} ergibt dies das Dulong-Petitsche 
Gesetz, welches in Gleichung~\eqref{eq:dulong-petit} wiedergegeben ist.
%
\begin{equation}
\label{eq:dulong-petit}
C_V = 3R
\end{equation}
%
\subsection{Quantenmechanische Betrachtung}
In der Quantenmechanik sind die Energien des harmonischen Oszillators 
gequantelt, d.h. die Energie des Oszillators kann nur vielfache des 
reduzierten Planck'schen Wirkungsquantums $\hbar$ mal der Kreisfrequenz 
$\omega$
betragen. Ebenso können nur diese Energiequanten auf- und abgegeben werden. 
Diese Aussage ist in Formel~\eqref{eq:quant} wiedergegeben.
%
\begin{equation}
\label{eq:quant}
\Delta u = n \hbar \omega
\end{equation}
%
Die Wahrscheinlichkeit, bei gegebener Temperatur $T$ die mittlere Energie 
$\left<u_\text{qu}\right>$ in einem eindimensionalen Oszilaltor des 
Festkörpers vorzufinden ist durch Formel~\eqref{eq:wahrscheinlichkeit} 
gegeben.
%
\begin{equation}
\label{eq:wahrscheinlichkeit}
\left<u_\text{qu}\right> = \frac{\hbar\omega}{\exp{(\hbar\omega/K_\text{B}T)}-1}
\end{equation}
%
Wird nun angenommen, dass die Kreisfrequenzen eines oszillierdenden 
Atoms in allen drei 
Raumrichtungen dieselben sind, so kann die mittlere innere Energie eines 
Festkörpers mit ein Mol Atomen $\left<U_\text{qu}\right>$ geschrieben 
werden als
%
\begin{equation}
\label{eq:quantenpetit}
\left<U_\text{qu}\right> = \frac{3N_\text{L}\hbar\omega}{\exp{(\hbar\omega/K_\text{B}T)}-1}
\end{equation}
%
Also hängt die mittlere innere Energie nicht mehr proportional von der 
Temperatur ab, wie es in der klassischen Betrachtung festgestellt wurde.
Nach dem Korrespondenzprinzip muss die klassische Physik aber als 
Grenzfall der Quantenphysik in ihr enthalten sein. Tatsächlich wird der 
Ausdruck~\eqref{eq:quantenpetit} im Grenzfall für $T \rightarrow \infty$ 
wieder zum klassischen Dulong-Petit Gesetz~\eqref{eq:dulong-petit}.