
\section{Durchführung}

\subsection{Vorbereitung der Apparatur}

Zunächst werden die Dichten der Glaskugeln, die aus dem Kästchen Nr.~2
stammen, bestimmt. Der Durchmesser der großen Kugel wird mit einer
Schieblehre gemessen. Die kleine Kugel wird in zwei rechte
Kunststoffwinkel eingezwängt und der Abstand der Winkel mit der
Schieblehre bestimmt. Dazu gibt es eine Meßreihe mit zehn Werten. Das
Gewicht wird mit einer elektronischen Waage bestimmt.

Dann wird die Aparatur vorbereitet. Erst wird mit der Libelle, die sich
am Viskosimeter befindet, kontrolliert ob dieses gerade steht. Danach
wird das destillierte Wasser eingefüllt, während darauf geachtet wird,
daß sich keine Luftblasen an der Rohrwand sammeln. Auch beim Einlegen
der Kugeln müssen Luftblasen vermieden werden. Nachdem die Aparatur mit
einer Schraube verschlossen worden ist, kann die Messung durchgeführt
werden.

\subsection{Messung}

Für die große und die kleine Kugel werden jeweils zehn Messungen bei
Raumtemperatur durchgeführt. Gemessen wird die Zeit, die die Kugel
benötigt um von der oberen Marke zur unteren zu gelangen. Das
Viskosimeter wird umgedreht, wenn die Kugel die untere Marke
überschritten hat, und die Messung wiederholt. Aus den gewonnenen
Meßwerten kann die Apparaturkonstante bestimmt werden.

Jetzt wird das Wasserbad langsam auf \SI{70}{\degreeCelsius} aufgeheizt
und für zehn verschiedene Temperaturen jeweils zweimal die Zeit, die die
große Kugel von Marke zu Marke benötigt, gemessen.
