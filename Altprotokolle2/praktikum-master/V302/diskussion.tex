% This work is licensed under the Creative Commons
% Attribution-NonCommercial 3.0 Unported License. To view a copy of this
% license, visit http://creativecommons.org/licenses/by-nc/3.0/.

\section{Diskussion}
Zum Schluss werden die Ergebnisse dieses Versuches beurteilt.  Die
Nullmessungsmethode aller Brückenschaltungen hat sich als äußerst
präzise erwiesen, was sich an den kleinen Standardabweichungen der
Mittelwerte erkennen lässt.

Der verwendete Funktionsgenerator besitzt einen geringen Kliffaktor von
$k = \SI{0.47}{\percent}$, weswegen die Messwerte bei der
\name{Wien}-\name{Robinson}-Brücke beinahe mit der Theoriekurve
übereinstimmen. Im Plot von Abb.~\ref{fig:wien_robinson_plot} ist es
zwar nicht gut zu erkennen, aber die Messwerte zeigen, dass die
Brückenspannung niemals komplett auf \SI{0}{\volt} fällt. Dies liegt
daran, dass der Funktionsgenerator schwache Oberschwingungen miterzeugt,
was durch den Klirrfaktor, der größer als Null ist, zum Ausdruck kommt.
