% This work is licensed under the Creative Commons
% Attribution-NonCommercial 3.0 Unported License. To view a copy of this
% license, visit http://creativecommons.org/licenses/by-nc/3.0/.

\section{Diskussion}
Bei der Berechnung der Zeitkonstanten $\tau$ treten drei verschiedene Werte auf. Als Mittelwert aller Zeitkonstanten ergibt sich ein Wert von $\tau$ = \SI{1.395}{\milli\second} $\pm$ \SI{0.045}{\milli\second}. Obwohl die Fehler der Zeitkonstanten bei den verschiedenen Bestimmungsmethoden äußerst gering ist, tritt zwischen den drei gefundenen Werten ein bemerkenswerter Unterschied ein. Es besteht die Möglichkeit, den wahren Wert der Zeitkonstante besser zu bestimmen, wenn der Wert des Widerstandes R des RC-Kreises bekannt ist.

Bei der Betrachtung der Amplitudenabnahme bei steigender Frequenz ist festzustellen, dass die Amplitude zwar abfällt, sie dies aber mit unzureichender Geschwindigkeit tut. Eine Hintereinanderschaltung mehrerer RC-Kreise führt somit zu einem guten Tiefpass-Filter. 

Bei hohen Frequenzen agiert der RC-Kreis ebenfalls als Integrator, wie es den Abbildungen im letzten Auswertungsteil zu entnehmen ist.