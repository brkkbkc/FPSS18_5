% This work is licensed under the Creative Commons 
% Attribution-NonCommercial 3.0 Unported License. To view a copy of this 
% license, visit http://creativecommons.org/licenses/by-nc/3.0/.

\section{Durchführung}

Die Durchführung des Versuchs besteht aus vier Teilen. Zunächst wird die
Linsengleichung~\eqref{eq:linsengleichung} an einer Sammellinse
verifiziert, danach die Brennweite einer unbekannten Sammellinse
bestimmt, als drittes die Brennweite für rotes, blaues und weißes Licht
nach der Methode von \name{Bessel}\footnote{\name{Bessel}, Friedrich
Wilhem (*1784, $\dagger$1846), deutscher Astronom, Mathematiker und
Geodät. Zitiert nach \textcite{wikipedia:friedrich-bessel}} bestimmt und
schließlich wird die Brennweite eines Linsensystems nach der Methode von
Herrn \name{Abbe}\footnote{\name{Abbe}, Ernst Karl, (*1480,
$\dagger$1905), deutscher Physiker, Statistiker, Optiker, Unternehmer
und Quelle. Sozialreformer. Zitiert nach
\textcite{wikipedia:ernst-abbe}} bestimmt.

\subsection{Verifikation der Linsengleichung}

Zur Verikation der Linsengleichung~\eqref{eq:linsengleichung} wird eine
Sammellinse auf die Bank zwischen Perl-L und Schirm gestellt. Jetzt wird
eine Gegenstandsweite~$g$ eingestellt, indem die Linse verschoben wird,
und dazu der Schirm so angepaßt, daß das Bild scharf ist, d.\,h. die
Bildweite~$b$ bestimmt. Es werden dabei zehn Messungen durchgeführt.

\subsection{Bestimmung der Brennweite einer unbekannten Linse}

Um die Brennweite einer unbekannten Sammellinse zu bestimmen, wird so
vorgegangen wie im Falle der Verifiktation der Linsengleichung. Es wird
eine Sammellinse, deren Brennweite nicht bekannt ist (in unserem Fall
ist dies durch eine Kunststoff—Linse gegeben, die mit Wasser gefüllt ist
und deren Wölbung sich durch die Menge an Wasser innerhalb varieren
läßt), auf die Bank gestellt und für verschiedene Gegenstandsweiten, die
Bildweiten bestimmt.

\subsection{Die Methode nach \name{Bessel}}

Bei dieser Methode wird der Abstand zwischen Gegenstand und Schirm
konstant gehalten und durch Verschieben der Linse zwei Positionen
gesucht, so daß das Bild scharf auf dem Schirm abgebildet ist. Es ergibt
sich dadurch eine symmetrische Linsenstellung, bei der Gegenstands- und
Bildweite jeweils vertauschen. Wenn dabei die Bildweite~$b$ größer als
die Gegenstandsweite~$g$ ist, wird das Bild vergrößert. Umgekehrt wird
es verkleinert. Die jeweiligen Gegenstands- und Bildweiten werden
notiert.

Wird der Abstand von Schirm und Gegenstand mit $e$ bezeichnet, d.\,h.
%
\begin{equation}
  e = g + b
\end{equation} 
%
und der Abstand der beiden Linsenpositionen, an denen der Gegenstand
scharf auf den Schirm abgebildet wird, mit $d$ bezeichnet, also
%
\begin{equation}
  d = |g - b|,
\end{equation}
%
dann ergibt sich aus der Linsengleichung~\eqref{eq:linsengleichung} die
Formel
%
\begin{equation} f = \frac{e^2 - d^2}{4e}
  \label{eq:f-bessel}
\end{equation}
%
für die Brennweite der Linse. Insgesamt werden für zehn verschiedene
Abstände $e$ jeweils die Gegenstands- und Bildweiten für die
symmetrischen Linsenpositionen bestimmt.

\subsection{Bestimmung der Brennweite eines Linsensystems nach
\name{Abbe}}

Die Methode von Abbe erlaubt es, die Brennweite und die Lage der
Hauptebenen eines Linsensystems durch Messung des Abbildungsmaßstabs~$V$
und einer Art Gegenstands- und Bildweite $g'$ bzw. $b'$ zu bestimmen. Es
wird in diesem Versuch ein System aus Sammel- und Zerstreuungslinse
verwendet, die nahe hintereinander stehen. Weil die Gegenstands- und
Bildweite bezüglich der Hauptebenen gemessen wird, deren Lage aber nicht
bekannt ist, mißt man statt dessen den Abstand des Gegenstands~$g'$
bzw. Bildes~$b'$ gegen einen festen Referenzpunkt. Die Abstände der
Hauptebenen von diesem Referenzpunkt werden mit $h$ und $h'$
bezeichnet. Es ist hier die äußerst rechte Kante des Reiters, auf dem
die Linsen befestigt sind, als ein solcher Referenzpunkt gewählt
worden. Mit Formel~\eqref{eq:abbildungsmassstab}
und~\eqref{eq:linsengleichung} stellt man folgende Beziehung zwischen
Brennweite und Abbildungsmaßstab und den gemessenen Längen her:
%
\begin{align}
  \label{eq:abbe-gegenstand}
  g' &= g + h  = f \left(1 + \frac{1}{V}\right) + h\\
  \label{eq:abbe-bild}
  b' &= b + h' = f (1 + V) + h'
\end{align}

Schirm und Linsensystem werden nun zehnmal so verschoben, daß das Bild
scharf auf dem Schirm erscheint, wobei jedes Mal Abbildungsmaßstab und
die Abstände~$g', b'$ gemessen werden.
