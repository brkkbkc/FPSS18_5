% This work is licensed under the Creative Commons
% Attribution-NonCommercial 3.0 Unported License. To view a copy of this
% license, visit http://creativecommons.org/licenses/by-nc/3.0/.

\section{Theorie}

\subsection{Entstehung der $\alpha$-Strahlen}

In diesem Versuch wird ein Americium-Präparat als Strahlungsquelle
verwendet. Americium ist ein radioaktives Metall. Bei Am beträgt die
Halbwertszeit $T_{1/2}$ = 458 a. Der Zerfall von Americium geschieht
nach dem in Formel \eqref{eq:zerfall} wiedergegebenen Vorgang.
\begin{equation}
  \label{eq:zerfall}
  {}^{241}_{95}\mathrm{Am} \longrightarrow {}^{237}_{93}\mathrm{Np} +
  {}^{4}_{2}\mathrm{He}^{++}
\end{equation}
Ein Americium-Atom zerfällt also in ein Neptunium-Atom und in ein
zweifach positiv geladenes Helium-Atom, was einem Heliumkern
entspricht. Diese emittierten Heliumkerne sind die in diesem Versuch
verwendeten $\alpha$-Teilchen.

\subsection{Reichweite der Strahlung bei hohen Energien}

Durchläuft ein $\alpha$-Teilchen ein Medium, so kann es zu elastischen
Stößen mit den dort befindlichen Teilchen kommen, wobei das
$\alpha$-Teilchen Energie verliert. Diese Energieabnahme ist allerdings
vernachlässigbar klein. Hauptursache für das Abnehmen der Energie der
$\alpha$-Teilchen ist die Anregung oder Dissoziation von Molekülen. bei
hoher Energie der $\alpha$-Strahlung kann man den Energieverlust
mithilfe der Bethe-Bloch-Gleichung \eqref{eq:bethe-bloch} beschreiben.
\begin{equation}
  \label{eq:bethe-bloch}
  -\frac{d E_\alpha}{dx} = \frac{z^2 e^4}{4 \pi \epsilon_0 m_e} \cdot 
  \frac{n Z}{v^2} \cdot \ln \left(\frac{2 m_e v^2}{I}\right)
\end{equation}
\begin{align*}
  z\colon &\text{Ladung der Strahlung}\\
  v\colon &\text{Geschwindigkeit der Strahlung}\\
  Z\colon &\text{Ordnungszahl}\\ 
  n\colon &\text{Teilchendichte des Mediums}\\
  I\colon &\text{Ionisierungsenergie des Mediums}
\end{align*}
Die Reichweite der Teilchen erhält man hierbei durch Integration von Null bis zur Energie der Teilchen über den Kehrwert des Energieverlusts.

\subsection{Reichweite der Strahlung bei geringen Energien}

Die in diesem Versuch untersuchte $\alpha$-Strahlung besitzt eine zu
geringe Energie, um den Energieverlust mithilfe der
Bethe-Bloch-Gleichung zu bestimmen. Im niedrigen Energiebereich kommt es
nämlich vermehrt zu Ladungsaustauschprozessen, weswegen in diesem
Protokoll die mittlere Reichweite  $R_m$ in Luft mithilfe der empirisch
gewonnenen Formel \eqref{eq:reichweite} bestimmt wird, welche man bei
$\alpha$-Strahlung mit Energien $E_{\alpha}$ $\le$
\SI{2.5}{\mega\electronvolt} verwenden kann (Die Reichweite ist hier in
\si{\milli\metre} angegeben, die Energie in \si{\mega\electronvolt}).
\begin{equation}
\label{eq:reichweite}
R_m = 3.1 \cdot {E_\alpha} ^{\frac{3}{2}}
\end{equation}

\subsection{Funktionsweise eines Halbleiter-Sperrschichtzählers}

In diesem Versuch wird ein Halbleiter-Sperrschichtzähler verwendet, um
die Anzahl der $\alpha$-Teilchen und deren Energie zu messen. Der
Halbleiter-Sperrschichtzähler wird in Sperrichtung betrieben. Trifft nun
ein $\alpha$-Teilchen auf den Detektor, so entstehen Elektronen-Loch
Paare, da die $\alpha$-Teilchen zweifach positiv geladen sind. Dadurch
kommt es also kurzzeitig zu einem Stromfluss, welcher als Impuls
registriert werden kann. Die Stärke des Impulses ist dabei proportional
zur Energie des $\alpha$-Teilchen, welches auf den Sperrschichtzähler
trifft.
