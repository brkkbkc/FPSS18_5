% This work is licensed under the Creative Commons
% Attribution-NonCommercial 3.0 Unported License. To view a copy of this
% license, visit http://creativecommons.org/licenses/by-nc/3.0/.

\section{Auswertung}

Aus den gewonnenen Daten, die nachfolgend vorgestellt werden, kann man
den Elastizitätsmodul der untersuchten Metalle und Legierungen
errechnen. Um eine lineare Regression\footnote{benutzt wurde hierfür die
  \texttt{ipython}-Umgebung mit der \texttt{numpy}-Bibliothek in der
  Version 1.6.2} anwenden zu können, stellt man die Zusammenhänge
\eqref{eq:durchbiegung-einseitig} und \eqref{eq:durchbiegung-beidseitig}
so dar:
%
\begin{align}
  \label{eq:lineare-durchbiegung-einseitig}
  D(x) &= f(Lx^2-\frac{x^3}{3}) & f(u) &:= \frac{F}{2EI}u
\end{align}
Dies ist für die einseitige Aufhängung des Stabes. Für die beidseitige
Aufhängung ergibt sich:
\begin{align}
  \label{eq:lineare-durchbiegung-beidseitig}
  D(x) &= \begin{cases}
    g(3L^2x-4x^3) & 0\le x\le \frac{L}{2}\\
    g(4x^3-12Lx^2 + 9L^2x - L^3) & \frac{L}{2}\le x\le L
  \end{cases}
  & g(u) &:= \frac{F}{48EI}u
\end{align}
%
Hier können nun die Größen $A_g = F/(48EI)$ und $A_f = F/(2EI)$ bestimmt
werden. Umstellen von der beiden Größen nach $E$ liefert:
\begin{align}
  \label{eq:e-modul}
  E &= \frac{F}{2A_fI} & E =& \frac{F}{48A_gI}
\end{align}
Da diese fehlerbehaftet sind, wird eine \name{Gauß}-Fehlerfortpflanzung
durchgeführt:
%
\begin{equation}
  \label{eq:gaussfehler-f}
  \Delta E = \sqrt{\left(\frac{\partial E}{\partial A_f}
      \cdot\Delta A_f\right)^2}
  = \frac{F\cdot\Delta A_f}{2A_f^2I}
\end{equation}
%
\begin{equation}
  \label{eq:gaussfehler-g}
  \Delta E = \sqrt{\left(\frac{\partial E}{\partial A_g}\cdot\Delta
      A_g\right)^2}
  = \frac{F\cdot\Delta A_g}{48A^2_gI}
\end{equation}

Zur Bestimmung des Fehlers, der beim Messen des Durchmessers gemacht
wird, wird die Formel für die Standardabweichung benutzt:

\begin{equation}
  \label{eq:durchmesser-fehler}
  \sqrt{\frac{1}{n-1} \sum_{i=1}^n (d_i - \bar{d})^2}
\end{equation}

Die Flächenträgheitsmomente werden wie folgt berechnet. Für den
Stahlstab mit kreisförmigem Querschnitt $Q = \{(r\cos\phi, r\sin\phi) :
0\le r\le \frac{d}{2}, 0\le\phi\le2\pi\}$ gilt:
\[
I = \int_Q y^2\:\d(x, y)
\]
Durch Anwendung der Transformation $T\colon M\to Q, (r, \phi) \mapsto
(r\cos\phi, r\sin\phi)$ mit $M = \{(r,\phi) : 0\le r\le d/2,
0\le\phi\le2\pi\}$ ergibt sich:
\[
I = \int_Q y^2\:\d(x, y) = \int_M r^3\sin^2\phi\:\d(r, \phi) =
\int_{r=0}^\frac{d}{2} r^3 \int_{\phi=0}^{2\pi} \sin^2\phi\:\d\phi\,\d r
= \frac{d^4\pi}{64}
\]

Das Flächenträgheitsmoment für die beiden Stäbe mit quadratischem
Querschnitt $Q = \{(x, y) : 0\le x\le d, 0\le y\le d\}$ (Aluminium und
Messing) ergibt sich so:
\[
\int_Q y^2\:\d(x, y) = \int_{y=0}^d\int_{x=0}^d y^2\:\d x\,\d y =
d\int_0^d y^2\:\d y = \frac{d^4}{3}
\]

\subsection{Bestimmung des Elastizitätsmoduls von Stahl}

Die gemessenen Durchbiegung in Abhängigkeit vom Ort können in der
Tabelle~\ref{tab:stahl} nachgesehen werden. Es wurden immer die
Durchbiegungen $D_v(x)$ ohne Gewicht und die Durchbiegungen $D_n(x)$ mit
Gewicht bestimmt. Die eingespannte Länge ist $L=\SI{49.5}{\centi\metre}$
und als Gewicht haben wir die Masse $m=\SI{1680.9}{\gram}$
verwendet. Gemäß Formel~\eqref{eq:echte_durchbiegung} errechnet sich die
effektive Durchbiegung~$D(x)$, die Grundlage für weitere Berechnungen
ist.

Die Stichprobe zur Bestimmung des Durchmessers $d$ findet man in
Tabelle~\ref{tab:stahl-durchmesser}. Nach der Bestimmung des
arithmetischen Mittels und der Standardabweichung nach
Formel~\eqref{eq:durchmesser-fehler} ergibt sich für den Durchmesser:
%
\begin{equation}
  d = \SI{9.985(3)}{\milli\meter}
\end{equation}
%
Abbildung~\ref{fig:stahl} zeigt eine Darstellung der Meßwerte und der
Regressionsgerade. Die roten Punkte wurden für die Regression
herangezogen. Daraus erhält man diese Gleichung:
%
\begin{equation}
  D(x) = \SI{0.016}{\per\cubic\metre} 
  \left(Lx^2 - \frac{x^3}{3}\right)  - \SI{0.001}{\metre}
\end{equation}
%
Der Fehler für die Steigung ergibt sich aus der Regression zu:
%
\begin{equation}
  \Delta A_f = \SI{0.0008}{\per\cubic\metre}
\end{equation}
%
Mithilfe der Steigung $A_f$ und Formel~\eqref{eq:e-modul} und nach der
\name{Gauss}-Fehlerfortpflanzung in Formel~\eqref{eq:gaussfehler-f} erhält man
folgenden Wert für $E$:
%
\begin{equation}
  \label{eq:wert-stahl-emodul}
  E = \SI{1078(53)}{\kilo\newton\per\milli\metre\squared}
\end{equation}
%

\begin{table}
  \centering\small
  \begin{tabular}{SSS}
    \toprule
    {$x/\si{\centi\metre}$} &
    {$D_v/\si{\micro\metre}$} &
    {$D_n/\si{\micro\metre}$} \\
    \midrule
    2.7 &    550 &    510 \\
    5.0 &    600 &    470 \\
    9.0 &    600 &    220 \\
    13.0 &    590 &    185 \\
    17.0 &    540 &    136 \\
    21.0 &    530 &    280 \\
    25.0 &    460 &    312 \\
    27.0 &    430 &    377 \\
    32.0 &    310 &    475 \\
    35.0 &    200 &    490 \\
    38.0 &    170 &    544 \\
    40.0 &    198 &    600 \\
    42.0 &    198 &    647 \\
    44.0 &    195 &    700 \\
    46.0 &    180 &    749 \\
    48.0 &    177 &    795 \\
    \bottomrule
  \end{tabular}
  \caption{Meßwerte zur Durchbiegung des Stahlstabes}
  \label{tab:stahl}
\end{table}

\begin{table}
  \centering\small
  \begin{tabular}{S}
    \toprule
    {$d/\si{\milli\metre}$}\\
    \midrule
    9.986 \\
    9.989 \\
    9.983 \\
    9.988 \\
    9.986 \\
    9.982 \\
    9.982 \\
    9.985 \\
    9.985 \\
    9.981 \\
    \bottomrule
  \end{tabular}
  \caption{Durchmesser des Stahlstabes}
  \label{tab:stahl-durchmesser}
\end{table}

\begin{figure}
  \centering
  \includegraphics[width=0.8\textwidth]{stahl-einseitig-kreisfoermig.pdf}
  \caption{Meßwerte mit linearer Regression für den Stahlstab}
  \label{fig:stahl}
\end{figure}

\subsection{Bestimmung des Elastizitätsmoduls von Aluminium}

Die gemessenen Durchbiegungen für den Aluminiumstab ist in
Tabelle~\ref{tab:aluminium} zu finden. Die Werte zum Durchmesser in
Tabelle~\ref{tab:aluminium-durchmesser}. Die eingespannte Länge beträgt
$L=\SI{49.6}{\centi\metre}$ und als Gewicht haben wir die Masse
$m=\SI{5416}{\gram}$ verwendet. Der Mittelwert des Durchmessers und
seine Standardabweichung nach Formel~\eqref{eq:durchmesser-fehler} betragen:
%
\begin{equation}
  d = \SI{9.985(3)}{\milli\metre}
\end{equation}
%
Nach der Regression, die in Abbildung~\ref{fig:aluminium} zu sehen ist,
erhält man folgende Gleichung:
%
\begin{equation}
  D(x) = \SI{0.044}{\per\cubic\metre} \left(Lx^2 - \frac{x^3}{3}\right)
\end{equation}
%
Der Fehler für die Steigung ergibt sich aus der Regression zu:
%
\begin{equation}
  \Delta A_f = \SI{0.0004}{\per\cubic\metre}
\end{equation}
%
Wie bereits beim Stahl bestimmt man hieraus mit der Steigung~$A_f$ und
Formel~\eqref{eq:e-modul} den E-Modul und mithilfe der Fehlerrechnung
aus Formel~\eqref{eq:gaussfehler-f} läßt sich folgender Wert angeben:
%
\begin{equation}
  E = \SI{73.0(7)}{\kilo\newton\per\milli\metre\squared}  
\end{equation}

\begin{table}
  \centering\small
  \begin{tabular}{SSS}
    \toprule
    {$x/\si{\centi\metre}$} & 
    {$D_v/\si{\micro\metre}$} & 
    {$D_n/\si{\micro\metre} $} \\
    \midrule
    2.7  &   920   &    930 \\
    7.5  &  1030   &  1170  \\
    12.5 &  1210   &  1550  \\
    17.5 &  1430   &  2060  \\
    22.5 &  1680   &  2680  \\
    27.5 &  2020   &  3410  \\
    30.0 &  2220   &  3800  \\
    32.0 &  2350   &  4150  \\
    34.0 &  2520   &  4480  \\
    36.0 &  2710   &  4850  \\
    38.0 &  2850   &  5230  \\
    40.0 &  3030   &  5610  \\
    42.0 &  3190   &  6000  \\
    44.0 &  3320   &  6410  \\
    46.0 &  3660   &  6770  \\
    48.0 &  3800   &  7240  \\
    \bottomrule
  \end{tabular}
  \caption{Meßwerte zur Durchbiegung des Aluminiumstabes}
  \label{tab:aluminium}
\end{table}

\begin{table}
  \centering\small
  \begin{tabular}{S}
    \toprule
    {$d/\si{\milli\metre}$}\\
    \midrule
    9.984 \\
    9.988 \\
    9.984 \\
    9.971 \\
    9.979 \\
    9.980 \\
    9.994 \\
    9.986 \\
    9.988 \\
    9.982 \\
    \bottomrule
  \end{tabular}
  \caption{Durchmesserx des Aluminiumstabes}
  \label{tab:aluminium-durchmesser}
\end{table}

\begin{figure}
  \centering
  \includegraphics[width=0.8\textwidth]{aluminium-einseitig-quadratisch.pdf}
  \caption{Meßwerte mit linearer Regression für den Aluminiumstab}
  \label{fig:aluminium}
\end{figure}


\subsection{Bestimmung des Elastizitätsmoduls von Messing}

Der Messingstab ist beidseitig auf die Länge $L =
\SI{55.1}{\centi\metre}$ eingespannt. Als Gewicht verwenden wir die
Masse $m = \SI{4.1}{\kilogram}$. Die Bestimmung des Durchmessers liefert
die Tabelle~\ref{tab:messing-durchmesser}. Mittelwert und
Standardabweichung nach Formel~\eqref{eq:durchmesser-fehler} ergeben sich so:
%
\begin{equation}
  d = \SI{9.983(9)}{\milli\meter}
\end{equation}
%
In Abbildung~\ref{fig:messing} sind die Regressionen
dargestellt. Die beiden Gleichungen lauten folgendermaßen:
%
\begin{align}
  D(x) &= \SI{0.0070}{\per\cubic\metre} 
  ( 3L^2x-4x^3 ) - \SI{0.0002}{\cubic\metre} 
  & 0 \le x \le \frac{L}{2}\\
  D(x) &= \SI{0.0063}{\per\cubic\metre}
  (4x^3 - 12Lx^2 + 9L^2x - L^3) + \SI{0.0001}{\cubic\metre} 
  & \frac{L}{2} \le x \le L
\end{align}
%
Die Fehler für die Steigungen ergeben sich aus der Regression zu:
%
\begin{align}
  \Delta A_f &= \SI{0.0004}{\per\cubic\metre}\\
  \Delta A_g &= \SI{0.0003}{\per\cubic\metre}
\end{align}
%
In Tabelle~\ref{tab:messing} finden sich die Meßwerte dazu. Als
Ergebnis erhält man aus Formel~\eqref{eq:e-modul}, in die man die beiden
Steigungen $A_f$ und $A_g$ einsetzt, zwei Werte für $E$:
\begin{align}
 E&=\SI{143.8(88)}{\kilo\newton\per\milli\metre\squared} &
 E&=\SI{159.4(54)}{\kilo\newton\per\milli\metre\squared}
\end{align}

\begin{figure}
  \centering
  \includegraphics[width=0.8\textwidth]{messing-beidseitig-quadratisch.pdf}
  \caption{Meßwerte mit linearer Regression für den Messingstab}
  \label{fig:messing}
\end{figure}

\begin{table}
  \centering
 
  \begin{tabular}{SSS}
    \toprule
    {$x/\si{\centi\metre}$} & 
    {$D_v/\si{\micro\metre}$} & 
    {$D_n/\si{\micro\metre} $} \\
    \midrule
    3.0 &  1010 &   1040 \\
    8.0 &  1080 &   1310 \\
    13.0&  1170 &   1640 \\
    18.0&  1230 &   1940 \\
    20.0&  1230 &   2040 \\
    22.0&  1250 &   2140 \\
    24.0&  1250 &   2220 \\
    26.0&  1280 &   2280 \\
    31.6&  2880 &   3910 \\
    33.0&  2860 &   3900 \\
    35.0&  2820 &   3810 \\
    37.0&  2770 &   3750 \\
    39.0&  2750 &   3680 \\
    44.0&  2680 &   3370 \\
    49.0&  2620 &   3020 \\
    54.0&  2600 &   2690 \\
    \bottomrule
  \end{tabular}
  \caption{Meßwerte für den Messingstab}
  \label{tab:messing}
\end{table}

\begin{table}
  \centering
  \begin{tabular}{S}
    \toprule
    {$d/\si{\milli\metre}$}\\
    \midrule
    9.979 \\
    9.986 \\
    9.987 \\
    9.975 \\
    9.973 \\
    9.981 \\
    9.983 \\
    9.997 \\
    9.974 \\
    9.998 \\
    \bottomrule
  \end{tabular}
  \caption{Meßwerte für den Durchmesser des Messingstabes}
  \label{tab:messing-durchmesser}
\end{table}
