\section{Diskussion}

Nach \textcite{demtroeder:exp1} liegt der Wert für der Elastizitätsmodul
von Aluminium bei \SI{71}{\kilo\newton\per\milli\metre^2} und der von
Stahl bei \SI{200}{\kilo\newton\per\milli\metre^2}.  Der in diesem
Protokoll bestimmte Wert für der Elastizitätsmodul von Aluminium beträgt
ca. \SI{73}{\kilo\newton\per\milli\metre^2} und weicht somit um
ca. \SI{2.1}{\percent} vom gefundenen Literaturwert ab.  Bei Stahl
ergibt sich somit bei einem bestimmten Wert von
ca. \SI{1078}{\kilo\newton\per\milli\metre^2} eine Abweichung von
\SI{439}{\percent}. Dieser Wert weicht so stark vom Literaturwert ab,
dass man es nur damit erklären kann, dass während der Vermessung des
Stabes die Messwerte falsch abgelesen wurden.  Der Wert für der
Elastizitätsmodul von Messing liegt nach \textcite{peter-brehm} zwischen
\SI{78}{\kilo\newton\per\milli\metre^2} und
\SI{123}{\kilo\newton\per\milli\metre^2}. Der Mittelwert der beiden
gefundenen Werte für Messing beträgt
ca. \SI{151,6}{\kilo\newton\per\milli\metre^2}. Somit ergibt sich bei
Betrachtung des größten Literaturwertes für der Elastizitätsmodul von
Messing eine Abweichung von ca. \SI{23.3}{\percent} Die relativen Fehler
in Prozent berechnen sich aus den gemessenen Werten $x$ und den
Literaturwerten $x_L$ nach Formel \eqref{eq:rel_fehler}.
%
\begin{equation}
\label{eq:rel_fehler}
\Delta x = 100 \cdot \frac{|x - x_L|}{x_L}    
\end{equation}
