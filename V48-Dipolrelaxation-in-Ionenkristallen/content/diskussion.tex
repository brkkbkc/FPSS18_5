\section{Diskussion}
\label{sec:Diskussion}
%  beide relaxationszeiten sollten gleih sein, ist ja selbe probe.
  Im Experiment wurde der Depolarisationsstrom $i(T)$ des Ionenkristalls KBr(Sr) als Funktion der Temperatur $T$ für zwei verschiedene Heizraten $b_2=\SI{1,83}{\kelvin}/\si{min}$ und $b_2=\SI{1,44}{\kelvin}/\si{min}$ aufgenommen.
  Aus den Messdaten wurde für beide Heizraten auf den zwei in der Theorie beschriebenen Wegen die Aktivierungsenergie $W$ und danach die charakteristische Relaxationszeit $\tau_0$ bestimmt, um die Relaxationszeit $\tau(T)$ von KBr(Sr) als Funktion der Temperatur der Form
  \begin{equation}
    \tau(T)=\tau_0\text{e}^{\frac{W}{k_BT}}
  \end{equation}
  zu erhalten.
  Da für beide Heizraten dasselbe Material untersucht wurde, sollten alle bestimmten Aktivierungsenergien bzw. charakteristische Relaxationszeiten übereinstimmen.\\
  \\ Die berechneten Werte für $W$ bei Methode 1 sind:
   \begin{center}
  $W_1=\SI{1,102}{\electronvolt}$
  \end{center}

   \begin{center}
  $W_2=\SI{1,020}{\electronvolt}$
  \end{center}
  (7\% Abweichung) und die Werte aus Methode 2 sind:
  \begin{center}
  $W_1=\SI{0,870}{\electronvolt}$
  \end{center}

  \begin{center}
  $W_2=\SI{1,156}{\electronvolt}$
  \end{center}
  (24\% Abweichung).
  Bei Vergleich der Mittelwerte aus jewails den zwei Werten von Methode 1 und 2
  \begin{equation}
    \overline{W}_{Methode1}=\SI{1,061}{\electronvolt}
  \end{equation}
  \begin{equation}
    \overline{W}_{Methode2}=\SI{1,013}{\electronvolt}
  \end{equation}
  zeigt einer erstaunlicherweise gute übereinstimmung, mit einer Abweichung von 5\%.

  Obwohl den Werten aus Methode 2 mehr zu vertrauen sein sollte, da hier, im Gegensatz zu Methode 1, der gesamte Graphenverlauf zur Berechnung der Aktivierungsenergie $W$ berücksichtigt wurde, weichen die Werte aus Methode 1 deutlich weniger voneinander ab.
  Der Verlgleich mit den Literaturwerten von $W$
  \begin{equation}
    W_{lit}=\SI{0,66 \pm 0,1}{\electronvolt}
  \end{equation}
  zeigt aber, dass der kleinere der beiden Werte aus Methode 2 mit einer Abweichung von 17\% am nächsten an dem Literaturwert liegt.
  Die grö"ste Abweichung zum Literaturwert hat mit 160\% der Wert $W_2$ für die zweite Heizrate aus Methode 2.

  Die Betrachtung der Heizraten in Abbildung (\ref{fig:plot0}) zeigt, dass kaum die Messdaten einer Heizrate bevorzugt werden sollten, da beide Heizraten während des Experimentes sehr konstant gehalten werden konnten.

Insgesamt stimmen die Messwerte bzw. deren Mittelwerte untereinander relativ gut überein, es liegen jedoch große Abweichungen zum Theoriewert vor.\\
  \\ Anders ist es bei den Werten für die charakteristische Relaxationszeit $\tau_0$ (Methode 1 und Methode 2)
  \begin{align}
    \tau_{0,1}&=\SI{9,33e-22}{\second}\\
    \tau_{0,2}&=\SI{3,62e-20}{\second}
  \end{align}
 und
 \begin{align}
   \tau_{0,1}&=\SI{4,02e-17}{\second}\\
   \tau_{0,2}&=\SI{7,24e-23}{\second} \; .
 \end{align}
 Die Temperaturen des lokalen Maximums von $i(T)$, $T_{max,1}=\SI{258,55}{\kelvin}$ und $T_{max,2} = \SI{256,85}{\kelvin}$, stimmen nahezu überein.
  Trotzdem liegen selbst bei $\tau_0$ für die fast übereinstimmenden Werte von $W$ aus Methode 1 Abweichungen von mehreren Gr"o"senordnungen vor, da schon sehr kleine Änderungen bei der zur Berechnung verwendeten Formel (\ref{tau_0}) gro"se Änderungen im Wert zu Folge haben.
  Ein Blick auf den Literaturwert
  \begin{equation}
    \tau_{0_{lit}}=\SI{4 \pm 2e-14}{\second}
  \end{equation} zeigt, dass die experimentellen Werte nicht nur untereinander sondern auch zum Literaturwert eine sehr gro"se Abweichung von mehreren Groö"senordnungen aufweisen.
  Es muss davon ausgegangen werden, dass nur extrem genaue Messwerte für $W$ und $b$ ein genaueres Ergebnis zulassen.
